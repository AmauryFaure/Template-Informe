\begin{sourcecode}[]{c}{Codigo en c}
#include <stdio.h>
int main(){
	int i, j, rows;
	
	printf("Enter number of rows: ");
	scanf("%d",&rows);
	
	for(i=1; i<=rows; ++i){
		for(j=1; j<=i; ++j){
			printf("* ");
		}
		printf("\n");
	}
	return 0;
}
\end{sourcecode}

\begin{sourcecode}[]{csharp}{Codigo en C\#}
/*
* C# Program to Get a Number and Display the Sum of the Digits 
*/
using System;
using System.Collections.Generic;
using System.Linq;
using System.Text;

namespace Program
{
	class Program
	{
		static void Main(string[] args)
		{
			int num, sum = 0, r;
			Console.WriteLine("Enter a Number : ");
			num = int.Parse(Console.ReadLine());
			while (num != 0)
			{
				r = num % 10;
				num = num / 10;
				sum = sum + r;
			}
			Console.WriteLine("Sum of Digits of the Number : "+sum);
			Console.ReadLine();
			
		}
	}
}
\end{sourcecode}

\newpage
\begin{sourcecode}{c++}{}
#include <iostream>
using namespace std;

int main()
{
	int n, sum = 0;
	
	cout << "Enter a positive integer: ";
	cin >> n;
	
	for (int i = 1; i <= n; ++i) {
		sum += i;
	}
	
	cout << "Sum = " << sum;
	return 0;
}
\end{sourcecode}

\begin{sourcecode}{docker}{DOCC}
version: '2'
services:
web:
build: .
ports:
- "5000:5000"
volumes:
- .:/code
- logvolume01:/var/log
links:
- redis
redis:
image: redis
volumes:
logvolume01: {}
\end{sourcecode}

\begin{sourcecode}[\label{codigo-matlab}]{matlab}{Ejemplo en Matlab.}
% Se crea gráfico
f = figure(1);
hold on;
movegui(f, 'center');
xlabel('td/Tn'); ylabel('FAD=Umax/Uf0');
title('Espectro de pulso de desplazamiento');

for j = 1:length(BETA)
	fad = ones(1, NDATOS); % Arreglo para el FAD, uno para cada r
	for i = 1:NDATOS
		[t, u_t, ~, ~] = main(BETA(j), r(i), M, K, F0, 0);
		fad(i) = max(abs(u_t)) / uf0;
	end
mx = find(fad == max(fad(:)));
fprintf('BETA=%.2f, MAX: FAD=%.3f, TD/TN=%.3f\n', BETA(j), fad(mx), tdtn(mx));
plot(tdtn, fad, 'DisplayName', strcat('\beta=', sprintf('%.2f', BETA(j))));
end	
\end{sourcecode}

\begin{sourcecode}[]{xml}{Ejemplo xml.}
<?xml version="1.0" encoding="utf-8"?>
<xs:schema attributeFormDefault="unqualified" elementFormDefault="qualified"
xmlns:xs="http://www.w3.org/2001/XMLSchema">
	<xs:element name="points">
		<xs:complexType>
			<xs:sequence>
				<xs:element maxOccurs="unbounded" name="point">
					<xs:complexType>
						<xs:attribute name="x" type="xs:unsignedShort" use="required" />
						<xs:attribute name="y" type="xs:unsignedShort" use="required" />
					</xs:complexType>
				</xs:element>
			</xs:sequence>
		</xs:complexType>
	</xs:element>
</xs:schema>
\end{sourcecode}

\begin{sourcecode}[\label{codigo-python}]{python}{Ejemplo en Python.}
import numpy as np

def incmatrix(genl1, genl2):
m = len(genl1)
n = len(genl2)
M = None # Comentario 1
VT = np.zeros((n*m, 1), int) # Comentario 2
\end{sourcecode}

\begin{sourcecode}[\label{codigo-java}]{java}{Ejemplo en Java.}
import java.io.IOException; 
import javax.servlet.*;

// Hola mundo
public class Hola extends GenericServlet {
	public void service(ServletRequest request, ServletResponse response)
	throws ServletException, IOException{
		response.setContentType("text/html");
		PrintWriter pw = response.getWriter();
		pw.println("Hola, mundo!");
		pw.close();
	}
}
\end{sourcecode}