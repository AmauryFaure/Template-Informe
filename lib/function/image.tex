% Template:     Informe/Reporte LaTeX
% Documento:    Funciones para insertar imágenes
% Versión:      4.7.2 (22/11/2017)
% Codificación: UTF-8
%
% Autor: Pablo Pizarro R.
%        Facultad de Ciencias Físicas y Matemáticas
%        Universidad de Chile
%        pablo.pizarro@ing.uchile.cl, ppizarror.com
%
% Manual template: [http://latex.ppizarror.com/Template-Informe/]
% Licencia MIT:    [https://opensource.org/licenses/MIT/]

\newcommand{\insertimage}[4][]{
	% Insertar una imagen
	% 	#1	Label (opcional)
	%	#2	Dirección de la imagen
	%	#3	Parámetros de la imagen
	%	#4	Leyenda de la imagen (opcional)
	\insertimageboxed[#1]{#2}{#3}{0}{#4}
}

\newcommand{\insertimageboxed}[5][]{
	% Insertar una imagen con recuadro
	% 	#1	Label (opcional)
	%	#2	Dirección de la imagen
	%	#3	Parámetros de la imagen
	%	#4	Ancho de la línea (en pt)
	%	#5	Leyenda de la imagen (opcional)
	\emptyvarerr{\insertimageboxed}{#2}{Direccion de la imagen no definida}
	\emptyvarerr{\insertimageboxed}{#3}{Parametros de la imagen no definidos}
	\emptyvarerr{\insertimageboxed}{#4}{Ancho de la linea no definido}
	\vspace{\margintopimages cm}
	\begin{figure}[H]
		\begingroup
			\setlength{\fboxsep}{0 pt}
			\setlength{\fboxrule}{#4 pt}
			\centering
			\fbox{\includegraphics[#3]{#2}}
		\endgroup
		\ifx\hfuzz#5\hfuzz
			\vspace{\captionlessmarginimage cm}
		\else
			\hspace{0cm}\caption{#5 #1}
		\fi
	\end{figure}
	\vspace{\marginbottomimages cm}
}

\newcommand{\insertdoubleimage}[8][]{
	% Insertar una imagen doble
	% 	#1	Label (opcional)
	%	#2	Dirección de la imagen 1
	%	#3	Parámetros de la imagen 1
	%	#4	Leyenda de la imagen 1
	%	#5	Dirección de la imagen 2
	%	#6	Parámetros de la imagen 2
	%	#7	Leyenda de la imagen 2
	%	#8	Leyenda de la imagen doble (opcional)
	\emptyvarerr{\insertdoubleimage}{#2}{Direccion de la imagen 1 no definida}
	\emptyvarerr{\insertdoubleimage}{#3}{Parametros de la imagen 1 no definidos}
	\emptyvarerr{\insertdoubleimage}{#5}{Direccion de la imagen 2 no definida}
	\emptyvarerr{\insertdoubleimage}{#6}{Parametros de la imagen 2 no definidos}
	\vspace{\margintopimages cm}
	\captionsetup{margin=0.45cm}
	\begin{figure}[H] \centering
		\subfloat[#4]{
			\includegraphics[#3]{#2}}
		\hspace{0.2cm}
		\subfloat[#7]{
			\includegraphics[#6]{#5}}
		\setcaptionmargincm{\captionlrmargin}
		\ifx\hfuzz#8\hfuzz
			\vspace{\captionlessmarginimage cm}
		\else
			\caption{#8 #1}
		\fi
	\end{figure}
	\setcaptionmargincm{\captionlrmargin}
	\vspace{\marginbottomimages cm}
}

\newcommand{\insertdoubleeqimage}[7][]{
	% Insertar una imagen doble, igual propiedades
	% 	#1	Label (opcional)
	%	#2	Dirección de la imagen 1
	%	#3	Leyenda de la imagen 1
	%	#4	Dirección de la imagen 2
	%	#5	Leyenda de la imagen 2
	%	#6	Propiedades de las imágenes
	%	#7 	Leyenda de la imagen doble (opcional)
	\insertdoubleimage[#1]{#2}{#6}{#3}{#4}{#6}{#5}{#7}
}

\newcommand{\inserttripleimage}[8][]{
	% Insertar una imagen triple
	% 	#1	Label (opcional)
	%	#2	Dirección de la imagen 1
	%	#3	Parámetros de la imagen 1
	%	#4	Dirección de la imagen 2
	%	#5	Parámetros de la imagen 2
	%	#6	Dirección de la imagen 3
	%	#7	Parámetros de la imagen 3
	%	#8	Leyenda de la imagen triple (opcional)
	\emptyvarerr{\inserttripleimage}{#2}{Direccion de la imagen 1 no definida}
	\emptyvarerr{\inserttripleimage}{#3}{Parametros de la imagen 1 no definidos}
	\emptyvarerr{\inserttripleimage}{#4}{Direccion de la imagen 2 no definida}
	\emptyvarerr{\inserttripleimage}{#5}{Parametros de la imagen 2 no definidos}
	\emptyvarerr{\inserttripleimage}{#6}{Direccion de la imagen 3 no definida}
	\emptyvarerr{\inserttripleimage}{#7}{Parametros de la imagen 3 no definidos}
	\vspace{\margintopimages cm}
	\captionsetup{margin=0.45cm}
	\begin{figure}[H] \centering
		\subfloat[]{
			\includegraphics[#3]{#2}}
		\hspace{0.1cm}
		\subfloat[]{
			\includegraphics[#5]{#4}}
		\hspace{0.1cm}
		\subfloat[]{
			\includegraphics[#7]{#6}}
		\setcaptionmargincm{\captionlrmargin}
		\ifx\hfuzz#8\hfuzz
			\vspace{\captionlessmarginimage cm}
		\else
			\caption{#8 #1}
		\fi
	\end{figure}
	\setcaptionmargincm{\captionlrmargin}
	\vspace{\marginbottomimages cm}
}

\newcommand{\inserttripleeqimage}[6][]{
	% Insertar una imagen triple, igual propiedades
	% 	#1	Label (opcional)
	%	#2	Dirección de la imagen 1
	%	#3	Dirección de la imagen 2
	%	#4	Dirección de la imagen 3
	%	#5	Propiedades de las imágenes
	%	#6	Leyenda de la imagen triple (opcional)
	\inserttripleimage[#1]{#2}{#5}{#3}{#5}{#4}{#5}{#6}
}

\newcommand{\insertquadimage}[7][]{
	% Insertar una imagen cuádruple, igual propiedades
	% 	#1	Label (opcional)
	%	#2	Dirección de la imagen 1
	%	#3	Dirección de la imagen 2
	%	#4	Dirección de la imagen 3
	%	#5	Dirección de la imagen 4
	%	#6	Propiedades de las imágenes
	%	#7	Leyenda de la imagen cuádruple (opcional)
	\emptyvarerr{\insertquadimage}{#2}{Direccion de la imagen 1 no definida}
	\emptyvarerr{\insertquadimage}{#3}{Direccion de la imagen 2 no definida}
	\emptyvarerr{\insertquadimage}{#4}{Direccion de la imagen 3 no definida}
	\emptyvarerr{\insertquadimage}{#5}{Direccion de la imagen 4 no definida}
	\emptyvarerr{\insertquadimage}{#6}{Propiedades de las imagenes no definidos}
	\vspace{\margintopimages cm}
	\captionsetup{margin=0.45cm}
	\begin{figure}[H] \centering
		\subfloat[]{
			\includegraphics[#6]{#2}}
		\hspace{0.1cm}
		\subfloat[]{
			\includegraphics[#6]{#3}}
		\hspace{0.1cm}
		\subfloat[]{
			\includegraphics[#6]{#4}}
		\hspace{0.1cm}
		\subfloat[]{
			\includegraphics[#6]{#5}}
		\setcaptionmargincm{\captionlrmargin}
		\ifx\hfuzz#7\hfuzz
			\vspace{\captionlessmarginimage cm}
		\else
			\caption{#7 #1}
		\fi
	\end{figure}
	\setcaptionmargincm{\captionlrmargin}
	\vspace{\marginbottomimages cm}
}

\newcommand{\insertpentaimage}[8][]{
	% Insertar una imagen quíntuple, igual propiedades
	% 	#1	Label (opcional)
	%	#2	Dirección de la imagen 1
	%	#3	Dirección de la imagen 2
	%	#4	Dirección de la imagen 3
	%	#5	Dirección de la imagen 4
	%	#6	Dirección de la imagen 5
	%	#7	Propiedades de las imágenes
	%	#8	Leyenda de la imagen quíntuple (opcional)
	\emptyvarerr{\insertpentaimage}{#2}{Direccion de la imagen 1 no definida}
	\emptyvarerr{\insertpentaimage}{#3}{Direccion de la imagen 2 no definida}
	\emptyvarerr{\insertpentaimage}{#4}{Direccion de la imagen 3 no definida}
	\emptyvarerr{\insertpentaimage}{#5}{Direccion de la imagen 4 no definida}
	\emptyvarerr{\insertpentaimage}{#6}{Direccion de la imagen 5 no definida}
	\emptyvarerr{\insertpentaimage}{#7}{Propiedades de las imagenes no definidas}
	\vspace{\margintopimages cm}
	\captionsetup{margin=0.45cm}
	\begin{figure}[H] \centering
		\subfloat[]{
			\includegraphics[#7]{#2}}
		\hspace{0.1cm}
		\subfloat[]{
			\includegraphics[#7]{#3}}
		\hspace{0.1cm}
		\subfloat[]{
			\includegraphics[#7]{#4}}
		\hspace{0.1cm}
		\subfloat[]{
			\includegraphics[#7]{#5}}
		\hspace{0.1cm}
		\subfloat[]{
			\includegraphics[#7]{#6}}
		\setcaptionmargincm{\captionlrmargin}
		\ifx\hfuzz#8\hfuzz
			\vspace{\captionlessmarginimage cm}
		\else
			\caption{#8 #1}
		\fi
	\end{figure}
	\setcaptionmargincm{\captionlrmargin}
	\vspace{\marginbottomimages cm}
}

\newcommand{\inserthexaimage}[9][]{
	% Insertar una imagen con 6 imágenes, igual propiedades
	% 	#1	Label (opcional)
	%	#2	Dirección de la imagen 1
	%	#3	Dirección de la imagen 2
	%	#4	Dirección de la imagen 3
	%	#5	Dirección de la imagen 4
	%	#6	Dirección de la imagen 5
	%	#7	Dirección de la imagen 6
	%	#8	Propiedades de las imágenes
	%	#9	Leyenda de la imagen global (opcional)
	\emptyvarerr{\inserthexaimage}{#2}{Direccion de la imagen 1 no definida}
	\emptyvarerr{\inserthexaimage}{#3}{Direccion de la imagen 2 no definida}
	\emptyvarerr{\inserthexaimage}{#4}{Direccion de la imagen 3 no definida}
	\emptyvarerr{\inserthexaimage}{#5}{Direccion de la imagen 4 no definida}
	\emptyvarerr{\inserthexaimage}{#6}{Direccion de la imagen 5 no definida}
	\emptyvarerr{\inserthexaimage}{#7}{Direccion de la imagen 6 no definida}
	\emptyvarerr{\inserthexaimage}{#8}{Propiedades de las imagenes no definidas}
	\vspace{\margintopimages cm}
	\captionsetup{margin=0.45cm}
	\begin{figure}[H] \centering
		\subfloat[]{
			\includegraphics[#8]{#2}}
		\hspace{0.1cm}
		\subfloat[]{
			\includegraphics[#8]{#3}}
		\hspace{0.1cm}
		\subfloat[]{
			\includegraphics[#8]{#4}}
		\hspace{0.1cm}
		\subfloat[]{
			\includegraphics[#8]{#5}}
		\hspace{0.1cm}
		\subfloat[]{
			\includegraphics[#8]{#6}}
		\hspace{0.1cm}
		\subfloat[]{
			\includegraphics[#8]{#7}}
		\setcaptionmargincm{\captionlrmargin}
		\ifx\hfuzz#9\hfuzz
			\vspace{\captionlessmarginimage cm}
		\else
			\caption{#9 #1}
		\fi
	\end{figure}
	\setcaptionmargincm{\captionlrmargin}
	\vspace{\marginbottomimages cm}
}

\newcommand{\insertimageleft}[4][]{
	% Insertar una imagen a la izquierda
	% 	#1	Label (opcional)
	%	#2	Dirección de la imagen
	%	#3	Ancho de la imagen (en textwidth)
	%	#4	Leyenda de la imagen (opcional)
	\insertimageleftboxed[#1]{#2}{#3}{0}{#4}
}

\newcommand{\insertimageleftboxed}[5][]{
	% Insertar una imagen a la izquierda
	% 	#1	Label (opcional)
	%	#2	Dirección de la imagen
	%	#3	Ancho de la imagen (en textwidth)
	%	#4	Ancho de la línea (en pt)
	%	#5	Leyenda de la imagen (opcional)
	\emptyvarerr{\insertimageleftboxed}{#2}{Direccion de la imagen no definida}
	\emptyvarerr{\insertimageleftboxed}{#3}{Ancho de la imagen no definido}
	\emptyvarerr{\insertimageleftboxed}{#4}{Ancho de la linea no definido}
	\begin{wrapfigure}{l}{#3\textwidth}
		\setcaptionmargincm{0}
		\ifthenelse{\equal{\figurecaptiontop}{true}}{}{
			\vspace{\marginfloatimages pt}}
		\begingroup
			\setlength{\fboxsep}{0 pt}
			\setlength{\fboxrule}{#4 pt}
			\centering
			\fbox{\includegraphics[width=\linewidth]{#2}}
		\endgroup
		\ifx\hfuzz#5\hfuzz
			\vspace{\captionlessmarginimage cm}
		\else
			\caption{#5 #1}
		\fi
	\end{wrapfigure}
	\setcaptionmargincm{\captionlrmargin}
}


\newcommand{\insertimageleftline}[5][]{
	% Insertar una imagen a la izquierda
	% 	#1	Label (opcional)
	%	#2	Dirección de la imagen
	%	#3	Ancho de la imagen (en textwidth)
	%	#4	Altura en líneas de la imagen
	%	#5	Leyenda de la imagen (opcional)
	\insertimageleftboxed[#1]{#2}{#3}{0}{#4}
}

\newcommand{\insertimageleftlineboxed}[6][]{
	% Insertar una imagen a la izquierda
	% 	#1	Label (opcional)
	%	#2	Dirección de la imagen
	%	#3	Ancho de la imagen (en textwidth)
	%	#4	Altura en líneas de la imagen
	%	#5	Ancho de la línea (en pt)
	%	#6	Leyenda de la imagen (opcional)
	\emptyvarerr{\insertimageleftlineboxed}{#2}{Direccion de la imagen no definida}
	\emptyvarerr{\insertimageleftlineboxed}{#3}{Ancho de la imagen no definido}
	\emptyvarerr{\insertimageleftlineboxed}{#4}{Altura en lineas de la imagen flotante izquierda no definida}
	\emptyvarerr{\insertimageleftlineboxed}{#5}{Ancho de la linea no definido}
	\begin{wrapfigure}[#4]{l}{#3\textwidth}
		\setcaptionmargincm{0}
		\ifthenelse{\equal{\figurecaptiontop}{true}}{}{
			\vspace{\marginfloatimages pt}}
		\begingroup
			\setlength{\fboxsep}{0 pt}
			\setlength{\fboxrule}{#5 pt}
			\centering
			\fbox{\includegraphics[width=\linewidth]{#2}}
		\endgroup
		\ifx\hfuzz#6\hfuzz
			\vspace{\captionlessmarginimage cm}
		\else
			\caption{#6 #1}
		\fi
	\end{wrapfigure}
	\setcaptionmargincm{\captionlrmargin}
}

\newcommand{\insertimageright}[4][]{
	% Insertar una imagen a la derecha
	% 	#1	Label (opcional)
	%	#2	Dirección de la imagen
	%	#3	Ancho de la imagen (en textwidth)
	%	#4	Leyenda de la imagen (opcional)
	\insertimagerightboxed[#1]{#2}{#3}{0}{#4}
}

\newcommand{\insertimagerightboxed}[5][]{
	% Insertar una imagen a la derecha
	% 	#1	Label (opcional)
	%	#2	Dirección de la imagen
	%	#3	Ancho de la imagen (en textwidth)
	%	#4	Ancho de la línea (en pt)
	%	#5	Leyenda de la imagen (opcional)
	\emptyvarerr{\insertimagerightboxed}{#2}{Direccion de la imagen no definida}
	\emptyvarerr{\insertimagerightboxed}{#3}{Ancho de la imagen no defindo}
	\emptyvarerr{\insertimagerightboxed}{#4}{Ancho de la linea no definido}
	\begin{wrapfigure}{r}{#3\textwidth}
		\setcaptionmargincm{0}
		\ifthenelse{\equal{\figurecaptiontop}{true}}{}{
			\vspace{\marginfloatimages pt}}
		\begingroup
			\setlength{\fboxsep}{0 pt}
			\setlength{\fboxrule}{#4 pt}
			\centering
			\fbox{\includegraphics[width=\linewidth]{#2}}
		\endgroup
		\ifx\hfuzz#5\hfuzz
			\vspace{\captionlessmarginimage cm}
		\else
			\caption{#5 #1}
		\fi
	\end{wrapfigure}
	\setcaptionmargincm{\captionlrmargin}
}

\newcommand{\insertimagerightline}[5][]{
	% Insertar una imagen a la derecha
	% 	#1	Label (opcional)
	%	#2	Dirección de la imagen
	%	#3	Ancho de la imagen (en textwidth)
	%	#4	Altura en líneas de la imagen
	%	#5	Leyenda de la imagen (opcional)
	\insertimagerightlineboxed[#1]{#2}{#3}{#4}{0}{#5}
}

\newcommand{\insertimagerightlineboxed}[6][]{
	% Insertar una imagen a la derecha
	% 	#1	Label (opcional)
	%	#2	Dirección de la imagen
	%	#3	Ancho de la imagen (en textwidth)
	%	#4	Altura en líneas de la imagen
	%	#5	Ancho de la línea (en pt)
	%	#6	Leyenda de la imagen (opcional)
	\emptyvarerr{\insertimagerightlineboxed}{#2}{Direccion de la imagen no definida}
	\emptyvarerr{\insertimagerightlineboxed}{#3}{Ancho de la imagen no defindo}
	\emptyvarerr{\insertimagerightlineboxed}{#4}{Altura en lineas de la imagen flotante derecha no definida}
	\emptyvarerr{\insertimagerightlineboxed}{#5}{Ancho de la linea no definido}
	\begin{wrapfigure}[#4]{r}{#3\textwidth}
		\setcaptionmargincm{0}
		\ifthenelse{\equal{\figurecaptiontop}{true}}{}{
			\vspace{\marginfloatimages pt}}
		\begingroup
			\setlength{\fboxsep}{0 pt}
			\setlength{\fboxrule}{#5 pt}
			\centering
			\fbox{\includegraphics[width=\linewidth]{#2}}
		\endgroup
		\ifx\hfuzz#6\hfuzz
			\vspace{\captionlessmarginimage cm}
		\else
			\caption{#6 #1}
		\fi
	\end{wrapfigure}
	\setcaptionmargincm{\captionlrmargin}
}
