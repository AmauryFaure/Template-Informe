% Template:     Informe/Reporte LaTeX
% Documento:    Funciones para insertar elementos
% Versión:      4.2.1 (07/07/2017)
% Codificación: UTF-8
%
% Autor: Pablo Pizarro R.
%        Facultad de Ciencias Físicas y Matemáticas
%        Universidad de Chile
%        pablo.pizarro@ing.uchile.cl, ppizarror.com
%
% Manual template: [http://ppizarror.com/Template-Informe/]
% Licencia MIT:    [https://opensource.org/licenses/MIT/]

\newcommand{\newp}{
	% Inserta nueva línea
	\hbadness=10000 \vspace{\defaultnewlinesize pt} \par
}

\newcommand{\newpar}[1]{
	% Insertar párrafo
	% 	#1	Párrafo
	\hbadness=10000 #1 \newp
}

\newcommand{\newparnl}[1]{
	% Insertar párrafo sin nueva línea al final
	% 	#1	Párrafo
	#1 \par
}

\newcommand{\lpow}[2]{
	% Insertar sub-índice, a_b
	% 	#1	Elemento inferior (a)
	%	#2	Elemento superior (b)
	\ensuremath{{#1}_{#2}}
}

\newcommand{\pow}[2]{
	% Insertar elevado, a^b
	% 	#1	Elemento inferior (a)
	%	#2	Elemento superior (b)
	\ensuremath{{#1}^{#2}}
}

\newcommand{\fracpartial}[2]{
	% Fracción de derivadas parciales af/ax
	% 	#1	Función a derivar (f)
	%	#2	Variable a derivar (x)
	\ensuremath{\pdv{#1}{#2}}
}

\newcommand{\fracdpartial}[2]{
	% Fracción de derivadas parciales dobles a^2f/ax^2
	% 	#1	Función a derivar (f)
	%	#2	Variable a derivar (x)
	\ensuremath{\pdv[2]{#1}{#2}}
}

\newcommand{\fracnpartial}[3]{
	% Fracción de derivadas parciales en n, a^nf/ax^n
	% 	#1	Función a derivar (f)
	%	#2	Variable a derivar (x)
	%	#3	Orden (n)
	\ensuremath{\pdv[#3]{#1}{#2}}
}

\newcommand{\fracderivat}[2]{
	% Fracción de derivadas df/dx
	% 	#1	Función a derivar (f)
	%	#2	Variable a derivar (x)
	\ensuremath{\dv{#1}{#2}}
}

\newcommand{\fracdderivat}[2]{
	% Fracción de derivadas dobles d^2/dx^2
	% 	#1	Función a derivar (f)
	%	#2	Variable a derivar (x)
	\ensuremath{\dv[2]{#1}{#2}}
}

\newcommand{\fracnderivat}[3]{
	% Fracción de derivadas en n d^nf/dx^n
	% 	#1	Función a derivar (f)
	%	#2	Variable a derivar (x)
	%	#3	Orden de la derivada (n)
	\ensuremath{\dv[#3]{#1}{#2}}
}

\newcommand{\topequal}[2]{
	% Llave superior de equivalencia
	% 	#1	Elemento a igualar
	%	#2	Igualdad
	\ensuremath{\overbrace{#1}^{\mathclap{#2}}}
}

\newcommand{\underequal}[2]{
	% Llave inferior de equivalencia
	% 	#1	Elemento a igualar
	%	#2	Igualdad
	\ensuremath{\underbrace{#1}_{\mathclap{#2}}}
}

\newcommand{\topsequal}[2]{
	% Rectángulo superior de equivalencia
	% 	#1	Elemento a igualar
	%	#2	Igualdad
	\ensuremath{\overbracket{#1}^{\mathclap{#2}}}
}

\newcommand{\undersequal}[2]{
	% Rectángulo inferior de equivalencia
	% 	#1	Elemento a igualar
	%	#2	Igualdad
	\ensuremath{\underbracket{#1}_{\mathclap{#2}}}
}

\newcommand{\itemresize}[2]{
	% Redimensiona un ítem en textwidth
	% 	#1	Tamaño del nuevo objeto (En textwidth)
	%	#2	Objeto a redimensionar
	\emptyvarerr{\itemresize}{#1}{Tamano del nuevo objeto no definido}
	\emptyvarerr{\itemresize}{#2}{Objeto a redimensionar no definido}
	\resizebox{#1\textwidth}{!}{#2}
}

\newcommand{\insertemptypage}{
	% Crea una página vacía
	\newpage
	\setcounter{templatepagecounter}{\thepage}
	\pagenumbering{gobble}
	\null
	\thispagestyle{empty}
	\newpage
	\pagenumbering{arabic}
	\setcounter{page}{\thetemplatepagecounter}
}

% Inserta un texto entre comillas
\newcommand{\quotes}[1]{\enquote*{#1}}

% Inserta un email con un link cliqueable
\newcommand{\insertemail}[1]{\href{mailto:#1}{\texttt{#1}}}
