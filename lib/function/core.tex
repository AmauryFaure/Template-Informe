% Template:     Informe/Reporte LaTeX
% Documento:    Funciones del núcleo del template
% Versión:      4.3.5 (30/07/2017)
% Codificación: UTF-8
%
% Autor: Pablo Pizarro R.
%        Facultad de Ciencias Físicas y Matemáticas
%        Universidad de Chile
%        pablo.pizarro@ing.uchile.cl, ppizarror.com
%
% Manual template: [http://latex.ppizarror.com/Template-Informe/]
% Licencia MIT:    [https://opensource.org/licenses/MIT/]

\newcommand{\throwerror}[2]{
	% Lanza un mensaje de error
	% 	#1	Función del error
	%	#2	Mensaje
	\errmessage{LaTeX Error: \noexpand#1 #2 (linea \the\inputlineno)}
	\stop
}

\newcommand{\throwwarning}[1]{
	% Lanza un mensaje de advertencia
	%	#1	Mensaje
	\errmessage{LaTeX Warning: #1 (linea \the\inputlineno)}
}

\newcommand{\throwbadconfig}[3]{
	% Lanza un mensaje de error indicando mala configuración
	%	#1	Mensaje de error
	% 	#2	Configuración usada
	%	#3	Valores esperados
	\errmessage{LaTeX Warning: #1 \noexpand #2=#2. Valores esperados: #3}
	\stop
}

\newcommand{\throwbadconfigondoc}[3]{
	% Lanza un mensaje de error indicando mala configuración dentro de begin{document}
	%	#1	Mensaje de error
	% 	#2	Configuración usada
	%	#3	Valores esperados
	\errmessage{#1 \noexpand #2=#2. Valores esperados: #3}
	\stop
}

\newcommand{\checkvardefined}[1]{
	% Comprueba si una variable está definida
	%	#1	Variable
	\ifthenelse{\isundefined{#1}}{
		\errmessage{LaTeX Warning: Variable \noexpand#1 no definida}
		\stop
	}{}
}

\newcommand{\emptyvarerr}[3]{
	% Lanza un mensaje de error si una variable no ha sido definida
	% 	#1	Función del error
	%	#2	Variable
	%	#3	Mensaje
	\ifx\hfuzz#2\hfuzz
		\errmessage{LaTeX Warning: \noexpand#1 #3 (linea \the\inputlineno)}
	\fi
}

\newcommand{\setcaptionmargincm}[1]{
	% Cambiar el margen de los caption
	% 	#1	Margen en centímetros
	\captionsetup{margin=#1cm}
}

\newcommand{\setpagemargincm}[4]{
	% Cambia márgenes de las páginas [cm]
	% 	#1	Margen izquierdo
	%	#2	Margen superior
	%	#3	Margen derecho
	%	#4	Margen inferior
	\newgeometry{left=#1cm, top=#2cm, right=#3cm, bottom=#4cm}
}

\newcommand{\changemargin}[2]{
	% Cambia los márgenes del documento
	%	#1 Margen izquierdo
	%	#2 Margen derecho
	\emptyvarerr{\changemargin}{#1}{Margen izquierdo no definido}
	\emptyvarerr{\changemargin}{#2}{Margen derecho no definido}
	\list{}{\rightmargin#2\leftmargin#1}\item[]
}
\let\endchangemargin=\endlist
