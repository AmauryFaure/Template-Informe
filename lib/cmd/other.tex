% Template:     Informe/Reporte LaTeX
% Documento:    Funciones para insertar elementos
% Versión:      5.1.4 (15/05/2018)
% Codificación: UTF-8
%
% Autor: Pablo Pizarro R. @ppizarror
%        Facultad de Ciencias Físicas y Matemáticas
%        Universidad de Chile
%        pablo.pizarro@ing.uchile.cl, ppizarror.com
%
% Manual template: [http://latex.ppizarror.com/Template-Informe/]
% Licencia MIT:    [https://opensource.org/licenses/MIT/]

% Insertar párrafo
\newcommand{\newp}{
	\hbadness=10000 \vspace{\defaultnewlinesize pt} \par
}

% Insertar párrafo
% 	#1	Párrafo
\newcommand{\newpar}[1]{
	\hbadness=10000 #1 \newp
}

% Insertar párrafo sin nueva línea al final
% 	#1	Párrafo
\newcommand{\newparnl}[1]{
	#1 \par
}

% Redimensiona un ítem en textwidth
% 	#1	Tamaño del nuevo objeto (En textwidth)
%	#2	Objeto a redimensionar
\newcommand{\itemresize}[2]{
	\emptyvarerr{\itemresize}{#1}{Tamano del nuevo objeto no definido}
	\emptyvarerr{\itemresize}{#2}{Objeto a redimensionar no definido}
	\resizebox{#1\textwidth}{!}{#2}
}

% Crea una página vacía
\newcommand{\insertemptypage}{
	\newpage
	\setcounter{templatepagecounter}{\thepage}
	\pagenumbering{gobble}
	\null
	\thispagestyle{empty}
	\newpage
	\pagenumbering{arabic}
	\setcounter{page}{\thetemplatepagecounter}
}

% Inserta un texto entre comillas
\newcommand{\quotes}[1]{\enquote*{#1}}

% Inserta un email con un link cliqueable
\newcommand{\insertemail}[1]{\href{mailto:#1}{\texttt{#1}}}
