% Template:     Informe/Reporte LaTeX
% Documento:    Funciones para crear columnas con contenido
% Versión:      6.1.1 (05/12/2018)
% Codificación: UTF-8
%
% Autor: Pablo Pizarro R. @ppizarror
%        Facultad de Ciencias Físicas y Matemáticas
%        Universidad de Chile
%        pablo.pizarro@ing.uchile.cl, ppizarror.com
%
% Manual template: [https://latex.ppizarror.com/Template-Informe/]
% Licencia MIT:    [https://opensource.org/licenses/MIT/]

% Crea dos columnas con contenido
%	#1 	Dimensión de la primera columna (En textwidth)
%	#2	Dimensión de la segunda columna (En textwidth)
%	#3	Distancia entre columnas
%	#4 	Contenido de la primera columna
%	#5	Contenido de la segunda columna
\newcommand{\createtwocolumn}[5]{
	\setcaptionmargincm{0}
	\begin{flushleft}
		\hspace{0cm}
		\begin{tabular}{c}
			\hspace{-0.34cm}
			\begin{minipage}[t]{#1\linewidth}
				\vspace{-2em}\nobreak~ #4
			\end{minipage}
			\hspace{\columnhspace cm}
			\hspace{#3}
			\begin{minipage}[t]{#2\linewidth}
				\vspace{-2em}\nobreak~ #5
			\end{minipage}
			\\
		\end{tabular}
		~
	\end{flushleft}
	\setcaptionmargincm{\captionlrmargin}
}

% Crea dos columnas con contenido y margen a la izquierda
%	#1 	Dimensión de la primera columna (En textwidth)
%	#2	Dimensión de la segunda columna (En textwidth)
%	#3	Margen izquierdo
%	#4	Distancia entre columnas
%	#5 	Contenido de la primera columna
%	#6	Contenido de la segunda columna
\newcommand{\createtwocolumnl}[6]{
	\createthreecolumn{0.001}{#1}{#2}{#3}{#4}{~}{#5}{#6}
}

% Crea dos columnas idénticas que usan la mitad del ancho de la página
%	#1 	Contenido de la primera columna
%	#2	Contenido de la segunda columna
\newcommand{\createhalfcolumn}[2]{
	\createtwocolumn{0.5}{0.5}{0cm}{#1}{#2}
}

% Crea dos columnas con contenido centrado
%	#1 	Dimensión de la primera columna (En textwidth)
%	#2	Dimensión de la segunda columna (En textwidth)
%	#3	Distancia entre columnas
%	#4 	Contenido de la primera columna
%	#5	Contenido de la segunda columna
\newcommand{\createtwocolumnc}[5]{
	\createtwocolumn{#1}{#2}{#3}{\begin{center}#4\end{center}}{\begin{center}#5\end{center}}
}

% Crea dos columnas con contenido centrado y margen a la izquierda
%	#1 	Dimensión de la primera columna (En textwidth)
%	#2	Dimensión de la segunda columna (En textwidth)
%	#3	Margen izquierdo
%	#4	Distancia entre columnas
%	#5 	Contenido de la primera columna
%	#6	Contenido de la segunda columna
\newcommand{\createtwocolumncl}[6]{
	\createtwocolumnl{#1}{#2}{#3}{#4}{\begin{center}#5\end{center}}{\begin{center}#6\end{center}}
}

% Crea dos columnas idénticas que usan la mitad del ancho de la página, centradas
%	#1 	Contenido de la primera columna
%	#2	Contenido de la segunda columna
\newcommand{\createhalfcolumnc}[2]{
	\createtwocolumnc{0.493}{0.493}{0cm}{#1}{#2}
}

% Crea tres columnas con contenido
%	#1 	Dimensión de la primera columna (En textwidth)
%	#2	Dimensión de la segunda columna (En textwidth)
%	#3	Dimensión de la tercera columna (En textwidth)
%	#4	Distancia entre columna 1-2
%	#5	Distancia entre columna 2-3
%	#6 	Contenido de la primera columna
%	#7	Contenido de la segunda columna
%	#8	Contenido de la tercera columna
\newcommand{\createthreecolumn}[8]{
	\setcaptionmargincm{0}
	\begin{flushleft}
		\hspace{0cm}
		\begin{tabular}{l}
			\hspace{-0.34cm}
			\begin{minipage}[t]{#1\linewidth}
				\vspace{-2em}\nobreak~ #6
			\end{minipage}
			\hspace{\columnhspace cm}
			\hspace{#4}
			\begin{minipage}[t]{#2\linewidth}
				\vspace{-2em}\nobreak~ #7
			\end{minipage}
			\hspace{\columnhspace cm}
			\hspace{#5}
			\begin{minipage}[t]{#3\linewidth}
				\vspace{-2em}\nobreak~ #8
			\end{minipage}
			\\
		\end{tabular}
		~
	\end{flushleft}
	\setcaptionmargincm{\captionlrmargin}
}

% Crea tres columnas idénticas que usan los tercios de la página
%	#1 	Contenido de la primera columna
%	#2	Contenido de la segunda columna
%	#3	Contenido de la tercera columna
\newcommand{\createthirdcolumn}[3]{
	\createthreecolumn{0.3333}{0.3333}{0.333}{0cm}{0cm}{#1}{#2}{#3}
}

% Crea tres columnas con contenido centrado
%	#1 	Dimensión de la primera columna (En textwidth)
%	#2	Dimensión de la segunda columna (En textwidth)
%	#3	Dimensión de la tercera columna (En textwidth)
%	#4	Distancia entre columna 1-2
%	#5	Distancia entre columna 2-3
%	#6 	Contenido de la primera columna
%	#7	Contenido de la segunda columna
%	#8	Contenido de la tercera columna
\newcommand{\createthreecolumnc}[8]{
	\createthreecolumn{#1}{#2}{#3}{#4}{#5}{\begin{center}#6\end{center}}{\begin{center}#7\end{center}}{\begin{center}#8\end{center}}
}

% Crea tres columnas idénticas que usan los tercios de la página, centradas
%	#1 	Contenido de la primera columna
%	#2	Contenido de la segunda columna
%	#3	Contenido de la tercera columna
\newcommand{\createthirdcolumnc}[3]{
	\createthreecolumnc{0.3333}{0.3333}{0.3333}{0cm}{0cm}{#1}{#2}{#3}
}

% END