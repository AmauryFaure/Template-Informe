% Template:     Informe/Reporte LaTeX
% Documento:    Definición de entornos
% Versión:      5.7.1 (06/10/2018)
% Codificación: UTF-8
%
% Autor: Pablo Pizarro R. @ppizarror
%        Facultad de Ciencias Físicas y Matemáticas
%        Universidad de Chile
%        pablo.pizarro@ing.uchile.cl, ppizarror.com
%
% Manual template: [http://latex.ppizarror.com/Template-Informe/]
% Licencia MIT:    [https://opensource.org/licenses/MIT/]

% Crea una sección de referencias solo para bibtex
\newenvironment{references}{
	% Verifica configuraciones
	\ifthenelse{\equal{\stylecitereferences}{bibtex}}{
	}{
		\throwerror{\references}{Solo se puede usar entorno references con estilo citas \noexpand\stylecitereferences=bibtex}
	}
	\begingroup
	
	% Se configura las referencias como una sección
	\ifthenelse{\equal{\donumrefsection}{true}}{
		\section{\namereferences}
	}{
		\sectionanum{\namereferences}
	}
	\renewcommand{\section}[2]{}
	
	% Inicia la bibliografía
	\begin{thebibliography}{99}
	}
	{
	\end{thebibliography}

	% Termina el grupo
	\endgroup
}

% Crea una sección de anexos
\newenvironment{anexo}{
	% Inicia el grupo en nueva página y sección
	\begingroup
	\clearpage
	\phantomsection
	\ifthenelse{\equal{\showappendixsectitle}{true}}{
		\appendixpage}{
	}

	% Añade al índice marcadores
	\appendixtitleon
	\appendicestocpagenum
	\appendixtitletocon
	\bookmarksetup{
		numbered,
		openlevel=0
	}

	% Crea la sección
	\begin{appendices}
		\bookmarksetupnext{level=part}
		\ifthenelse{\equal{\showappendixsecindex}{true}}{}{
			\belowpdfbookmark{\nameappendixsection}{contents}
		}
		\setcounter{secnumdepth}{3}
		\setcounter{tocdepth}{3}
		\ifthenelse{\equal{\appendixindepobjnum}{true}}{
			\counterwithin{equation}{section}
			\counterwithin{figure}{section}
			\counterwithin{lstlisting}{section}
			\counterwithin{table}{section}}{
		}
	}{
	\end{appendices}

	% Reestablece índice marcador
	\bookmarksetupnext{level=0}
	\endgroup
}

% Inicia código fuente con parámetros
%	#1	Label (opcional)
%	#2	Estilo de código
%	#3	Parámetros
%	#4	Caption
\newcommand{\coreinitsourcecodep}[4]{
	\emptyvarerr{sourcecodep}{#2}{Estilo no definido}
	\checkvalidsourcecodestyle{#2}
	\ifthenelse{\equal{\showlinenumbers}{true}}{
		\rightlinenumbers}{
	}
	\lstset{
		backgroundcolor=\color{\sourcecodebgcolor}
	}
	\ifthenelse{\equal{\codecaptiontop}{true}}{
		\ifx\hfuzz#4\hfuzz
			\ifx\hfuzz#3\hfuzz
				\lstset{
					style=#2
				}
			\else
				\lstset{
					style=#2,
					#3
				}
			\fi
		\else
			\ifx\hfuzz#3\hfuzz
				\lstset{
					caption={#4 #1},
					captionpos=t,
					style=#2
				}
			\else
				\lstset{
					caption={#4 #1},
					captionpos=t,
					style=#2,
					#3
				}
			\fi
		\fi
	}{
		\ifx\hfuzz#4\hfuzz
			\ifx\hfuzz#3\hfuzz
				\lstset{
					style=#2
				}
			\else
				\lstset{
					style=#2,
					#3
				}
			\fi
		\else
			\ifx\hfuzz#3\hfuzz
				\lstset{
					caption={#4 #1},
					captionpos=b,
					style=#2
				}
			\else
				\lstset{
					caption={#4 #1},
					captionpos=b,
					style=#2,
					#3
				}
			\fi
		\fi	
	}
}

% Inserta código fuente con parámetros
%	#1	Label (opcional)
%	#2	Estilo de código
%	#3	Parámetros
%	#4	Caption
\lstnewenvironment{sourcecodep}[4][]{
	\coreinitsourcecodep{#1}{#2}{#3}{#4}
}{
	\ifthenelse{\equal{\showlinenumbers}{true}}{
		\leftlinenumbers}{
	}
}

% Importa código fuente desde un archivo con parámetros
%	#1	Label (opcional)
%	#2	Estilo de código
%	#3	Parámetros
%	#4  Archivo de código fuente
%	#5	Caption
\newcommand{\importsourcecodep}[5][]{
	\coreinitsourcecodep{#1}{#2}{#3}{#5}
	\inputlisting{#4}
	\ifthenelse{\equal{\showlinenumbers}{true}}{
		\leftlinenumbers}{
	}
}

% Inicia código fuente sin parámetros
%	#1	Label (opcional)
%	#2	Estilo de código
%	#3	Caption
\newcommand{\coreinitsourcecode}[3]{
	\emptyvarerr{\equationresize}{#2}{Estilo no definido}
	\checkvalidsourcecodestyle{#2}
	\ifthenelse{\equal{\showlinenumbers}{true}}{
		\rightlinenumbers}{
	}
	\lstset{
		backgroundcolor=\color{\sourcecodebgcolor}
	}
	\ifthenelse{\equal{\codecaptiontop}{true}}{
		\ifx\hfuzz#3\hfuzz
			\lstset{
				style=#2
			}
		\else
			\lstset{
				style=#2,
				caption={#3 #1},
				captionpos=t
			}
		\fi
	}{
		\ifx\hfuzz#3\hfuzz
			\lstset{
				style=#2
			}
		\else
			\lstset{
				style=#2,
				caption={#3 #1},
				captionpos=b
			}
		\fi
	}
}

% Inserta código fuente sin parámetros
%	#1	Label (opcional)
%	#2	Estilo de código
%	#3	Caption
\lstnewenvironment{sourcecode}[3][]{
	\coreinitsourcecode{#1}{#2}{#3}
}{
	\ifthenelse{\equal{\showlinenumbers}{true}}{
		\leftlinenumbers}{
	}
}

% Importa código fuente desde un archivo con parámetros
%	#1	Label (opcional)
%	#2	Estilo de código
%	#3  Archivo de código fuente
%	#4	Caption
\newcommand{\importsourcecode}[4][]{
	\coreinitsourcecode{#1}{#2}{#4}
	\lstinputlisting{#3}
	\ifthenelse{\equal{\showlinenumbers}{true}}{
		\leftlinenumbers}{
	}
}

% Itemize en negrita
%	#1	Parámetros opcionales
\newenvironment{itemizebf}[1][]{
	\begin{itemize}[font=\bfseries,#1]
	}{
	\end{itemize}
}

% Enumerate en negrita
%	#1	Parámetros opcionales
\newenvironment{enumeratebf}[1][]{
	\begin{enumerate}[font=\bfseries,#1]
	}{
	\end{enumerate}
}

% Crea una sección de resumen
\newenvironment{resumen}{
	% Tipo de título para abstract
	\sectionfont{\color{\titlecolor} \fontsizetitle \styletitle \selectfont}
	
	% Inserta un título sin número, sin cabecera y sin aparecer en el índice, para que aparezca en el índice utilizar la función \sectionanumheadless
	\sectionanumnoiheadless{\nameabstract}}{
	
	% Salta de página si está imprimiendo por ambas caras
	\ifthenelse{\equal{\addemptypagetwosides}{true}}{
		\checkoddpage
		\ifoddpage
			\newpage
			\null
			\thispagestyle{empty}
			\newpage
			\addtocounter{page}{-1}
		\else
		\fi}{
	}
}

% Crea una sección de imágenes múltiples
%	#1	Label (opcional)
%	#2	Caption
\newenvironment{images}[2][]{
	% Modifica globales
	\def\envimageslabelvar {#1}
	\def\envimagescaptionvar {#2}
	\def\GLOBALenvimageintialized {true}
	
	% Configura caption y márgenes
	\vspace{\margintopimages cm}
	\captionsetup{margin=\captionmarginmultimg cm}
	
	% Inicia la figura
	\begin{figure}[H] \centering
		\vspace{-\marginrightmultimage cm}
		\vspace{-\marginrightmultimage cm}
		\vspace{-\marginrightmultimage cm}
		}{
		\setcaptionmargincm{\captionlrmargin}
		\ifx\hfuzz\envimagescaptionvar\hfuzz
			\vspace{\captionlessmarginimage cm}
		\else
			\caption{\envimagescaptionvar\envimageslabelvar}
		\fi
	\end{figure}

	% Reestablece caption y márgenes
	\setcaptionmargincm{\captionlrmargin}
	\vspace{\marginbottomimages cm}
	
	% Reestablece globales
	\def\GLOBALenvimageintialized {false}
}

% END