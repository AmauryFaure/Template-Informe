% Template:     Informe/Reporte LaTeX
% Documento:    Importación de librerías
% Versión:      3.0.0-beta-7 (01/04/2017)
% Codificación: UTF-8
%
% Autor: Pablo Pizarro R.
%        Facultad de Ciencias Físicas y Matemáticas.
%        Universidad de Chile.
%        pablo.pizarro@ing.uchile.cl, ppizarror.com
%
% Sitio web del proyecto: [http://ppizarror.com/Template-Informe/]
% Licencia: MIT           [https://opensource.org/licenses/MIT]

\usepackage{amsmath}                  % Fórmulas matemáticas
\usepackage{amssymb}                  % Símbolos matemáticos
\usepackage{amsthm}                   % Teoremas matemáticos
\usepackage{array}                    % Añade nuevas características a las tablas
\usepackage{bigstrut}                 % Líneas horizontales en tablas
\usepackage{booktabs}                 % Permite manejar elem. visuales en tablas
\usepackage[makeroom]{cancel}         % Cancelar términos en fórmulas
\usepackage{caption}                  % Leyendas
\usepackage{color}                    % Colores
\usepackage{colortbl}                 % Administración de color en tablas
\usepackage{datetime}                 % Fechas
\usepackage[inline]{enumitem}         % Permite enumerar ítems
\usepackage[bottom, norule]{footmisc} % Estilo pié de página
\usepackage{fancyhdr}                 % Encabezados y pié de páginas
\usepackage{float}                    % Administrador de posiciones de objetos
\usepackage{textcomp, gensymb}        % Simbología común
\usepackage{geometry}                 % Dimensiones y geometría del documento
\usepackage{graphicx}                 % Propiedades extra para los gráficos
\usepackage{ifthen}                   % Permite el manejo de condicionales
\usepackage{mathtools}                % Permite utilizar notaciones matemáticas
\usepackage[version=4]{mhchem}        % Fórmulas químicas
\usepackage{multicol}                 % Múltiples columnas
\usepackage{pdfpages}                 % Permite administrar páginas en pdf
\usepackage{lipsum}                   % Permite crear textos dummy
\usepackage{longtable}                % Permite utilizar tablas en varias hojas
\usepackage{listings}                 % Permite añadir código fuente
\usepackage{rotating}                 % Permite rotación de objetos
\usepackage{sectsty}                  % Cambia el estilo de los títulos
\usepackage{selinput}                 % Compatibilidad con acentos
\usepackage{setspace}                 % Cambia el espacio entre líneas
\usepackage{subfig}                   % Permite agrupar imágenes
\usepackage{tikz}                     % Permite dibujar
\usepackage{ulem}                     % Permite tachar, subrayar, etc
\usepackage{url}                      % Permite añadir enlaces
\usepackage{wasysym}                  % Contiene caracteres misceláneos
\usepackage{wrapfig}                  % Permite comprimir imágenes
\usepackage{xcolor}                   % Paquete de colores avanzado

% LIBRERÍAS DEPENDIENTES
\usetikzlibrary{babel}                % Asociado a tikz
\usepackage{chngcntr}                 % Agrega números de secciones a las leyendas
\usepackage{epstopdf}                 % Convierte archivos .eps a pdf
\usepackage{multirow}                 % Agrega nuevas opciones a las tablas
\ifthenelse{                          % Permite añadir enlaces y referencias
\equal{\showborderonlinks}{false}}{   % Links sin recuadro rojo
\usepackage[hidelinks]{hyperref}}{
\usepackage{hyperref}}