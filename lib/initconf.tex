% Template:     Informe/Reporte LaTeX
% Documento:    Configuración inicial del template
% Versión:      4.2.7 (09/07/2017)
% Codificación: UTF-8
%
% Autor: Pablo Pizarro R.
%        Facultad de Ciencias Físicas y Matemáticas
%        Universidad de Chile
%        pablo.pizarro@ing.uchile.cl, ppizarror.com
%
% Manual template: [http://ppizarror.com/Template-Informe/]
% Licencia MIT:    [https://opensource.org/licenses/MIT/]

% Se revisa si las variables no han sido borradas
\checkvardefined{\nombredelinforme}
\checkvardefined{\temaatratar}
\checkvardefined{\autordeldocumento}
\checkvardefined{\nombredelcurso}
\checkvardefined{\codigodelcurso}
\checkvardefined{\nombreuniversidad}
\checkvardefined{\nombrefacultad}
\checkvardefined{\departamentouniversidad}
\checkvardefined{\localizacionuniversidad}

% Se añade \xspace a las variables
\makeatletter
	\g@addto@macro\nombredelinforme\xspace
	\g@addto@macro\temaatratar\xspace
	\g@addto@macro\autordeldocumento\xspace
	\g@addto@macro\nombredelcurso\xspace
	\g@addto@macro\codigodelcurso\xspace
	\g@addto@macro\nombreuniversidad\xspace
	\g@addto@macro\nombrefacultad\xspace
	\g@addto@macro\departamentouniversidad\xspace
	\g@addto@macro\localizacionuniversidad\xspace
\makeatother

% Se crean variables si se borraron
\ifthenelse{\isundefined{\tablaintegrantes}}{
	\errmessage{LaTeX Warning: Se borro la variable \noexpand\tablaintegrantes, creando una vacia}
	\def\tablaintegrantes {}
}{}

% Se define metadata del pdf
\hypersetup{
	bookmarksopen={\cfgpdfbookmarkopen},
	bookmarksopenlevel={\cfgbookmarksopenlevel},
	bookmarkstype={toc},
	pdfauthor={\autordeldocumento},
	pdfcenterwindow={\cfgpdfcenterwindow},
	pdfcreator={LaTeX, pdfLaTeX},
	pdfdisplaydoctitle={\cfgpdfdisplaydoctitle},
	pdffitwindow={\cfgpdffitwindow},
	pdfinfo={
		Curso.Codigo={\codigodelcurso},
		Curso.Nombre={\nombredelcurso},
		Documento.Autor={\autordeldocumento},
		Documento.Tema={\temaatratar},
		Documento.Titulo={\nombredelinforme},
		Template.Autor.Alias={ppizarror},
		Template.Autor.Email={pablo.pizarro@ing.uchile.cl},
		Template.Autor.Nombre={Pablo Pizarro R.},
		Template.Autor.Web={http://ppizarror.com/},
		Template.Codificacion={UTF-8},
		Template.Fecha={09/07/2017},
		Template.Licencia.Tipo={MIT},
		Template.Licencia.Web={https://opensource.org/licenses/MIT/},
		Template.Nombre={Template-Informe},
		Template.Tipo={Normal},
		Template.Version.Dev={4.2.7-1-N},
		Template.Version.Hash={6CC0BE369E23588566C56F922A7FF0C1},
		Template.Version.Release={4.2.7},
		Template.Web.Dev={https://github.com/ppizarror/Template-Informe/},
		Template.Web.Manual={http://ppizarror.com/Template-Informe/},
		Universidad.Departamento={\departamentouniversidad},
		Universidad.Nombre={\nombreuniversidad},
		Universidad.Ubicacion={\localizacionuniversidad},
	},
	pdfkeywords={\nombreuniversidad, \codigodelcurso \nombredelcurso, \localizacionuniversidad},
	pdfpagemode={UseOutlines},
	pdfproducer={Template-Informe v4.2.7 | (Pablo Pizarro R.) ppizarror.com},
	pdfstartpage={1},
	pdfstartview={FitH},
	pdfsubject={\temaatratar},
	pdftitle={\nombredelinforme},
	pdftoolbar={\cfgpdftoolbar},
	pdfview={FitH}
}

% Establece la carpeta de imágenes por defecto
\graphicspath{{./\defaultimagefolder}}

% Definición de valores e dimensiones
\setlength{\headheight}{64pt} % Tamaño de la cabecera sin fancyhdr
\setcounter{MaxMatrixCols}{20} % Número máximo de columnas en matrices
\newcounter{templatepagecounter} % Administra números de páginas
\setlength{\footnotemargin}{3mm} % Margen del footnote
\renewcommand{\baselinestretch}{\defaultinterline} % Ajuste del entrelineado

% Configuración de los colores
\color{\maintextcolor} % Color principal
\arrayrulecolor{\tablelinecolor} % Color de las lineas de las tablas
\sethlcolor{\highlightcolor} % Color del subrayado por defecto
\ifthenelse{\equal{\showborderonlinks}{true}}{
	\hypersetup{
		% Color de links con borde
		citebordercolor=\citecolor,
		linkbordercolor=\linkcolor,
		urlbordercolor=\urlcolor
	}
}{
	\hypersetup{
		% Color de links sin borde
		hidelinks,
		colorlinks=true,
		citecolor=\citecolor,
		linkcolor=\linkcolor,
		urlcolor=\urlcolor
	}
}

% Configuración de las leyendas
\setcaptionmargincm{\captionlrmargin} % Márgenes de las leyendas por defecto
\ifthenelse{\equal{\captiontextbold}{true}}{ % Texto en negrita en etiquetas
	\renewcommand{\captiontextbold}{bf}}{
	\renewcommand{\captiontextbold}{}
}
\captionsetup{ % Se configura el texto de los caption
	labelfont={color=\captioncolor, \captiontextbold},
	textfont={color=\captiontextcolor},
	singlelinecheck=on
}
\floatsetup[figure]{ % Configuración de márgenes en las figuras
	captionskip=\captiontbmarginfigure pt
}
\floatsetup[table]{ % Configuración de márgenes en las tablas
	captionskip=\captiontbmargintable pt
}
\ifthenelse{\equal{\figurecaptiontop}{true}}{ % Caption superior en figuras
	\floatsetup[figure]{position=above}}{
}
\ifthenelse{\equal{\tablecaptiontop}{true}}{ % Caption superior en tablas
	\floatsetup[table]{position=top}
	}{
	\floatsetup[table]{position=bottom}
}
\ifthenelse{\equal{\captionalignment}{justified}}{ % Leyenda justificada
	\captionsetup{
		justification=justified,
		format=plain
	}
}{
	\ifthenelse{\equal{\captionalignment}{centered}}{ % Leyenda centrada
		\captionsetup{
			justification=centering
		}
	}{
		\ifthenelse{\equal{\captionalignment}{left}}{ % Leyenda izq
			\captionsetup{
				justification=raggedright,
				singlelinecheck=false
			}
		}{
			\ifthenelse{\equal{\captionalignment}{right}}{ % Leyenda der
				\captionsetup{
					justification=raggedleft,
					singlelinecheck=false
				}
			}{
				\throwbadconfig{Posicion de leyendas desconocida}{\captionalignment}{justified,centered,left,right}
			}
		}
	}
}

% Configuración de referencias y citas
\ifthenelse{\equal{\stylecitereferences}{natbib}}{
	\bibliographystyle{apa}
	\setlength{\bibsep}{\natbibrefsep pt}
}{
	\ifthenelse{\equal{\stylecitereferences}{apacite}}{
		\bibliographystyle{apacite}
		\setlength{\bibitemsep}{\apaciterefsep pt}
	}{
		\ifthenelse{\equal{\stylecitereferences}{bibtex}}{
			\bibliographystyle{apa}
			\newlength{\bibitemsep}\setlength{\bibitemsep}{.2\baselineskip plus .05\baselineskip minus .05\baselineskip}
			\newlength{\bibparskip}\setlength{\bibparskip}{0pt}
			\let\oldthebibliography\thebibliography
			\renewcommand\thebibliography[1]{
				\oldthebibliography{#1}
				\setlength{\parskip}{\bibitemsep}
				\setlength{\itemsep}{\bibparskip}
			}
			\setlength{\bibitemsep}{\bibtexrefsep pt}
		}{
			\throwbadconfig{Estilo citas desconocido}{\stylecitereferences}{bibtex,apacite,natbib}
		}
	}
}
\makeatletter
\ifthenelse{\equal{\twocolumnreferences}{true}}{
	% Referencias en 2 columnas
	\renewenvironment{thebibliography}[1]
	{\begin{multicols}{2}[\section*{\refname}]
		\@mkboth{\MakeUppercase\refname}{\MakeUppercase\refname}
		\list{\@biblabel{\@arabic\c@enumiv}}
		{\settowidth\labelwidth{\@biblabel{#1}}
			\leftmargin\labelwidth
			\advance\leftmargin\labelsep
			\@openbib@code
			\usecounter{enumiv}
			\let\p@enumiv\@empty
			\renewcommand\theenumiv{\@arabic\c@enumiv}}
		\sloppy
		\clubpenalty 4000
		\@clubpenalty \clubpenalty
		\widowpenalty 4000
		\sfcode`\.\@m}
		{\def\@noitemerr
		{\@latex@warning{Ambiente `thebibliography' no definido}}
		\endlist\end{multicols}}}{}
\makeatother

% Configuración anexo
\patchcmd{\appendices}{\quad}{\sectionappendixlastchar\quad}{}{}

% Configuración de códigos fuente
\lstset{
	extendedchars=true,
	keepspaces=true,
	columns=flexible,
	literate=
	{á}{{\'a}}1 {é}{{\'e}}1 {í}{{\'i}}1 {ó}{{\'o}}1 {ú}{{\'u}}1
	{Á}{{\'A}}1 {É}{{\'E}}1 {Í}{{\'I}}1 {Ó}{{\'O}}1 {Ú}{{\'U}}1
	{à}{{\`a}}1 {è}{{\`e}}1 {ì}{{\`i}}1 {ò}{{\`o}}1 {ù}{{\`u}}1
	{À}{{\`A}}1 {È}{{\'E}}1 {Ì}{{\`I}}1 {Ò}{{\`O}}1 {Ù}{{\`U}}1
	{ä}{{\"a}}1 {ë}{{\"e}}1 {ï}{{\"i}}1 {ö}{{\"o}}1 {ü}{{\"u}}1
	{Ä}{{\"A}}1 {Ë}{{\"E}}1 {Ï}{{\"I}}1 {Ö}{{\"O}}1 {Ü}{{\"U}}1
	{â}{{\^a}}1 {ê}{{\^e}}1 {î}{{\^i}}1 {ô}{{\^o}}1 {û}{{\^u}}1
	{Â}{{\^A}}1 {Ê}{{\^E}}1 {Î}{{\^I}}1 {Ô}{{\^O}}1 {Û}{{\^U}}1
	{œ}{{\oe}}1 {Œ}{{\OE}}1 {æ}{{\ae}}1 {Æ}{{\AE}}1 {ß}{{\ss}}1
	{ű}{{\H{u}}}1 {Ű}{{\H{U}}}1 {ő}{{\H{o}}}1 {Ő}{{\H{O}}}1
	{ç}{{\c c}}1 {Ç}{{\c C}}1 {ø}{{\o}}1 {å}{{\r a}}1 {Å}{{\r A}}1
	{€}{{\EUR}}1 {£}{{\pounds}}1
}

% Configuración de hbox y vbox
\hfuzz=200pt \vfuzz=200pt
\hbadness=\maxdimen \vbadness=\maxdimen

% Se activa el word-wrap para textos con \texttt{}
\ttfamily \hyphenchar\the\font=`\-
	
% Se define el tipo de texto de los url
\urlstyle{tt}

% Se activa el modo estricto de revisión de números de página
\strictpagecheck
