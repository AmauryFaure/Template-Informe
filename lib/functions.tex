% Template:     Informe/Reporte LaTeX
% Documento:    Definición de funciones
% Versión:      3.3.6 (06/05/2017)
% Codificación: UTF-8
%
% Autor: Pablo Pizarro R.
%        Facultad de Ciencias Físicas y Matemáticas.
%        Universidad de Chile.
%        pablo.pizarro@ing.uchile.cl, ppizarror.com
%
% Sitio web del proyecto: [http://ppizarror.com/Template-Informe/]
% Licencia: MIT           [https://opensource.org/licenses/MIT]

\newcommand{\throwerror}[2]{
	% Lanza un mensaje de error
	% 	#1	Función del error
	%	#2	Mensaje
	\errmessage{Error: \noexpand#1 #2 (linea \the\inputlineno)}
}

\newcommand{\emptyvarerr}[3]{
	% Lanza un mensaje de error si una variable no ha sido definida
	% 	#1	Función del error
	%	#2	Variable
	%	#3	Mensaje
	\ifx\hfuzz#2\hfuzz
		\throwerror{#1}{#3}
	\fi
}

\newcommand{\referenceindexentry}{
	% Inserta las referencias al índice
	\addcontentsline{toc}{section}{\namereferences}
	\ifthenelse{\equal{\showheadertitle}{true}}{
	\fancyhead[L]{\nouppercase{\namereferences}}}{}
}

\newcommand{\newemptypage}{
	% Crea una página vacía
	\newpage\null\thispagestyle{empty}\newpage
	\addtocounter{page}{-1}
}

\newcommand{\setcaptionmargincm}[1]{
	% Cambiar el margen
	% 	#1	Margen en centímetros
	\captionsetup{margin=#1cm}
}

\newcommand{\setpagemargincm}[4]{
	% Cambia márgenes de las páginas [cm]
	% 	#1	Margen izquierdo
	%	#2	Margen superior
	%	#3	Margen derecho
	%	#4	Margen inferior
	\newgeometry{left=#1cm, top=#2cm, right=#3cm, bottom=#4cm}
}

\newcommand{\newp}{
	% Inserta nueva línea	
	\hbadness=10000 \vspace{\defaultnewlinesize} \par
}

\newcommand{\newpar}[1]{
	% Insertar párrafo
	% 	#1	Párrafo
	\hbadness=10000 #1 \newp
}

\newcommand{\newparnl}[1]{
	% Insertar párrafo sin nueva línea al final
	% 	#1	Párrafo
	#1 \par
}

\newcommand{\lpow}[2]{
	% Insertar sub-índice, a_b
	% 	#1	Elemento inferior (a)
	%	#2	Elemento superior (b)
	{#1}_{#2}
}

\newcommand{\pow}[2]{
	% Insertar elevado, a^b
	% 	#1	Elemento inferior (a)
	%	#2	Elemento superior (b)
	{#1}^{#2}
}

\newcommand{\fracpartial}[2]{
	% Fracción de derivadas parciales af/ax
	% 	#1	Función a derivar (f)
	%	#2	Variable a derivar (x)
	\frac{\partial #1}{\partial #2}
}

\newcommand{\fracdpartial}[2]{
	% Fracción de derivadas parciales dobles a^2f/ax^2
	% 	#1	Función a derivar (f)
	%	#2	Variable a derivar (x)
	\frac{{\partial}^{2} #1}{\partial {#2}^{2}}
}

\newcommand{\fracnpartial}[3]{
	% Fracción de derivadas parciales en n, a^nf/ax^n
	% 	#1	Función a derivar (f)
	%	#2	Variable a derivar (x)
	%	#3	Orden (n)
	\frac{{\partial}^{#3} #1}{\partial {#2}^{#3}}
}

\newcommand{\fracderivat}[2]{
	% Fracción de derivadas df/dx
	% 	#1	Función a derivar (f)
	%	#2	Variable a derivar (x)
	\frac{\text{d} #1}{\text{d} #2}
}

\newcommand{\fracdderivat}[2]{
	% Fracción de derivadas dobles d^2/dx^2
	% 	#1	Función a derivar (f)
	%	#2	Variable a derivar (x)
	\frac{{\text{d}}^{2} #1}{\text{d} {#2}^{2}}
}

\newcommand{\fracnderivat}[3]{
	% Fracción de derivadas en n d^nf/dx^n
	% 	#1	Función a derivar (f)
	%	#2	Variable a derivar (x)
	%	#3	Orden de la derivada (n)	
	\frac{{\text{d}}^{#3} #1}{\text{d} {#2}^{#3}}
}

\newcommand{\topequal}[2]{
	% Llave superior de equivalencia
	% 	#1	Elemento a igualar
	%	#2	Igualdad
	\overbrace{#1}^{\mathclap{#2}}
}

\newcommand{\underequal}[2]{
	% Llave inferior de equivalencia
	% 	#1	Elemento a igualar
	%	#2	Igualdad
	\underbrace{#1}_{\mathclap{#2}}
}

\newcommand{\topsequal}[2]{
	% Rectángulo superior de equivalencia
	% 	#1	Elemento a igualar
	%	#2	Igualdad
	\overbracket{#1}^{\mathclap{#2}}
}

\newcommand{\undersequal}[2]{
	% Rectángulo inferior de equivalencia
	% 	#1	Elemento a igualar
	%	#2	Igualdad
	\underbracket{#1}_{\mathclap{#2}}
}

\newcommand{\resizeitem}[2]{
	% Crea un resizebox de tamaño textwidth
	% 	#1	Tamaño del nuevo objeto (En textwidth)
	%	#2	Objeto a redimensionar
	\emptyvarerr{\resizeitem}{#1}{Tamano del nuevo objeto no definido}
	\emptyvarerr{\resizeitem}{#2}{Objeto a redimensionar no definido}
	\resizebox{#1\textwidth}{!}{#2}
}

\newcommand{\newtitleanum}[1]{
	% Insertar un título sin número
	% 	#1	Título
	\emptyvarerr{\newtitleanum}{#1}{Titulo no definido}
	\addcontentsline{toc}{section}{#1}
	\section*{#1}
	\ifthenelse{\equal{\showheadertitle}{true}}{
		\fancyhead[L]{\nouppercase{#1}}}{}
	\stepcounter{section}
}

\newcommand{\newtitleanumheadless}[1]{
	% Insertar un título sin número sin alterar el header
	% 	#1	Título
	\emptyvarerr{\newtitleanumheadless}{#1}{Titulo no definido}
	\addcontentsline{toc}{section}{#1}
	\section*{#1}
	\stepcounter{section}
}

\newcommand{\newsubtitleanum}[1]{
	% Insertar un subtítulo sin número
	% 	#1	Subtítulo
	\emptyvarerr{\newsubtitleanum}{#1}{Subtitulo no definido}
	\addcontentsline{toc}{subsection}{#1}
	\subsection*{#1}
	\stepcounter{subsection}
}

\newcommand{\newsubsubtitleanum}[1]{
	% Insertar un sub-subtítulo sin número
	% 	#1	Sub-subtítulo
	\emptyvarerr{\newsubsubtitleanum}{#1}{Sub-subtitulo no definido}
	\addcontentsline{toc}{subsubsection}{#1}
	\subsubsection*{#1}
	\stepcounter{subsubsection}
}

\newcommand{\newtitleanumnoi}[1]{
	% Insertar un título sin número sin indexar
	% 	#1	Título
	\emptyvarerr{\newtitleanumnoi}{#1}{Titulo no definido}
	\section*{#1}
	\ifthenelse{\equal{\showheadertitle}{true}}{
		\fancyhead[L]{\nouppercase{#1}}}{}
	\stepcounter{section}
}

\newcommand{\newtitleanumnoiheadless}[1]{
	% Insertar un título sin número sin indexar sin cambiar el header
	% 	#1	Título
	\emptyvarerr{\newtitleanumnoiheadless}{#1}{Titulo no definido}
	\section*{#1}
	\ifthenelse{\equal{\showheadertitle}{true}}{
		\fancyhead[L]{\nouppercase{#1}}}{}
	\stepcounter{section}
}

\newcommand{\newsubtitleanumnoi}[1]{
	% Insertar un subtítulo sin número sin indexar
	% 	#1	Subtítulo
	\emptyvarerr{\newsubtitleanumnoi}{#1}{Subtitulo no definido}
	\subsection*{#1}
	\stepcounter{subsection}
}

\newcommand{\newsubsubtitleanumnoi}[1]{
	% Insertar un sub-subtítulo sin número sin indexar
	% 	#1	Sub-subtítulo
	\emptyvarerr{\newsubsubtitleanumnoi}{#1}{Sub-subtitulo no definido}
	\addcontentsline{toc}{subsubsection}{#1}
	\subsubsection*{#1}
	\stepcounter{subsubsection}
}

\newcommand{\insertindextitle}[2]{
	% Insertar un título en un índice
	%	#1	Título
	%	#2	Margen superior en pt. (opcional), 10pt por defecto
	\emptyvarerr{\insertindextitle}{#1}{Titulo no definido}
	\ifx\hfuzz#2\hfuzz
		\addtocontents{toc}{\protect\addvspace{\defaultindextitlemargin}}
	\else
		\addtocontents{toc}{\protect\addvspace{#2 pt}}
	\fi
	\addtocontents{toc}{\noindent\hyperref[swpn]{\textbf{#1}}}
}

\newcommand{\insertequation}[2][]{
	% Insertar una ecuación
	% 	#1	Label (opcional)
	%	#2	Ecuación
	\emptyvarerr{\insertequation}{#2}{Ecuacion no definida}
	\vspace{-0.1cm}
	\begin{equation}
		\text{#1} #2
	\end{equation}
	\vspace{-0.26cm}
	\par
}

\newcommand{\insertequationcaptioned}[3][]{
	% Insertar una ecuación con leyenda
	% 	#1	Label (opcional)
	%	#2	Ecuación
	%	#3	Caption
	\emptyvarerr{\insertequationcaptioned}{#2}{Ecuacion no definida}
	\ifx\hfuzz#3\hfuzz
		\insertequation[#1]{#2}
	\else
		\vspace{0cm}
		\begin{equation}
		\text{#1} #2
		\end{equation}
		\begin{center}
			\vspace{-0.15cm}
			\textit{#3} \par
			\vspace{0.05cm}
		\end{center}	
	\fi
}

\newcommand{\insertequationgathered}[2][]{
	% Insertar una ecuación con el ambiente gather
	% 	#1	Label (opcional)
	%	#2	Ecuación
	\emptyvarerr{\insertequationgathered}{#2}{Ecuacion no definida}
	\vspace{-0.4cm}
	\begin{gather}
		\text{#1} #2
	\end{gather}
	\par
	\vspace{-0.10cm}
}

\newcommand{\insertequationgatheredcaptioned}[3][]{
	% Insertar una ecuación (gather) con leyenda
	% 	#1	Label (opcional)
	%	#2	Ecuación
	%	#3	Caption
	\emptyvarerr{\insertequationgatheredcaptioned}{#2}{Ecuacion no definida}
	\ifx\hfuzz#3\hfuzz
		\insertequationgathered[#1]{#2}
	\else
		\vspace{0cm}
		\begin{gather}
			\text{#1} #2
		\end{gather}
		\begin{center}
			\vspace{-0.15cm}
			\textit{#3} \par
		\end{center}
	\fi
}

\newcommand{\insertequationalign}[2][]{
	% Insertar una ecuación con el ambiente align
	% 	#1	Label (opcional)
	%	#2	Ecuación
	\emptyvarerr{\insertequationalign}{#2}{Ecuacion no definida}
	\vspace{-0.4cm}
	\begin{align}
		\text{#1} #2
	\end{align}
	\par
	\vspace{-0.10cm}
}

\newcommand{\insertequationaligncaptioned}[3][]{
	% Insertar una ecuación (align) con leyenda
	% 	#1	Label (opcional)
	%	#2	Ecuación
	%	#3	Caption
	\emptyvarerr{\insertequationaligncaptioned}{#2}{Ecuacion no definida}
	\ifx\hfuzz#3\hfuzz
		\insertequationalign[#1]{#2}
	\else
		\vspace{0cm}
		\begin{align}
			\text{#1} #2
		\end{align}
		\begin{center}
			\vspace{-0.15cm}
			\textit{#3} \par
		\end{center}
	\fi
}

\newcommand{\insertimage}[4][]{
	% Insertar una imagen
	% 	#1	Label (opcional)
	%	#2	Dirección de la imagen
	%	#3	Parámetros de la imagen
	%	#4	Caption de la imagen
	\emptyvarerr{\insertimage}{#2}{Direccion de la imagen no definida}
	\emptyvarerr{\insertimage}{#3}{Parametros de la imagen no definidos}
	\vspace{\defaultmargintopimages}
	\begin{figure}[H]
		\centering
		\includegraphics[#3]{\defaultimagefolder#2}
		\ifx\hfuzz#4\hfuzz
			\vspace{\defaultcaptionlessmargin}
		\else
			\caption{#4 #1}
		\fi
	\end{figure}
	\vspace{\defaultmarginbottomimages}
}

\newcommand{\insertimageboxed}[4][]{
	% Insertar una imagen con recuadro
	% 	#1	Label (opcional)
	%	#2	Dirección de la imagen
	%	#3	Parámetros de la imagen
	%	#4	Caption de la imagen
	\emptyvarerr{\insertimageboxed}{#2}{Direccion de la imagen no definida}
	\emptyvarerr{\insertimageboxed}{#3}{Parametros de la imagen no definidos}
	\vspace{\defaultmargintopimages}
	\begin{figure}[H]
		\centering
		\fbox{\includegraphics[#3]{\defaultimagefolder#2}}
		\ifx\hfuzz#4\hfuzz
			\vspace{\defaultcaptionlessmargin}
		\else
			\caption{#4 #1}
		\fi
	\end{figure}
	\vspace{\defaultmarginbottomimages}
}

\newcommand{\insertimagefixed}[5][]{
	% Insertar una imagen de ancho fijo a la página
	% 	#1	Label (opcional)
	%	#2	Dirección de la imagen
	%	#3	Parámetros de la imagen
	%	#4	Tamaño de la imagen en textwidth
	%	#5	Caption de la imagen
	\emptyvarerr{\insertimagefixed}{#2}{Direccion de la imagen no definida}
	\emptyvarerr{\insertimagefixed}{#3}{Parametros de la imagen no definidos}
	\emptyvarerr{\insertimagefixed}{#4}{Tamano de la imagen (textwidth) no definida}
	\vspace{\defaultmargintopimages}
	\begin{figure}[H]
		\centering
		\resizebox{#3\textwidth}{!}{
			\includegraphics[#4]{\defaultimagefolder#2}
		}
		\ifx\hfuzz#5\hfuzz
			\vspace{\defaultcaptionlessmargin}
		\else
			\caption{#5 #1}
		\fi
	\end{figure}
	\vspace{\defaultmarginbottomimages}
}

\newcommand{\insertimageboxedfixed}[5][]{
	% Insertar una imagen recuadrada de ancho fijo
	% 	#1	Label (opcional)
	%	#2	Dirección de la imagen
	%	#3	Parámetros de la imagen
	%	#4	Tamaño de la imagen en textwidth
	%	#5	Caption de la imagen
	\emptyvarerr{\insertimageboxedfixed}{#2}{Direccion de la imagen no definida}
	\emptyvarerr{\insertimageboxedfixed}{#3}{Parametros de la imagen no definidos}
	\emptyvarerr{\insertimageboxedfixed}{#4}{Tamano de la imagen no definida}
	\vspace{\defaultmargintopimages}
	\begin{figure}[H]
		\centering
		\resizebox{#3\textwidth}{!}{
			\fbox{\includegraphics[#4]{\defaultimagefolder#2}}
		}
		\ifx\hfuzz#5\hfuzz
			\vspace{\defaultcaptionlessmargin}
		\else
			\caption{#5 #1}
		\fi
	\end{figure}
	\vspace{\defaultmarginbottomimages}
}

\newcommand{\insertdoubleimage}[8][]{
	% Insertar una imagen doble
	% 	#1	Label (opcional)
	%	#2	Dirección de la imagen 1
	%	#3	Parámetros de la imagen 1
	%	#4	Caption de la imagen 1
	%	#5	Dirección de la imagen 2
	%	#6	Parámetros de la imagen 2
	%	#7	Caption de la imagen 2
	%	#8	Caption de la imagen doble
	\emptyvarerr{\insertdoubleimage}{#2}{Direccion de la imagen 1 no definida}
	\emptyvarerr{\insertdoubleimage}{#3}{Parametros de la imagen 1 no definidos}
	\emptyvarerr{\insertdoubleimage}{#5}{Direccion de la imagen 2 no definida}
	\emptyvarerr{\insertdoubleimage}{#6}{Parametros de la imagen 2 no definidos}
	\vspace{\defaultmargintopimages}
	\captionsetup{margin=0.45cm}
	\begin{figure}[H] \centering
		\subfloat[#4]{
			\includegraphics[#3]{\defaultimagefolder#2}}
		\hspace{0.2cm}
		\subfloat[#7]{
			\includegraphics[#6]{\defaultimagefolder#5}}
		\setcaptionmargincm{\defaultcaptionmargin}
		\ifx\hfuzz#8\hfuzz
			\vspace{\defaultcaptionlessmargin}
		\else
			\caption{#8 #1}
		\fi
	\end{figure}
	\setcaptionmargincm{\defaultcaptionmargin}
	\vspace{\defaultmarginbottomimages}
}

\newcommand{\insertdoubleeqimage}[7][]{
	% Insertar una imagen doble, igual propiedades
	% 	#1	Label (opcional)
	%	#2	Dirección de la imagen 1
	%	#3	Caption de la imagen 1
	%	#4	Dirección de la imagen 2
	%	#5	Caption de la imagen 2
	%	#6	Propiedades de las imágenes
	%	#7 	Caption de la imagen doble
	\insertdoubleimage[#1]{#2}{#6}{#3}{#4}{#6}{#5}{#7}
}

\newcommand{\inserttripleimage}[8][]{
	% Insertar una imagen triple
	% 	#1	Label (opcional)
	%	#2	Dirección de la imagen 1
	%	#3	Parámetros de la imagen 1
	%	#4	Dirección de la imagen 2
	%	#5	Parámetros de la imagen 2
	%	#6	Dirección de la imagen 3
	%	#7	Parámetros de la imagen 3
	%	#8	Caption de la imagen triple
	\emptyvarerr{\inserttripleimage}{#2}{Direccion de la imagen 1 no definida}
	\emptyvarerr{\inserttripleimage}{#3}{Parametros de la imagen 1 no definidos}
	\emptyvarerr{\inserttripleimage}{#4}{Direccion de la imagen 2 no definida}
	\emptyvarerr{\inserttripleimage}{#5}{Parametros de la imagen 2 no definidos}
	\emptyvarerr{\inserttripleimage}{#6}{Direccion de la imagen 3 no definida}
	\emptyvarerr{\inserttripleimage}{#7}{Parametros de la imagen 3 no definidos}
	\vspace{\defaultmargintopimages}
	\captionsetup{margin=0.45cm}
	\begin{figure}[H] \centering
		\subfloat[]{
			\includegraphics[#3]{\defaultimagefolder#2}}
		\hspace{0.1cm}
		\subfloat[]{
			\includegraphics[#5]{\defaultimagefolder#4}}
		\hspace{0.1cm}
		\subfloat[]{
			\includegraphics[#7]{\defaultimagefolder#6}}
		\setcaptionmargincm{\defaultcaptionmargin}
		\ifx\hfuzz#8\hfuzz
			\vspace{\defaultcaptionlessmargin}
		\else
			\caption{#8 #1}
		\fi
	\end{figure}
	\setcaptionmargincm{\defaultcaptionmargin}
	\vspace{\defaultmarginbottomimages}
}

\newcommand{\inserttripleeqimage}[6][]{
	% Insertar una imagen triple, igual propiedades
	% 	#1	Label (opcional)
	%	#2	Dirección de la imagen 1
	%	#3	Dirección de la imagen 2
	%	#4	Dirección de la imagen 3
	%	#5	Propiedades de las imágenes
	%	#6	Caption de la imagen triple
	\inserttripleimage[#1]{#2}{#5}{#3}{#5}{#4}{#5}{#6}
}

\newcommand{\insertquadimage}[7][]{
	% Insertar una imagen cuádruple, igual propiedades
	% 	#1	Label (opcional)
	%	#2	Dirección de la imagen 1
	%	#3	Dirección de la imagen 2
	%	#4	Dirección de la imagen 3
	%	#5	Dirección de la imagen 4
	%	#6	Propiedades de las imágenes
	%	#7	Caption de la imagen cuádruple
	\emptyvarerr{\insertquadimage}{#2}{Direccion de la imagen 1 no definida}
	\emptyvarerr{\insertquadimage}{#3}{Direccion de la imagen 2 no definida}
	\emptyvarerr{\insertquadimage}{#4}{Direccion de la imagen 3 no definida}
	\emptyvarerr{\insertquadimage}{#5}{Direccion de la imagen 4 no definida}
	\emptyvarerr{\insertquadimage}{#6}{Propiedades de las imagenes no definidos}
	\vspace{\defaultmargintopimages}
	\captionsetup{margin=0.45cm}
	\begin{figure}[H] \centering
		\subfloat[]{
			\includegraphics[#6]{\defaultimagefolder#2}}
		\hspace{0.1cm}
		\subfloat[]{
			\includegraphics[#6]{\defaultimagefolder#3}}
		\hspace{0.1cm}
		\subfloat[]{
			\includegraphics[#6]{\defaultimagefolder#4}}
		\hspace{0.1cm}
		\subfloat[]{
			\includegraphics[#6]{\defaultimagefolder#5}}
		\setcaptionmargincm{\defaultcaptionmargin}
		\ifx\hfuzz#7\hfuzz
			\vspace{\defaultcaptionlessmargin}
		\else
			\caption{#7 #1}
		\fi
	\end{figure}
	\setcaptionmargincm{\defaultcaptionmargin}
	\vspace{\defaultmarginbottomimages}
}

\newcommand{\insertpentaimage}[8][]{
	% Insertar una imagen quíntuple, igual propiedades
	% 	#1	Label (opcional)
	%	#2	Dirección de la imagen 1
	%	#3	Dirección de la imagen 2
	%	#4	Dirección de la imagen 3
	%	#5	Dirección de la imagen 4
	%	#6	Dirección de la imagen 5
	%	#7	Propiedades de las imágenes
	%	#8	Caption de la imagen quíntuple
	\emptyvarerr{\insertpentaimage}{#2}{Direccion de la imagen 1 no definida}
	\emptyvarerr{\insertpentaimage}{#3}{Direccion de la imagen 2 no definida}
	\emptyvarerr{\insertpentaimage}{#4}{Direccion de la imagen 3 no definida}
	\emptyvarerr{\insertpentaimage}{#5}{Direccion de la imagen 4 no definida}
	\emptyvarerr{\insertpentaimage}{#6}{Direccion de la imagen 5 no definida}
	\emptyvarerr{\insertpentaimage}{#7}{Propiedades de las imagenes no definidas}
	\vspace{\defaultmargintopimages}
	\captionsetup{margin=0.45cm}
	\begin{figure}[H] \centering
		\subfloat[]{
			\includegraphics[#7]{\defaultimagefolder#2}}
		\hspace{0.1cm}
		\subfloat[]{
			\includegraphics[#7]{\defaultimagefolder#3}}
		\hspace{0.1cm}
		\subfloat[]{
			\includegraphics[#7]{\defaultimagefolder#4}}
		\hspace{0.1cm}
		\subfloat[]{
			\includegraphics[#7]{\defaultimagefolder#5}}
		\hspace{0.1cm}
		\subfloat[]{
			\includegraphics[#7]{\defaultimagefolder#6}}
		\setcaptionmargincm{\defaultcaptionmargin}
		\ifx\hfuzz#8\hfuzz
			\vspace{\defaultcaptionlessmargin}
		\else
			\caption{#8 #1}
		\fi
	\end{figure}
	\setcaptionmargincm{\defaultcaptionmargin}
	\vspace{\defaultmarginbottomimages}
}

\newcommand{\inserthexaimage}[9][]{
	% Insertar una imagen con 6 imágenes, igual propiedades
	% 	#1	Label (opcional)
	%	#2	Dirección de la imagen 1
	%	#3	Dirección de la imagen 2
	%	#4	Dirección de la imagen 3
	%	#5	Dirección de la imagen 4
	%	#6	Dirección de la imagen 5
	%	#7	Dirección de la imagen 6
	%	#8	Propiedades de las imágenes
	%	#9	Caption de la imagen global
	\emptyvarerr{\inserthexaimage}{#2}{Direccion de la imagen 1 no definida}
	\emptyvarerr{\inserthexaimage}{#3}{Direccion de la imagen 2 no definida}
	\emptyvarerr{\inserthexaimage}{#4}{Direccion de la imagen 3 no definida}
	\emptyvarerr{\inserthexaimage}{#5}{Direccion de la imagen 4 no definida}
	\emptyvarerr{\inserthexaimage}{#6}{Direccion de la imagen 5 no definida}
	\emptyvarerr{\inserthexaimage}{#7}{Direccion de la imagen 6 no definida}
	\emptyvarerr{\inserthexaimage}{#8}{Propiedades de las imagenes no definidas}
	\vspace{\defaultmargintopimages}
	\captionsetup{margin=0.45cm}
	\begin{figure}[H] \centering
		\subfloat[]{
			\includegraphics[#8]{\defaultimagefolder#2}}
		\hspace{0.1cm}
		\subfloat[]{
			\includegraphics[#8]{\defaultimagefolder#3}}
		\hspace{0.1cm}
		\subfloat[]{
			\includegraphics[#8]{\defaultimagefolder#4}}
		\hspace{0.1cm}
		\subfloat[]{
			\includegraphics[#8]{\defaultimagefolder#5}}
		\hspace{0.1cm}
		\subfloat[]{
			\includegraphics[#8]{\defaultimagefolder#6}}
		\hspace{0.1cm}
		\subfloat[]{
			\includegraphics[#8]{\defaultimagefolder#7}}
		\setcaptionmargincm{\defaultcaptionmargin}
		\ifx\hfuzz#9\hfuzz
			\vspace{\defaultcaptionlessmargin}
		\else
			\caption{#9 #1}
		\fi
	\end{figure}
	\setcaptionmargincm{\defaultcaptionmargin}
	\vspace{\defaultmarginbottomimages}
}

\newcommand{\insertimageleft}[5][]{
	% Insertar una imagen a la izquierda
	% 	#1	Label (opcional)
	%	#2	Dirección de la imagen
	%	#3	Ancho de la imagen (en textwidth)
	%	#4	Altura en líneas de la imagen
	%	#5	Caption de la imagen
	\emptyvarerr{\insertimageleft}{#2}{Direccion de la imagen no definida}
	\emptyvarerr{\insertimageleft}{#3}{Ancho de la imagen no defindo}
	\emptyvarerr{\insertimageleft}{#4}{Altura en lineas de la imagen no definida}
	\begin{wrapfigure}[#4]{l}{#3\textwidth}
		\setcaptionmargincm{0}
		\ifthenelse{\equal{\figurecaptiontop}{true}}{}{
		\vspace{\defaultmarginfloatimages}}
		\centering
		\includegraphics[width=\linewidth]{\defaultimagefolder#2}
		\ifx\hfuzz#5\hfuzz
			\vspace{\defaultcaptionlessmargin}
		\else
			\caption{#5 #1}
		\fi
		\setcaptionmargincm{\defaultcaptionmargin}
	\end{wrapfigure}
}

\newcommand{\insertimageright}[5][]{
	% Insertar una imagen a la derecha
	% 	#1	Label (opcional)
	%	#2	Dirección de la imagen
	%	#3	Ancho de la imagen (en textwidth)
	%	#4	Altura en líneas de la imagen
	%	#5	Caption de la imagen
	\emptyvarerr{\insertimageright}{#2}{Direccion de la imagen no definida}
	\emptyvarerr{\insertimageright}{#3}{Ancho de la imagen no defindo}
	\emptyvarerr{\insertimageright}{#4}{Altura en lineas de la imagen no definida}
	\begin{wrapfigure}[#4]{r}{#3\textwidth}
		\setcaptionmargincm{0}
		\ifthenelse{\equal{\figurecaptiontop}{true}}{}{}
		\centering
		\includegraphics[width=\linewidth]{\defaultimagefolder#2}
		\ifx\hfuzz#5\hfuzz
			\vspace{\defaultcaptionlessmargin}
		\else
			\caption{#5 #1}
		\fi
		\setcaptionmargincm{\defaultcaptionmargin}
	\end{wrapfigure}
}