% Template:     Informe/Reporte LaTeX
% Documento:    Definición de entornos
% Versión:      4.1.2 (04/07/2017)
% Codificación: UTF-8
%
% Autor: Pablo Pizarro R.
%        Facultad de Ciencias Físicas y Matemáticas
%        Universidad de Chile
%        pablo.pizarro@ing.uchile.cl, ppizarror.com
%
% Manual template: [http://ppizarror.com/Template-Informe/]
% Licencia MIT:    [https://opensource.org/licenses/MIT/]

% Crea una sección de referencias solo para bibtex
\newenvironment{references}{
	\ifthenelse{\equal{\stylecitereferences}{bibtex}}{
	}{
		\throwerror{}{Solo se puede usar entorno references con estilo citas \noexpand\stylecitereferences=bibtex}
	}
	\begingroup
	% Se configura las referencias como una sección
	\ifthenelse{\equal{\referencenumsection}{true}}{
		\section{\namereferences}
	}{
		\sectionanum{\namereferences}
	}
	\renewcommand{\section}[2]{}
	\begin{thebibliography}{99}
	}
	{ 
	\end{thebibliography}
	\endgroup
}

% Crea una sección de resumen
\newenvironment{resumen}{
	% Tipo de título para abstract
	\sectionfont{\color{\titlecolor} \fontsizetitle \styletitle \selectfont}
	
	% Inserta un título sin número, sin cabecera y sin aparecer en el índice, para que aparezca en el índice utilizar la función \sectionanumheadless
	\sectionanumnoiheadless{\nameabstract}}{
	
	% Salta de página si está imprimiendo por ambas caras
	\ifthenelse{\equal{\addemptypagetwosides}{true}}{
		\checkoddpage
		\ifoddpage
			\newpage
			\null
			\thispagestyle{empty}
			\newpage
			\addtocounter{page}{-1}
		\else
		\fi
	}{}
}

% Columna centrada en tablas
\newcolumntype{P}[1]{
	>{\centering\arraybackslash}p{#1}
}
