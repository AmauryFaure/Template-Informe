\begin{sourcecode}{matlab}{Ejemplo en Matlab.}
%% Energía
syms f(o1,o2,P)
syms Df(o1,o2,P) DDf(o1,o2,P)
% Ecuación de energía total
f(o1,o2,P) = 1/2*k*(o1-o0)^2+1/2*k*(o2-o1)^2-...
5
% Primeras derivadas
Df(o1,o2,P) = gradient(f(o1,o2,P), [o1 o2]);
% Segundas derivadas
DDf(o1,o2,P) = hessian(f(o1,o2,P), [o1 o2]);

%% Solucionar
o1 = zeros(size(P_e)); % rad, Adivinanza de o1 inicial
o2 = zeros(size(P_e)); % rad, Adivinanza de o2 inicial
o1(1) = o0;
o2(1) = o0;
P_a = zeros(size(P_e)); % N, P aproximado
n = zeros(size(P_e)); % Para analizar número de iteraciones
for i = 1:length(P_e)
if i >1
o1(i) = o1(i-1);
o2(i) = o2(i-1);
end
while abs(P_a(i)-P_e(i)) > 0.5
delta = double(subs(-inv(DDf(o1(i),o2(i),P_e(i)))*...
Df(o1(i),o2(i),P_e(i))));
o1(i) = o1(i) + delta(1);
o2(i) = o2(i) + delta(2);
eq = Df(o1(i),o2(i),P);
P_a(i) = solve(eq(1),P);
n(i) = n(i)+1;
end
end

function d_fs = d_S(e,ps)
% Rigidez acero en base a deformación
[ey,eu,Es1,Es2] = deal(ps{:});
d_fs = zeros(size(e));
for i = 1:length(e)
es = e(i);
if abs(e(i)) <= ey
d_fs(i) = sign(es)*Es1;
elseif abs(e(i)) <= eu
d_fs(i) = sign(es)*Es2;
else
d_fs(i) = 0;
end
end
end
function fc = C(e,pc)
% Tensiones Hormigón
[er,e0,Ec,fc_] = deal(pc{:});
fc = zeros(size(e));
for i = 1:length(e)
ec = e(i);
if ec < -er
fc(i) = 0;
elseif ec < 0
fc(i) = ec*Ec;
elseif ec < 2*e0
fc(i) = fc_*(2*(ec/e0)-(ec/e0)^2);
else
fc(i) = 0;
end
end
end
function d_fc =

%P2 T2 No Lineal, Diego Abarca Aguilar
%% Inicializar
close all; clear all; clc;
%% Datos
L = 2; % m
k = 10*10^3; % N/m
o0 = 5; %
o0 = o0*pi/180; % rad
%% Datos simbólicos (si se quiere obtener ecuaciones)
% syms L k o0
%% Cargas a evaluar
Pcr = (3-sqrt(5))/2*(k/L); % N, Pcr caso linealizado
P_e = linspace(0,Pcr,100); % N, P exacto
%% Energía
syms f(o1,o2,P)
syms Df(o1,o2,P) DDf(o1,o2,P)
% Ecuación de energía total
f(o1,o2,P) = 1/2*k*(o1-o0)^2+1/2*k*(o2-o1)^2-...
P*(L-L*cos(o2)+L-L*cos(o1));
% Primeras derivadas
Df(o1,o2,P) = gradient(f(o1,o2,P), [o1 o2]);
% Segundas derivadas
DDf(o1,o2,P) = hessian(f(o1,o2,P), [o1 o2]);
%% Solucionar
o1 = zeros(size(P_e)); % rad, Adivinanza de o1 inicial
o2 = zeros(size(P_e)); % rad, Adivinanza de o2 inicial
o1(1) = o0;
o2(1) = o0;
P_a = zeros(size(P_e)); % N, P aproximado
n = zeros(size(P_e)); % Para analizar número de iteraciones
for i = 1:length(P_e)
if i >1
o1(i) = o1(i-1);
o2(i) = o2(i-1);
end
while abs(P_a(i)-P_e(i)) > 0.5
delta = double(subs(-inv(DDf(o1(i),o2(i),P_e(i)))*...
Df(o1(i),o2(i),P_e(i))));
o1(i) = o1(i) + delta(1);
13
o2(i) = o2(i) + delta(2);
eq = Df(o1(i),o2(i),P);
P_a(i) = solve(eq(1),P);
n(i) = n(i)+1;
end
end
figure(1)
plot(o1*180/pi,P_a)
hold on
plot(o2*180/pi,P_a)
ylabel('P [N]')
legend('\theta_1','\theta_2')
figure(2)
plot(n,P_a)
xlabel('Número de iteraciones')
ylabel('P [N]')
xlim([0 2])
\end{sourcecode}