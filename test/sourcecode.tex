\begin{sourcecode}[]{cuda}{Un poco de cuda.}
__global__ void foo(){
}

__global__ void addKernel(int *c, const int *a, const int *b){
	int i = threadIdx.x;
	c[i] = a[i] + b[i];
}

foo<<<n,m>>>();
\end{sourcecode}

\begin{sourcecode}[]{bash}{Un poco de bash.}
# Muestra toda la información de la batería
function battr-info {
	wrks-scripts
	data=$(ioreg -l -w0 |grep Capacity)
	python2.7 battery_info.py $data
	cd - >> config/.empty
}

function qtest-java {
	if [ -z "${1}" ]; then
	echo-err 'Nombre fuente no definido'
	else
	vim $1
	javac -encoding ISO-8859-1 $1 $2 $3
	blankspace=""
	first=$1
	first=${first/.java/$blankspace}
	java $first
	fi
}

PATH=$PATH:"/Library/Frameworks/Python.framework/Versions/2.7/bin:${PATH}"
PATH=$PATH:"/Applications/Utilities/Lynxlet.app/Contents/Resources/lynx/bin"
export PATH

alias ga='git add '
dig +short myip.opendns.com @resolver1.opendns.com
gcc "$@" -o $first

::claramente esto no funcionará::
sudo x, call y
rem esto es un comentario en windows
history -c rm a, ls -d killall e mv | grep | awk python 'hola'
vim uwu printf 'have you seen him' -z
git doge ssh cp cd
\end{sourcecode}

\newpage
\begin{sourcecode}[]{c}{Ejemplo en C.}
#include <stdio.h>
int main(){
	int i, j, rows;
	
	printf("Enter number of rows: ");
	scanf("%d",&rows);
	
	for(i=1; i<=rows; ++i){
		for(j=1; j<=i; ++j){
			printf("* ");
		}
		printf("\n");
	}
	return 0;
}
\end{sourcecode}

\begin{sourcecode}[]{csharp}{Ejemplo en C\#.}
/*
* C# Program to Get a Number and Display the Sum of the Digits 
*/
using System;
using System.Collections.Generic;
using System.Linq;
using System.Text;

namespace Program
{
	class Program
	{
		static void Main(string[] args)
		{
			int num, sum = 0, r;
			Console.WriteLine("Enter a Number : ");
			num = int.Parse(Console.ReadLine());
			while (num != 0)
			{
				r = num % 10;
				num = num / 10;
				sum = sum + r;
			}
			Console.WriteLine("Sum of Digits of the Number : "+sum);
			Console.ReadLine();
			
		}
	}
}
\end{sourcecode}

\newpage
\begin{sourcecode}{cpp}{Suma en C++.}
#include <iostream>
using namespace std;

int main()
{
	int n, sum = 0;
	
	cout << "Enter a positive integer: ";
	cin >> n;
	
	for (int i = 1; i <= n; ++i) {
		sum += i;
	}
	
	cout << "Sum = " << sum;
	return 0;
}
\end{sourcecode}

\begin{sourcecode}{docker}{Docker.}
version: '2'
services:
web:
build: .
ports:
- "5000:5000"
volumes:
- .:/code
- logvolume01:/var/log
links:
- redis
redis:
image: redis
volumes:
logvolume01: {}
\end{sourcecode}

\begin{sourcecode}{plaintext}{Resultado del análiis con TEFAME.}
TEFAME - Toolbox para Elemento Finitos y Analisis
Matricial de Estructuras en MATLAB

Propiedades de entrada modelo:

Nodos: 
Numero de nodos: 4 
Coordenadas nodo N1: 0 0
Coordenadas nodo N2: 800 0
Coordenadas nodo N3: 400 400
Coordenadas nodo N4: 400 800

Elementos: 
Numero de elementos: 6 
Elemento E1:	Largo: 800         Area: 20        Eo: 200000    
Elemento E2:	Largo: 565.6854    Area: 20        Eo: 200000    
Elemento E3:	Largo: 565.6854    Area: 20        Eo: 200000    
Elemento E4:	Largo: 894.4272    Area: 20        Eo: 200000    
Elemento E5:	Largo: 400         Area: 20        Eo: 200000    
Elemento E6:	Largo: 894.4272    Area: 20        Eo: 200000    

Resultados del analisis:

Desplazamientos nodos: 
Desplazamientos nodo N1: 0 0
Desplazamientos nodo N2: 0.016 0
Desplazamientos nodo N3: 0.008 -0.013
Desplazamientos nodo N4: 0.053 -0.016

Reacciones: 
Reacciones nodo N1: -80 -20
Reacciones nodo N2: 0 140
Reacciones nodo N3: 0 0
Reacciones nodo N4: 0 0

Esfuerzos Elementos: 
Elemento E1: -78.4273       TRACCION
Elemento E2: 23.836         COMPRESION
Elemento E3: 23.836         COMPRESION
Elemento E4: -41.2047       TRACCION
Elemento E5: 33.7093        COMPRESION
Elemento E6: 137.6807       COMPRESION
\end{sourcecode}

\begin{sourcecode}[\label{codigo-html5}]{html5}{Ejemplo en HTML5.}
<!DOCTYPE html>
<html>
<head>
	<title>Página</title>
</head>
<body>
	<style>
		.titulo {
			color: #ff0000;
		}
	</style>
	<div class="titulo">Hola</div>
</body>
</html>
\end{sourcecode}

\begin{sourcecode}[\label{codigo-matlab}]{matlab}{Ejemplo en Matlab.}
% Se crea gráfico
f = figure(1);
hold on;
movegui(f, 'center');
xlabel('td/Tn'); ylabel('FAD=Umax/Uf0');
title('Espectro de pulso de desplazamiento');

for j = 1:length(BETA)
	fad = ones(1, NDATOS); % Arreglo para el FAD, uno para cada r
	for i = 1:NDATOS
		[t, u_t, ~, ~] = main(BETA(j), r(i), M, K, F0, 0);
		fad(i) = max(abs(u_t)) / uf0;
	end
mx = find(fad == max(fad(:)));
fprintf('BETA=%.2f, MAX: FAD=%.3f, TD/TN=%.3f\n', BETA(j), fad(mx), tdtn(mx));
plot(tdtn, fad, 'DisplayName', strcat('\beta=', sprintf('%.2f', BETA(j))));
end	
\end{sourcecode}

\begin{sourcecode}[]{xml}{Ejemplo xml.}
<?xml version="1.0" encoding="utf-8"?>
<xs:schema attributeFormDefault="unqualified" elementFormDefault="qualified"
xmlns:xs="http://www.w3.org/2001/XMLSchema">
	<xs:element name="points">
		<xs:complexType>
			<xs:sequence>
				<xs:element maxOccurs="unbounded" name="point">
					<xs:complexType>
						<xs:attribute name="x" type="xs:unsignedShort" use="required" />
						<xs:attribute name="y" type="xs:unsignedShort" use="required" />
					</xs:complexType>
				</xs:element>
			</xs:sequence>
		</xs:complexType>
	</xs:element>
</xs:schema>
\end{sourcecode}

\begin{sourcecode}[\label{codigo-python}]{python}{Ejemplo en Python.}
import numpy as np

def incmatrix(genl1, genl2):
m = len(genl1)
n = len(genl2)
M = None # Comentario 1
VT = np.zeros((n*m, 1), int) # Comentario 2
\end{sourcecode}

\begin{sourcecode}[\label{codigo-java}]{java}{Ejemplo en Java.}
import java.io.IOException; 
import javax.servlet.*;

// Hola mundo
public class Hola extends GenericServlet {
	public void service(ServletRequest request, ServletResponse response)
	throws ServletException, IOException{
		response.setContentType("text/html");
		PrintWriter pw = response.getWriter();
		pw.println("Hola, mundo!");
		pw.close();
	}
}
\end{sourcecode}

\begin{sourcecode}{js}{Ejemplo en Javascript.}
$.urlParam = function (name) {
	let results = new RegExp('[\?&]' + name + '=([^&#]*)').exec(window.location.href);
	if (results == null) {
		return null;
	} else {
		return decodeURI(results[1]) || 0;
	}
};
\end{sourcecode}

\begin{sourcecode}{json}{Un arreglo en JSON.}
{"menu": {
	"id": "file",
	"value": "File",
	"popup": {
		"menuitem": [
		{"value": "New", "onclick": "CreateNewDoc()"},
		{"value": "Open", "onclick": "OpenDoc()"},
		{"value": "Close", "onclick": "CloseDoc()"}
		]
	}
}}
\end{sourcecode}

\begin{sourcecode}{latex}{Imágenes múltiples.}
\begin{images}[\label{imagenmultiple}]{Ejemplo de imagen múltiple.}
	\addimage{ejemplos/test-image}{width=6.5cm}{Ciudad}
	\addimage{ejemplos/test-image-wrap}{width=5cm}{Apolo}
	\addimage{ejemplos/test-image}{width=12cm}{Ciudad más grande}
\end{images}
\end{sourcecode}

\newpage
\begin{sourcecode}[\label{ejemplito-perl}]{perl}{Algo de perl.}
#!/usr/bin/perl
use strict;
use warnings;

# first, create your message
use Email::MIME;
my $message = Email::MIME->create(
  header_str => [
    From    => 'you@example.com',
    To      => 'friend@example.com',
    Subject => 'Happy birthday!',
  ],
  attributes => {
    encoding => 'quoted-printable',
    charset  => 'ISO-8859-1',
  },
  body_str => "Happy birthday to you!\n",
);

# send the message
use Email::Sender::Simple qw(sendmail);
sendmail($message);
\end{sourcecode}

\newpage
\begin{sourcecode}{php}{Ejemplo php.}
<?php
$target_dir = "uploads/";
$target_file = $target_dir . basename($_FILES["fileToUpload"]["name"]);
$uploadOk = 1;
$imageFileType = strtolower(pathinfo($target_file,PATHINFO_EXTENSION));
// Check if image file is a actual image or fake image
if(isset($_POST["submit"])) {
    $check = getimagesize($_FILES["fileToUpload"]["tmp_name"]);
    if($check !== false) {
        echo "File is an image - " . $check["mime"] . ".";
        $uploadOk = 1;
    } else {
        echo "File is not an image.";
        $uploadOk = 0;
    }
}
?>
\end{sourcecode}

\newpage
\begin{sourcecode}[]{ruby}{Ejemplo con ruby.}
class DataFile < ActiveRecord::Base
    attr_accessor :upload

  def self.save_file(upload)   

    file_name = upload['datafile'].original_filename  if  (upload['datafile'] !='')    
    file = upload['datafile'].read    

    file_type = file_name.split('.').last
    new_name_file = Time.now.to_i
    name_folder = new_name_file
    new_file_name_with_type = "#{new_name_file}." + file_type

    image_root = "#{RAILS_CAR_IMAGES}"


    Dir.mkdir(image_root + "#{name_folder}");
      File.open(image_root + "#{name_folder}/" + new_file_name_with_type, "wb")  do |f|  
        f.write(file) 
      end

  end
end
\end{sourcecode}

\newpage
\begin{sourcecode}{sql}{Merge two tables.}
SELECT ChargeNum, CategoryID, SUM(Hours)
FROM KnownHours
GROUP BY ChargeNum, CategoryID
UNION ALL
SELECT ChargeNum, 'Unknown' AS CategoryID, SUM(Hours)
FROM UnknownHours
GROUP BY ChargeNum
\end{sourcecode}

\begin{sourcecode}{scala}{Código en scala.}
object Test {
	def main(args: Array[String]) {
		var a = 0;
		// for loop execution with a range
		for( a <- 1 to 10){
			println( "Value of a: " + a );
		}
	}
}
\end{sourcecode}

\begin{sourcecode}{kotlin}{Kotlin en acción.}
/* Block comment */
package hello
import kotlin.collections.* // line comment

/**
* Doc comment here for `SomeClass`
* @see Iterator#next()
*/
@Deprecated("Deprecated class")
private class MyClass<out T : Iterable<T>>(var prop1 : Int) {
	fun foo(nullable : String?, r : Runnable, f : () -> Int, 
	fl : FunctionLike, dyn: dynamic) {
		println("length\nis ${nullable?.length} \e")
		val ints = java.util.ArrayList<Int?>(2)
		ints[0] = 102 + f() + fl()
		val myFun = { -> "" };
		var ref = ints.size
		ints.lastIndex + globalCounter
		ints.forEach lit@ {
			if (it == null) return@lit
			println(it + ref)
		}
		dyn.dynamicCall()
		dyn.dynamicProp = 5
	}
	
	val test = """
		hello
		world
		kotlin
	"""
	override fun hashCode(): Int {
		return super.hashCode() * 31
	}
}
fun Int?.bar() {
	if (this != null) {
		println(message = toString())
	}
	else {
		println(this.toString())
	}
}
var globalCounter : Int = 5
get = field
abstract class Abstract {
}
object Obj
enum class E { A, B }
interface FunctionLike {
	operator fun invoke() = 1
}
\end{sourcecode}

\begin{sourcecodep}{xml}{firstnumber=100, basicstyle={\fontsize{7}{10}\selectfont\ttfamily}}{}
<?xml version="1.0" encoding="utf-8"?>
<xs:schema attributeFormDefault="unqualified" elementFormDefault="qualified"
   xmlns:xs="http://www.w3.org/2001/XMLSchema">
  <xs:element name="points">
    <xs:complexType>
      <xs:sequence>
        <xs:element maxOccurs="unbounded" name="point">
          <xs:complexType>
            <xs:attribute name="x" type="xs:unsignedShort" use="required" />
            <xs:attribute name="y" type="xs:unsignedShort" use="required" />
          </xs:complexType>
        </xs:element>
      </xs:sequence>
    </xs:complexType>
  </xs:element>
</xs:schema>
\end{sourcecodep}

\begin{sourcecode}{pseudocode}{Algoritmo.}
input: int N, int D % Comentario
output: int // Comentario 2
begin # key
	LET $\gets$ 0 /* comentario 1 */
	while N $\geq$ D /** comentario 2 */
		N $\gets$ N - D
		res $\gets$ res + 1      
	end
	return res
end    
\end{sourcecode}

\newpage
\begin{sourcecode}{glsl}{Noise shader.}
#ifdef GL_ES
precision mediump float;
#endif

uniform vec2 u_resolution;
uniform vec2 u_mouse;
uniform float u_time;

// 2D Random
float random (in vec2 st) {
	return fract(sin(dot(st.xy,
	vec2(12.9898,78.233)))
	* 43758.5453123);
}

// 2D Noise based on Morgan McGuire @morgan3d
// https://www.shadertoy.com/view/4dS3Wd
float noise (in vec2 st) {
	vec2 i = floor(st);
	vec2 f = fract(st);
	
	// Four corners in 2D of a tile
	float a = random(i);
	float b = random(i + vec2(1.0, 0.0));
	float c = random(i + vec2(0.0, 1.0));
	float d = random(i + vec2(1.0, 1.0));
	
	// Smooth Interpolation
	
	// Cubic Hermine Curve.  Same as SmoothStep()
	vec2 u = f*f*(3.0-2.0*f);
	// u = smoothstep(0.,1.,f);
	
	// Mix 4 coorners percentages
	return mix(a, b, u.x) +
	(c - a)* u.y * (1.0 - u.x) +
	(d - b) * u.x * u.y;
}

void main() {
	vec2 st = gl_FragCoord.xy/u_resolution.xy;
	
	// Scale the coordinate system to see
	// some noise in action
	vec2 pos = vec2(st*5.0);
	
	// Use the noise function
	float n = noise(pos);
	
	gl_FragColor = vec4(vec3(n), 1.0);
}
\end{sourcecode}

% \newpage
% \importsourcecode{latex}{test/general.tex}{Carga código fuente archivo LaTeX.}
% \importsourcecode{latex}{lib/etc/example.tex}{Carga código fuente ejemplo LaTeX.}
% \importsourcecode{latex}{lib/cfg/init.tex}{Carga código fuente init LaTeX.}