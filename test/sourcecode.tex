\begin{sourcecode}[]{opensees}{Programa en TCL+OpenSees.}
# --------------------------------------------------------------------------------------------------
# Example 6. 2D RC Frame
#		Silvia Mazzoni & Frank McKenna, 2006
# nonlinearBeamColumn element, inelastic fiber section
#

# SET UP ----------------------------------------------------------------------------
wipe;				# clear memory of all past model definitions
model BasicBuilder -ndm 2 -ndf 3;	# Define the model builder, ndm=#dimension, ndf=#dofs
set dataDir Data;			# set up name of data directory (can remove this)
file mkdir $dataDir; 			# create data directory
set GMdir "../GMfiles/";			# ground-motion file directory
source LibUnits.tcl;			# define units
source DisplayPlane.tcl;		# procedure for displaying a plane in model
source DisplayModel2D.tcl;		# procedure for displaying 2D perspective of model
source BuildRCrectSection.tcl;		# procedure for definining RC fiber section

# define GEOMETRY -------------------------------------------------------------
# define structure-geometry paramters
set LCol [expr 14*$ft];		# column height
set LBeam [expr 24*$ft];		# beam length
set NStory 3;			# number of stories above ground level -------------- you can change this.
set NBay 3;			# number of bays (max 9) ------------------------------you can change this.

# define NODAL COORDINATES
for {set level 1} {$level <=[expr $NStory+1]} {incr level 1} {
	set Y [expr ($level-1)*$LCol];
	for {set pier 1} {$pier <= [expr $NBay+1]} {incr pier 1} {
		set X [expr ($pier-1)*$LBeam];
		set nodeID [expr $level*10+$pier]
		node $nodeID $X $Y;		# actually define node
	}
}


# determine support nodes where ground motions are input, for multiple-support excitation
set iSupportNode ""
set level 1
for {set pier 1} {$pier <= [expr $NBay+1]} {incr pier 1} {
	set nodeID [expr $level*10+$pier]
	lappend iSupportNode $nodeID
}


# BOUNDARY CONDITIONS
fixY 0.0 1 1 0;		# pin all Y=0.0 nodes

# calculated MODEL PARAMETERS, particular to this model
puts "Number of Stories: $NStory Number of bays: $NBay"
# Set up parameters that are particular to the model for displacement control
set IDctrlNode [expr ($NStory+1)*10+1];		# node where displacement is read for displacement control
set IDctrlDOF 1;		# degree of freedom of displacement read for displacement control
set LBuilding [expr $NStory*$LCol];	# total building height

# Define SECTIONS -------------------------------------------------------------
set SectionType FiberSection;		# options: Elastic FiberSection

# define section tags:
set ColSecTag 1
set BeamSecTag 2

# Section Properties:
set HCol [expr 24*$in];		# square-Column width
set BCol $HCol
set HBeam [expr 42*$in];		# Beam depth -- perpendicular to bending axis
set BBeam [expr 24*$in];		# Beam width -- parallel to bending axis

if {$SectionType == "Elastic"} {
	# material properties:
	set fc 4000*$psi;			# concrete nominal compressive strength
	set Ec [expr 57*$ksi*pow($fc/$psi,0.5)];	# concrete Young's Modulus
	# column section properties:
	set AgCol [expr $HCol*$BCol];		# rectuangular-Column cross-sectional area
	set IzCol [expr 0.5*1./12*$BCol*pow($HCol,3)];	# about-local-z Rect-Column gross moment of inertial
	# beam sections:
	set AgBeam [expr $HBeam*$BBeam];		# rectuangular-Beam cross-sectional area
	set IzBeam [expr 0.5*1./12*$BBeam*pow($HBeam,3)];	# about-local-z Rect-Beam cracked moment of inertial
	
	section Elastic $ColSecTag $Ec $AgCol $IzCol 
	section Elastic $BeamSecTag $Ec $AgBeam $IzBeam 
	
} elseif {$SectionType == "FiberSection"} {
	# MATERIAL parameters 
	source LibMaterialsRC.tcl;	# define library of Reinforced-concrete Materials
	
	# FIBER SECTION properties 
	# Column section geometry:
	set cover [expr 2.5*$in];	# rectangular-RC-Column cover
	set numBarsTopCol 8;		# number of longitudinal-reinforcement bars on top layer
	set numBarsBotCol 8;		# number of longitudinal-reinforcement bars on bottom layer
	set numBarsIntCol 6;		# TOTAL number of reinforcing bars on the intermediate layers
	set barAreaTopCol [expr 1.*$in*$in];	# longitudinal-reinforcement bar area
	set barAreaBotCol [expr 1.*$in*$in];	# longitudinal-reinforcement bar area
	set barAreaIntCol [expr 1.*$in*$in];	# longitudinal-reinforcement bar area
	
	set numBarsTopBeam 6;		# number of longitudinal-reinforcement bars on top layer
	set numBarsBotBeam 6;		# number of longitudinal-reinforcement bars on bottom layer
	set numBarsIntBeam 2;		# TOTAL number of reinforcing bars on the intermediate layers
	set barAreaTopBeam [expr 1.*$in*$in];	# longitudinal-reinforcement bar area
	set barAreaBotBeam [expr 1.*$in*$in];	# longitudinal-reinforcement bar area
	set barAreaIntBeam [expr 1.*$in*$in];	# longitudinal-reinforcement bar area
	
	set nfCoreY 20;		# number of fibers in the core patch in the y direction
	set nfCoreZ 20;		# number of fibers in the core patch in the z direction
	set nfCoverY 20;		# number of fibers in the cover patches with long sides in the y direction
	set nfCoverZ 20;		# number of fibers in the cover patches with long sides in the z direction
	# rectangular section with one layer of steel evenly distributed around the perimeter and a confined core.
	BuildRCrectSection $ColSecTag $HCol $BCol $cover $cover $IDconcCore  $IDconcCover $IDSteel $numBarsTopCol $barAreaTopCol $numBarsBotCol $barAreaBotCol $numBarsIntCol $barAreaIntCol  $nfCoreY $nfCoreZ $nfCoverY $nfCoverZ
	BuildRCrectSection $BeamSecTag $HBeam $BBeam $cover $cover $IDconcCore  $IDconcCover $IDSteel $numBarsTopBeam $barAreaTopBeam $numBarsBotBeam $barAreaBotBeam $numBarsIntBeam $barAreaIntBeam  $nfCoreY $nfCoreZ $nfCoverY $nfCoverZ
	
} else {
	puts "No section has been defined"
	return -1
}


# define ELEMENTS
# set up geometric transformations of element
#   separate columns and beams, in case of P-Delta analysis for columns
set IDColTransf 1; # all columns
set IDBeamTransf 2; # all beams
set ColTransfType Linear ;			# options, Linear PDelta Corotational 
geomTransf $ColTransfType $IDColTransf  ; 	# only columns can have PDelta effects (gravity effects)
geomTransf Linear $IDBeamTransf

# Define Beam-Column Elements
set np 5;	# number of Gauss integration points for nonlinear curvature distribution-- np=2 for linear distribution ok
# columns
set N0col 100;	# column element numbers
set level 0
for {set level 1} {$level <=$NStory} {incr level 1} {
	for {set pier 1} {$pier <= [expr $NBay+1]} {incr pier 1} {
		set elemID [expr $N0col  + $level*10 +$pier]
		set nodeI [expr  $level*10 + $pier]
		set nodeJ  [expr  ($level+1)*10 + $pier]
		element nonlinearBeamColumn $elemID $nodeI $nodeJ $np $ColSecTag $IDColTransf;		# columns
	}
}
# beams
set N0beam 200;	# beam element numbers
set M0 0
for {set level 2} {$level <=[expr $NStory+1]} {incr level 1} {
	for {set bay 1} {$bay <= $NBay} {incr bay 1} {
		set elemID [expr $N0beam + $level*10 +$bay]
		set nodeI [expr  $M0 + $level*10 + $bay]
		set nodeJ  [expr  $M0 + $level*10 + $bay+1]
		element nonlinearBeamColumn $elemID $nodeI $nodeJ $np $BeamSecTag $IDBeamTransf;	# beams
	}
}

# Define GRAVITY LOADS, weight and masses
# calculate dead load of frame, assume this to be an internal frame (do LL in a similar manner)
# calculate distributed weight along the beam length
set GammaConcrete [expr 150*$pcf];   		# Reinforced-Concrete floor slabs
set Tslab [expr 6*$in];			# 6-inch slab
set Lslab [expr 2*$LBeam/2]; 			# assume slab extends a distance of $LBeam1/2 in/out of plane
set Qslab [expr $GammaConcrete*$Tslab*$Lslab]; 
set QdlCol [expr $GammaConcrete*$HCol*$BCol];	# self weight of Column, weight per length
set QBeam [expr $GammaConcrete*$HBeam*$BBeam];	# self weight of Beam, weight per length
set QdlBeam [expr $Qslab + $QBeam]; 	# dead load distributed along beam.
set WeightCol [expr $QdlCol*$LCol];  		# total Column weight
set WeightBeam [expr $QdlBeam*$LBeam]; 	# total Beam weight

# assign masses to the nodes that the columns are connected to 
# each connection takes the mass of 1/2 of each element framing into it (mass=weight/$g)
set iFloorWeight ""
set WeightTotal 0.0
set sumWiHi 0.0;		# sum of storey weight times height, for lateral-load distribution
for {set level 2} {$level <=[expr $NStory+1]} {incr level 1} { ;		
	set FloorWeight 0.0
	if {$level == [expr $NStory+1]}  {
		set ColWeightFact 1;		# one column in top story
	} else {
		set ColWeightFact 2;		# two columns elsewhere
	}
	for {set pier 1} {$pier <= [expr $NBay+1]} {incr pier 1} {;
		if {$pier == 1 || $pier == [expr $NBay+1]} {
			set BeamWeightFact 1;	# one beam at exterior nodes
		} else {;
			set BeamWeightFact 2;	# two beams elewhere
		}
		set WeightNode [expr $ColWeightFact*$WeightCol/2 + $BeamWeightFact*$WeightBeam/2]
		set MassNode [expr $WeightNode/$g];
		set nodeID [expr $level*10+$pier]
		mass $nodeID $MassNode 0.0 0.0 0.0 0.0 0.0;			# define mass
		set FloorWeight [expr $FloorWeight+$WeightNode];
	}
	lappend iFloorWeight $FloorWeight
	set WeightTotal [expr $WeightTotal+ $FloorWeight]
	set sumWiHi [expr $sumWiHi+$FloorWeight*($level-1)*$LCol];		# sum of storey weight times height, for lateral-load distribution
}
set MassTotal [expr $WeightTotal/$g];						# total mass

# LATERAL-LOAD distribution for static pushover analysis
# calculate distribution of lateral load based on mass/weight distributions along building height
# Fj = WjHj/sum(WiHi)  * Weight   at each floor j
set iFj ""
for {set level 2} {$level <=[expr $NStory+1]} {incr level 1} { ;	
	set FloorWeight [lindex $iFloorWeight [expr $level-1-1]];
	set FloorHeight [expr ($level-1)*$LCol];
	set NodeFactor [expr $NBay+1];
	lappend iFj [expr $FloorWeight*$FloorHeight/$sumWiHi/$NodeFactor*$WeightTotal];		# per node per floor
}
# create node and load vectors for lateral-load distribution in static analysis
set iFPush ""
set iNodePush ""
for {set level 2} {$level <=[expr $NStory+1]} {incr level 1} {
	set FPush [lindex $iFj [expr $level-1-1]];		# lateral load coefficient
	for {set pier 1} {$pier <= [expr $NBay+1]} {incr pier 1} {
		set nodeID [expr $level*10+$pier]
		lappend iNodePush $nodeID
		lappend iFPush $FPush
	}
}

# Define RECORDERS -------------------------------------------------------------
set FreeNodeID [expr ($NStory+1)*10+($NBay+1)];					# ID: free node
set SupportNodeFirst [lindex $iSupportNode 0];						# ID: first support node
set SupportNodeLast [lindex $iSupportNode [expr [llength $iSupportNode]-1]];			# ID: last support node
set FirstColumn [expr $N0col+1*10+1];							# ID: first column
recorder Node -file $dataDir/DFree.out -time -node $FreeNodeID  -dof 1 2 3 disp;				# displacements of free node
recorder Node -file $dataDir/DBase.out -time -nodeRange $SupportNodeFirst $SupportNodeLast -dof 1 2 3 disp;	# displacements of support nodes


# Define DISPLAY -------------------------------------------------------------
DisplayModel2D NodeNumbers

# define GRAVITY -------------------------------------------------------------
# GRAVITY LOADS # define gravity load applied to beams and columns -- eleLoad applies loads in local coordinate axis
pattern Plain 101 Linear {
	for {set level 1} {$level <=$NStory} {incr level 1} {
		for {set pier 1} {$pier <= [expr $NBay+1]} {incr pier 1} {
			set elemID [expr $N0col  + $level*10 +$pier]
			eleLoad -ele $elemID -type -beamUniform 0 -$QdlCol; 	# COLUMNS
		}
	}
	for {set level 2} {$level <=[expr $NStory+1]} {incr level 1} {
		for {set bay 1} {$bay <= $NBay} {incr bay 1} {
			set elemID [expr $N0beam + $level*10 +$bay]
			eleLoad -ele $elemID  -type -beamUniform -$QdlBeam; 	# BEAMS
		}
	}
}
# Gravity-analysis parameters -- load-controlled static analysis
set Tol 1.0e-8;			# convergence tolerance for test
variable constraintsTypeGravity Plain;		# default;
if {  [info exists RigidDiaphragm] == 1} {
	if {$RigidDiaphragm=="ON"} {
		variable constraintsTypeGravity Lagrange;	#  large model: try Transformation
	};	# if rigid diaphragm is on
};	# if rigid diaphragm exists
constraints $constraintsTypeGravity ;     		# how it handles boundary conditions
numberer RCM;			# renumber dof's to minimize band-width (optimization), if you want to
system BandGeneral ;		# how to store and solve the system of equations in the analysis (large model: try UmfPack)
test NormDispIncr $Tol 6 ; 		# determine if convergence has been achieved at the end of an iteration step
algorithm Newton;			# use Newton's solution algorithm: updates tangent stiffness at every iteration
set NstepGravity 10;  		# apply gravity in 10 steps
set DGravity [expr 1./$NstepGravity]; 	# first load increment;
integrator LoadControl $DGravity;	# determine the next time step for an analysis
analysis Static;			# define type of analysis static or transient
analyze $NstepGravity;		# apply gravity

# ------------------------------------------------- maintain constant gravity loads and reset time to zero
loadConst -time 0.0

puts "Model Built"
\end{sourcecode}

\begin{sourcecode}[]{cuda}{Un poco de cuda.}
__global__ void foo(){
}

__global__ void addKernel(int *c, const int *a, const int *b){
	int i = threadIdx.x;
	c[i] = a[i] + b[i];
}

foo<<<n,m>>>();
\end{sourcecode}

\begin{sourcecode}[]{bash}{Un poco de bash.}
# Muestra toda la información de la batería
function battr-info {
	wrks-scripts
	data=$(ioreg -l -w0 |grep Capacity)
	python2.7 battery_info.py $data
	cd - >> config/.empty
}

function qtest-java {
	if [ -z "${1}" ]; then
	echo-err 'Nombre fuente no definido'
	else
	vim $1
	javac -encoding ISO-8859-1 $1 $2 $3
	blankspace=""
	first=$1
	first=${first/.java/$blankspace}
	java $first
	fi
}

PATH=$PATH:"/Library/Frameworks/Python.framework/Versions/2.7/bin:${PATH}"
PATH=$PATH:"/Applications/Utilities/Lynxlet.app/Contents/Resources/lynx/bin"
export PATH

alias ga='git add '
dig +short myip.opendns.com @resolver1.opendns.com
gcc "$@" -o $first

::claramente esto no funcionará::
sudo x, call y
rem esto es un comentario en windows
history -c rm a, ls -d killall e mv | grep | awk python 'hola'
vim uwu printf 'have you seen him' -z
git doge ssh cp cd
\end{sourcecode}

\newpage
\begin{sourcecode}[]{c}{Ejemplo en C.}
#include <stdio.h>
int main(){
	int i, j, rows;
	
	printf("Enter number of rows: ");
	scanf("%d",&rows);
	
	for(i=1; i<=rows; ++i){
		for(j=1; j<=i; ++j){
			printf("* ");
		}
		printf("\n");
	}
	return 0;
}
\end{sourcecode}

\begin{sourcecode}[]{csharp}{Ejemplo en C\#.}
/*
* C# Program to Get a Number and Display the Sum of the Digits 
*/
using System;
using System.Collections.Generic;
using System.Linq;
using System.Text;

namespace Program
{
	class Program
	{
		static void Main(string[] args)
		{
			int num, sum = 0, r;
			Console.WriteLine("Enter a Number : ");
			num = int.Parse(Console.ReadLine());
			while (num != 0)
			{
				r = num % 10;
				num = num / 10;
				sum = sum + r;
			}
			Console.WriteLine("Sum of Digits of the Number : "+sum);
			Console.ReadLine();
			
		}
	}
}
\end{sourcecode}

\newpage
\begin{sourcecode}{cpp}{Suma en C++.}
#include <iostream>
using namespace std;

int main()
{
	int n, sum = 0;
	
	cout << "Enter a positive integer: ";
	cin >> n;
	
	for (int i = 1; i <= n; ++i) {
		sum += i;
	}
	
	cout << "Sum = " << sum;
	return 0;
}
\end{sourcecode}

\begin{sourcecode}{docker}{Docker.}
version: '2'
services:
web:
build: .
ports:
- "5000:5000"
volumes:
- .:/code
- logvolume01:/var/log
links:
- redis
redis:
image: redis
volumes:
logvolume01: {}
\end{sourcecode}

\begin{sourcecode}{plaintext}{Resultado del análiis con TEFAME.}
TEFAME - Toolbox para Elemento Finitos y Analisis
Matricial de Estructuras en MATLAB

Propiedades de entrada modelo:

Nodos: 
Numero de nodos: 4 
Coordenadas nodo N1: 0 0
Coordenadas nodo N2: 800 0
Coordenadas nodo N3: 400 400
Coordenadas nodo N4: 400 800

Elementos: 
Numero de elementos: 6 
Elemento E1:	Largo: 800         Area: 20        Eo: 200000    
Elemento E2:	Largo: 565.6854    Area: 20        Eo: 200000    
Elemento E3:	Largo: 565.6854    Area: 20        Eo: 200000    
Elemento E4:	Largo: 894.4272    Area: 20        Eo: 200000    
Elemento E5:	Largo: 400         Area: 20        Eo: 200000    
Elemento E6:	Largo: 894.4272    Area: 20        Eo: 200000    

Resultados del analisis:

Desplazamientos nodos: 
Desplazamientos nodo N1: 0 0
Desplazamientos nodo N2: 0.016 0
Desplazamientos nodo N3: 0.008 -0.013
Desplazamientos nodo N4: 0.053 -0.016

Reacciones: 
Reacciones nodo N1: -80 -20
Reacciones nodo N2: 0 140
Reacciones nodo N3: 0 0
Reacciones nodo N4: 0 0

Esfuerzos Elementos: 
Elemento E1: -78.4273       TRACCION
Elemento E2: 23.836         COMPRESION
Elemento E3: 23.836         COMPRESION
Elemento E4: -41.2047       TRACCION
Elemento E5: 33.7093        COMPRESION
Elemento E6: 137.6807       COMPRESION
\end{sourcecode}

\begin{sourcecode}[\label{codigo-html5}]{html5}{Ejemplo en HTML5.}
<!DOCTYPE html>
<html>
<head>
	<title>Página</title>
</head>
<body>
	<style>
		.titulo {
			color: #ff0000;
		}
	</style>
	<div class="titulo">Hola</div>
</body>
</html>
\end{sourcecode}

\begin{sourcecode}[\label{codigo-matlab}]{matlab}{Ejemplo en Matlab.}
% Se crea gráfico
f = figure(1);
hold on;
movegui(f, 'center');
xlabel('td/Tn'); ylabel('FAD=Umax/Uf0');
title('Espectro de pulso de desplazamiento');

for j = 1:length(BETA)
	fad = ones(1, NDATOS); % Arreglo para el FAD, uno para cada r
	for i = 1:NDATOS
		[t, u_t, ~, ~] = main(BETA(j), r(i), M, K, F0, 0);
		fad(i) = max(abs(u_t)) / uf0;
	end
mx = find(fad == max(fad(:)));
fprintf('BETA=%.2f, MAX: FAD=%.3f, TD/TN=%.3f\n', BETA(j), fad(mx), tdtn(mx));
plot(tdtn, fad, 'DisplayName', strcat('\beta=', sprintf('%.2f', BETA(j))));
end	
\end{sourcecode}

\begin{sourcecode}[]{xml}{Ejemplo xml.}
<?xml version="1.0" encoding="utf-8"?>
<xs:schema attributeFormDefault="unqualified" elementFormDefault="qualified"
xmlns:xs="http://www.w3.org/2001/XMLSchema">
	<xs:element name="points">
		<xs:complexType>
			<xs:sequence>
				<xs:element maxOccurs="unbounded" name="point">
					<xs:complexType>
						<xs:attribute name="x" type="xs:unsignedShort" use="required" />
						<xs:attribute name="y" type="xs:unsignedShort" use="required" />
					</xs:complexType>
				</xs:element>
			</xs:sequence>
		</xs:complexType>
	</xs:element>
</xs:schema>
\end{sourcecode}

\begin{sourcecode}[\label{codigo-python}]{python}{Ejemplo en Python.}
import numpy as np

def incmatrix(genl1, genl2):
m = len(genl1)
n = len(genl2)
M = None # Comentario 1
VT = np.zeros((n*m, 1), int) # Comentario 2
\end{sourcecode}

\begin{sourcecode}[\label{codigo-java}]{java}{Ejemplo en Java.}
import java.io.IOException; 
import javax.servlet.*;

// Hola mundo
public class Hola extends GenericServlet {
	public void service(ServletRequest request, ServletResponse response)
	throws ServletException, IOException{
		response.setContentType("text/html");
		PrintWriter pw = response.getWriter();
		pw.println("Hola, mundo!");
		pw.close();
	}
}
\end{sourcecode}

\begin{sourcecode}{js}{Ejemplo en Javascript.}
$.urlParam = function (name) {
	let results = new RegExp('[\?&]' + name + '=([^&#]*)').exec(window.location.href);
	if (results == null) {
		return null;
	} else {
		return decodeURI(results[1]) || 0;
	}
};
\end{sourcecode}

\begin{sourcecode}{json}{Un arreglo en JSON.}
{"menu": {
	"id": "file",
	"value": "File",
	"popup": {
		"menuitem": [
		{"value": "New", "onclick": "CreateNewDoc()"},
		{"value": "Open", "onclick": "OpenDoc()"},
		{"value": "Close", "onclick": "CloseDoc()"}
		]
	}
}}
\end{sourcecode}

\begin{sourcecode}{latex}{Imágenes múltiples.}
\begin{images}[\label{imagenmultiple}]{Ejemplo de imagen múltiple.}
	\addimage{ejemplos/test-image}{width=6.5cm}{Ciudad}
	\addimage{ejemplos/test-image-wrap}{width=5cm}{Apolo}
	\addimage{ejemplos/test-image}{width=12cm}{Ciudad más grande}
\end{images}
\end{sourcecode}

\newpage
\begin{sourcecode}[\label{ejemplito-perl}]{perl}{Algo de perl.}
#!/usr/bin/perl
use strict;
use warnings;

# first, create your message
use Email::MIME;
my $message = Email::MIME->create(
  header_str => [
    From    => 'you@example.com',
    To      => 'friend@example.com',
    Subject => 'Happy birthday!',
  ],
  attributes => {
    encoding => 'quoted-printable',
    charset  => 'ISO-8859-1',
  },
  body_str => "Happy birthday to you!\n",
);

# send the message
use Email::Sender::Simple qw(sendmail);
sendmail($message);
\end{sourcecode}

\newpage
\begin{sourcecode}{php}{Ejemplo php.}
<?php
$target_dir = "uploads/";
$target_file = $target_dir . basename($_FILES["fileToUpload"]["name"]);
$uploadOk = 1;
$imageFileType = strtolower(pathinfo($target_file,PATHINFO_EXTENSION));
// Check if image file is a actual image or fake image
if(isset($_POST["submit"])) {
    $check = getimagesize($_FILES["fileToUpload"]["tmp_name"]);
    if($check !== false) {
        echo "File is an image - " . $check["mime"] . ".";
        $uploadOk = 1;
    } else {
        echo "File is not an image.";
        $uploadOk = 0;
    }
}
?>
\end{sourcecode}

\newpage
\begin{sourcecode}[]{ruby}{Ejemplo con ruby.}
class DataFile < ActiveRecord::Base
    attr_accessor :upload

  def self.save_file(upload)   

    file_name = upload['datafile'].original_filename  if  (upload['datafile'] !='')    
    file = upload['datafile'].read    

    file_type = file_name.split('.').last
    new_name_file = Time.now.to_i
    name_folder = new_name_file
    new_file_name_with_type = "#{new_name_file}." + file_type

    image_root = "#{RAILS_CAR_IMAGES}"


    Dir.mkdir(image_root + "#{name_folder}");
      File.open(image_root + "#{name_folder}/" + new_file_name_with_type, "wb")  do |f|  
        f.write(file) 
      end

  end
end
\end{sourcecode}

\newpage
\begin{sourcecode}{sql}{Merge two tables.}
SELECT ChargeNum, CategoryID, SUM(Hours)
FROM KnownHours
GROUP BY ChargeNum, CategoryID
UNION ALL
SELECT ChargeNum, 'Unknown' AS CategoryID, SUM(Hours)
FROM UnknownHours
GROUP BY ChargeNum
\end{sourcecode}

\begin{sourcecode}{scala}{Código en scala.}
object Test {
	def main(args: Array[String]) {
		var a = 0;
		// for loop execution with a range
		for( a <- 1 to 10){
			println( "Value of a: " + a );
		}
	}
}
\end{sourcecode}

\begin{sourcecode}{kotlin}{Kotlin en acción.}
/* Block comment */
package hello
import kotlin.collections.* // line comment

/**
* Doc comment here for `SomeClass`
* @see Iterator#next()
*/
@Deprecated("Deprecated class")
private class MyClass<out T : Iterable<T>>(var prop1 : Int) {
	fun foo(nullable : String?, r : Runnable, f : () -> Int, 
	fl : FunctionLike, dyn: dynamic) {
		println("length\nis ${nullable?.length} \e")
		val ints = java.util.ArrayList<Int?>(2)
		ints[0] = 102 + f() + fl()
		val myFun = { -> "" };
		var ref = ints.size
		ints.lastIndex + globalCounter
		ints.forEach lit@ {
			if (it == null) return@lit
			println(it + ref)
		}
		dyn.dynamicCall()
		dyn.dynamicProp = 5
	}
	
	val test = """
		hello
		world
		kotlin
	"""
	override fun hashCode(): Int {
		return super.hashCode() * 31
	}
}
fun Int?.bar() {
	if (this != null) {
		println(message = toString())
	}
	else {
		println(this.toString())
	}
}
var globalCounter : Int = 5
get = field
abstract class Abstract {
}
object Obj
enum class E { A, B }
interface FunctionLike {
	operator fun invoke() = 1
}
\end{sourcecode}

\begin{sourcecodep}{xml}{firstnumber=100, basicstyle={\fontsize{7}{10}\selectfont\ttfamily}}{}
<?xml version="1.0" encoding="utf-8"?>
<xs:schema attributeFormDefault="unqualified" elementFormDefault="qualified"
   xmlns:xs="http://www.w3.org/2001/XMLSchema">
  <xs:element name="points">
    <xs:complexType>
      <xs:sequence>
        <xs:element maxOccurs="unbounded" name="point">
          <xs:complexType>
            <xs:attribute name="x" type="xs:unsignedShort" use="required" />
            <xs:attribute name="y" type="xs:unsignedShort" use="required" />
          </xs:complexType>
        </xs:element>
      </xs:sequence>
    </xs:complexType>
  </xs:element>
</xs:schema>
\end{sourcecodep}

\begin{sourcecode}{pseudocode}{Algoritmo.}
input: int N, int D % Comentario
output: int // Comentario 2
begin # key
	LET $\gets$ 0 /* comentario 1 */
	while N $\geq$ D /** comentario 2 */
		N $\gets$ N - D
		res $\gets$ res + 1      
	end
	return res
end    
\end{sourcecode}

\newpage
\begin{sourcecode}{glsl}{Noise shader.}
#ifdef GL_ES
precision mediump float;
#endif

uniform vec2 u_resolution;
uniform vec2 u_mouse;
uniform float u_time;

// 2D Random
float random (in vec2 st) {
	return fract(sin(dot(st.xy,
	vec2(12.9898,78.233)))
	* 43758.5453123);
}

// 2D Noise based on Morgan McGuire @morgan3d
// https://www.shadertoy.com/view/4dS3Wd
float noise (in vec2 st) {
	vec2 i = floor(st);
	vec2 f = fract(st);
	
	// Four corners in 2D of a tile
	float a = random(i);
	float b = random(i + vec2(1.0, 0.0));
	float c = random(i + vec2(0.0, 1.0));
	float d = random(i + vec2(1.0, 1.0));
	
	// Smooth Interpolation
	
	// Cubic Hermine Curve.  Same as SmoothStep()
	vec2 u = f*f*(3.0-2.0*f);
	// u = smoothstep(0.,1.,f);
	
	// Mix 4 coorners percentages
	return mix(a, b, u.x) +
	(c - a)* u.y * (1.0 - u.x) +
	(d - b) * u.x * u.y;
}

void main() {
	vec2 st = gl_FragCoord.xy/u_resolution.xy;
	
	// Scale the coordinate system to see
	// some noise in action
	vec2 pos = vec2(st*5.0);
	
	// Use the noise function
	float n = noise(pos);
	
	gl_FragColor = vec4(vec3(n), 1.0);
}
\end{sourcecode}

\begin{sourcecode}[]{matlab}{Ejemplo en Matlab.}
% Se crea gráfico
%% hola
f = figure(1);
hold on;
movegui(f, 'center');
xlabel('td/Tn'); ylabel('FAD=Umax/Uf0');
title('Espectro de pulso de desplazamiento');

for j = 1:length(BETA)
	fad = ones(1, NDATOS); % Arreglo para el FAD, uno para cada r
	for i = 1:NDATOS
		[t, u_t, ~, ~] = main(BETA(j), r(i), M, K, F0, 0);
		fad(i) = max(abs(u_t)) / uf0;
	end
mx = find(fad == max(fad(:)));
fprintf('BETA=%.2f, MAX: FAD=%.3f, TD/TN=%.3f\n', BETA(j), fad(mx), tdtn(mx));
plot(tdtn, fad, 'DisplayName', strcat('\beta=', sprintf('%.2f', BETA(j))));
end	
\end{sourcecode}

\begin{sourcecode}{r}{Ejemplo en r.}
#
# Genera una correlacion de los generos con todas sus variables
#
corr_data_genre <- function(data, genre) {
	num_genre <- data[data$genre == genre, -c(1:4)]
	corr_genre <- cor(num_genre)
	corrplot(corr_genre, type = "lower", order = "hclust", tl.col = "black", tl.srt = 30, main = toupper(genre), tl.cex = 0.7, mar=c(0,0,1,0))
}

#
# Crea matriz de generos
#
create_track_genre_df <- function(df, title) {
	# Obtiene la lista de generos ordenados
	genre <- data.frame(genre=unique(df$genre))
	track_ids <- unique(df$track_id)
	df_genre <- df[, -c(2:3, 5:15)]
	df_genre <- df_genre[order(df_genre$track_id), ]
	track_genres <- data.frame(track_id=track_ids)
	
	# Crea un data frame indicando para cada tema si tiene cada genero o no
	df_full <- data.frame(track_id=df_genre$track_id,
	'_'=df_genre$genre==genre[1, 1],
	'Opera'=df_genre$genre==genre[1, 1],
	'A_Capella'=df_genre$genre==genre[2, 1],
	'Alternative'=df_genre$genre==genre[3, 1],
	'Blues'=df_genre$genre==genre[4, 1],
	'Dance'=df_genre$genre==genre[5, 1],
	'Pop'=df_genre$genre==genre[6, 1],
	'Electronic'=df_genre$genre==genre[7, 1],
	'RyB'=df_genre$genre==genre[8, 1],
	'Children_Music'=df_genre$genre==genre[9, 1],
	'Folk'=df_genre$genre==genre[10, 1],
	'Anime'=df_genre$genre==genre[11, 1],
	'Rap'=df_genre$genre==genre[12, 1],
	'Classical'=df_genre$genre==genre[13, 1],
	'Reggae'=df_genre$genre==genre[14, 1],
	'Hip_Hop'=df_genre$genre==genre[15, 1],
	'Comedy'=df_genre$genre==genre[16, 1],
	'Country'=df_genre$genre==genre[17, 1],
	'Reggaeton'=df_genre$genre==genre[18, 1],
	'Ska'=df_genre$genre==genre[19, 1],
	'Indie'=df_genre$genre==genre[20, 1],
	'Rock'=df_genre$genre==genre[21, 1],
	'Soul'=df_genre$genre==genre[22, 1],
	'Soundtrack'=df_genre$genre==genre[23, 1],
	'Jazz'=df_genre$genre==genre[24, 1],
	'World'=df_genre$genre==genre[25, 1],
	'Movie'=df_genre$genre==genre[26, 1])
	cols <- sapply(df_full, is.logical)
	df_full[,cols] <- lapply(df_full[,cols], as.numeric)
	
	# Fusiona
	track_genres_df <- aggregate(.~track_id, df_full[,-c(2, 27)], sum)
	
	# Crea un grafico de correlacion
	track_genres_df_noid <- track_genres_df[, -c(1)]
	corr_genre <- cor(track_genres_df_noid)
	corrplot(corr_genre, type = "lower", order = "hclust", tl.col = "black", tl.srt = 30, main = title, tl.cex = 0.7, mar=c(0,0,1,0))
	
	return(track_genres_df)
}

# Crea histograma con todas las variables del dataset
create_hist_plot <- function(df, probability) {
	select_plot <- function(data, tag){
		hist(data,
		col="#00AFBB",
		border="#015f66",
		ylab=NULL,
		xlab=NULL,
		prob=probability,
		main=toupper(tag),
		breaks=50
		)
	}
	par(mfrow=c(3,4))
	select_plot(df$popularity, "popularity")
	select_plot(df$acousticness, "acousticness")
	select_plot(datos_num$danceability, "danceability")
	select_plot(datos_num$duration_ms, "duration_ms")
	select_plot(datos_num$energy, "energy")
	if (probability){
		title(ylab="Probability")
	}else{
		title(ylab="Frequence") 
	}
	select_plot(datos_num$instrumentalness, "instrumentalness")
	select_plot(datos_num$liveness, "liveness")
	select_plot(datos_num$loudness, "loudness")
	select_plot(datos_num$speechiness, "speechiness")
	select_plot(datos_num$tempo, "tempo")
	select_plot(datos_num$valence, "valence")
}

#
# Genera un histograma de varios generos con una variable
#
create_hist_genre <- function(df, genres, var){
	gi <- 0
	colors <- c()
	genr <- c()
	for (g in genres) {
		df_g <- df[df$genre == g, var] # elige el data
		color <- rgb(runif(1), runif(1), runif(1), 0.5)
		if (gi == 0) {
			hist(df_g[[1]], col=color, xlab="", ylab="", main="", border=color, breaks=50)
		} else {
			hist(df_g[[1]], col=color, border=color, breaks=50, add=T, xlab=var)
		}
		colors <- c(colors, color)
		gi <- gi + 1
		genr <- c(genr, g)
	}
	title(main=toupper(var), sub="", xlab=var, ylab="Frecuency")
	op <- par(cex = 0.75)
	legend('topleft', genr, fill = colors, ncol=2)
}

#
# Genera un histograma de varios generos con una variable
#
create_prob_genre <- function(df, genres, var, pos, ncols, fsize){
	if (pos == "l") {
		legp <- "topleft"
	} else if(pos == "c") {
		legp <- "top"
	} else if(pos == "r") {
		legp <- "topright"
	} else if (pos == "bl") {
		legp <- "bottomleft"
	} else if(pos == "bc") {
		legp <- "bottom"
	} else if(pos == "br") {
		legp <- "bottomright"
	} else {
		legp <- "center"
	}
	gi <- 0
	tranp <- rgb(0, 0, 0, 0) # color transparente
	colors <- c()
	genr <- c()
	for (g in genres) {
		df_g <- df[df$genre == g, var] # elige el data
		color <- rgb(runif(1), runif(1), runif(1), 1)
		if (gi == 0) {
			hist(df_g[[1]], col=tranp, xlab="", ylab="", main="", border=tranp, prob = TRUE, breaks=50)
			lines(density(df_g[[1]]), col=color, xlab="", ylab="", main="", lw=2)
		} else {
			hist(df_g[[1]], col=tranp, border=tranp, breaks=50, prob = TRUE, add=T, xlab=var)
			lines(density(df_g[[1]]), col=color, lw=2)
		}
		colors <- c(colors, color)
		gi <- gi + 1
		genr <- c(genr, g)
	}
	title(main=toupper(var), sub="", xlab=var, ylab="Density")
	grid()
	op <- par(cex = fsize)
	legend(legp, genr, fill = colors, ncol=ncols)
}

# Tamaño normal
create_prob_genre_normal <- function(df, genres, var, pos) {
	create_prob_genre(df, genres, var, pos, 2, 0.75)
}

# Tamaño pequeño
create_prob_genre_small <- function(df, genres, var, pos) {
	create_prob_genre(df, genres, var, pos, 4, 0.6)
}

# Tamaño grande
create_prob_genre_large <- function(df, genres, var, pos) {
	create_prob_genre(df, genres, var, pos, 1, 1)
}
\end{sourcecode}

\begin{sourcecode}{css}{Código CSS.}
.fecha-estilo {
	color: #819198;
	font-size: 15px;
	position: relative;
	bottom: 0;
}

.btn {
	-moz-box-shadow: 0 0 5px 0 rgba(0, 0, 0, 0.75);
	-webkit-box-shadow: 0 0 5px 0 rgba(0, 0, 0, 0.75);
	background-color: rgba(100, 100, 100, 0.4);
	border-radius: 0.3rem;
	border: 1px solid rgba(255, 255, 255, 0.2);
	box-shadow: 0 0 5px 0 rgba(0, 0, 0, 0.75);
	color: rgba(255, 255, 255, 1);
	display: inline-block;
	margin-bottom: 1rem;
	margin-left: 0;
	margin-right: 0.5rem;
	opacity: 1.0;
	outline: none;
	transition: color 0.2s, background-color 0.2s, border-color 0.2s;
}

.btn:hover {
	opacity: 1.0;
}

.btn-pro {
	-moz-box-shadow: 0 0 5px 0 rgba(10, 10, 10, 0.75);
	-webkit-box-shadow: 0 0 5px 0 rgba(10, 10, 10, 0.75);
	background-color: rgba(210, 210, 210, 0.85);
	border-radius: 0.3rem;
	border: 1px solid rgba(255, 255, 255, 0.5);
	box-shadow: 0 0 5px 0 rgba(10, 10, 10, 0.75);
	color: rgba(20, 20, 20, 1);
	display: inline-block;
	font-weight: bolder;
	margin-bottom: 1rem;
	margin-left: 0;
	margin-right: 0.5rem;
	opacity: 0.9;
	outline: none;
	transition: color 0.2s, background-color 0.2s, border-color 0.2s;
}
\end{sourcecode}

% \newpage
% \importsourcecode{latex}{test/general.tex}{Carga código fuente archivo LaTeX.}
% \importsourcecode{latex}{lib/etc/example.tex}{Carga código fuente ejemplo LaTeX.}
% \importsourcecode{latex}{lib/cfg/init.tex}{Carga código fuente init LaTeX.}