% Template:     Informe/Reporte LaTeX
% Advertencia:  Documento generado automáticamente a partir del main.tex y
%               los archivos .tex de la carpeta lib/
% Versión:      3.4.5 (12/05/2017)
% Codificación: UTF-8
%
% Autor: Pablo Pizarro R.
%        Facultad de Ciencias Físicas y Matemáticas.
%        Universidad de Chile.
%        pablo.pizarro@ing.uchile.cl, ppizarror.com
%
% Sitio web del proyecto: [http://ppizarror.com/Template-Informe/]
% Licencia: MIT           [https://opensource.org/licenses/MIT]

% CREACIÓN DEL DOCUMENTO, FUENTE E IDIOMA
\documentclass[letterpaper,11pt]{article} % Articulo tamaño carta, 11pt
\usepackage[utf8]{inputenc}               % Codificación UTF-8
\usepackage[T1]{fontenc}                  % Soporta caracteres acentuados
\usepackage{lmodern}                      % Tipografía Computer Modern
\usepackage[spanish]{babel}               % Idioma del documento en español
\def\templateversion{3.4.5}               % Versión del template
                                                                                                          
% INFORMACIÓN DEL DOCUMENTO
\def\nombredelinforme {Título del informe}
\def\temaatratar {Tema a tratar}
\def\fecharealizacion {\today}
\def\fechaentrega {\today}

\def\autordeldocumento {Nombre del autor o grupo}
\def\nombredelcurso {Curso}
\def\codigodelcurso {CO-1234}

\def\nombreuniversidad {Universidad de Chile}
\def\nombrefacultad {Facultad de Ciencias Físicas y Matemáticas}
\def\departamentouniversidad {Departamento de la Universidad}
\def\imagendepartamento {images/departamentos/fcfm}
\def\imagendepartamentoescala {0.2}
\def\localizacionuniversidad {Santiago, Chile}

% INTEGRANTES, PROFESORES Y FECHAS
\def\tablaintegrantes {
\begin{minipage}{1.012\textwidth}
\begin{flushright}
\begin{tabular}{ll}
	Integrantes:
		& \begin{tabular}[t]{@{}l@{}}
			Integrante 1 \\
			Integrante 2 \\
			Integrante 3
		\end{tabular} \\
	Profesores:
		& \begin{tabular}[t]{@{}l@{}}
			Profesor 1 \\
			Profesor 2
		\end{tabular} \\
	Auxiliares:
		& \begin{tabular}[t]{@{}l@{}}
			Auxiliar 1 \\
			Auxiliar 2
		\end{tabular}\\
	Ayudantes:
		& \begin{tabular}[t]{@{}l@{}}
			Ayudante 1 \\
			Ayudante 2
		\end{tabular}\\
	\multicolumn{2}{l}{Ayudante del laboratorio: Ayudante 1} \\
	& \\
	\multicolumn{2}{l}{Fecha de realización: \fecharealizacion} \\
	\multicolumn{2}{l}{Fecha de entrega: \fechaentrega} \\
	\multicolumn{2}{l}{\localizacionuniversidad}
\end{tabular}
\end{flushright}
\end{minipage}}

% CONFIGURACIÓN DE LAS LEYENDAS
\def\captionbottommargin {4}          % Margen inferior de las leyendas [pt]
\def\captiontopmargin {9}             % Margen superior de las leyendas [pt]
\def\centeredcaption {false}          % Leyenda centrada al tener varias líneas
\def\captionlessmargin {0.1cm}        % Margen sup/inf. de fig. si no hay leyenda
\def\captionlrmargin {2.7}            % Márgenes izq/der de las leyendas [cm]
\def\figurecaptiontop {false}         % Leyenda arriba de las imágenes
\def\showsectiononcaption {false}     % Muestra el número de sección en las leyendas

% CONFIGURACIÓN DEL ÍNDICE
\def\indextitlemargin {7pt}           % Margen títulos en índice
\def\indexdepth {3}                   % Profundidad del índice
\def\showindex {true}                 % Muestra el índice
\def\showindexofcontents {true}       % Muestra la lista de contenidos
\def\showindexoffigures {true}        % Muestra la lista de figuras
\def\showindexofsourcecode {false}    % Muestra la lista de códigos fuente
\def\showindexoftables {true}         % Muestra la lista de tablas

% MÁRGENES DE FIGURAS
\def\marginbottomimages {-0.2cm}      % Margen inferior figura
\def\marginfloatimages {-13pt}        % Margen superior figura flotante
\def\margintopimages {0.0cm}          % Margen superior figura

% OTRAS CONFIGURACIONES
\def\defaultimagefolder {images/}     % Carpeta de las imágenes
\def\defaultinterline {1.0}           % Interlineado por defecto
\def\defaultnewlinesize {11pt}        % Tamaño del salto de línea
\def\showborderonlinks {false}        % Muestra un recuadro en cada enlace
\def\tablepadding {1.0}               % Padding de las tablas

% ESTILO DE TÍTULOS
\def\etypefontsubsubtitle {\bfseries} % Estilo sub-subtítulos
\def\etypefontsubsubtitlei{\bfseries} % Estilo sub-subtítulos en el índice
\def\etypefontsubtitle {\bfseries}    % Estilo subtítulos
\def\etypefontsubtitlei {\bfseries}   % Estilo subtítulos en el índice
\def\etypefonttitle {\bfseries}       % Estilo títulos
\def\etypefonttitlei {\bfseries}      % Estilo títulos en el índice
\def\typefontsubsubtitle {\large}     % Tamaño sub-subtítulos
\def\typefontsubsubtitlei {\large}    % Tamaño sub-subtítulos en el índice
\def\typefontsubtitle {\Large}        % Tamaño subtítulos
\def\typefontsubtitlei {\Large}       % Tamaño subtítulos en el índice
\def\typefonttitle {\huge}            % Tamaño títulos
\def\typefonttitlei {\huge}           % Tamaño títulos en el índice

% NOMBRE DE OBJETOS
\def\nameportraitpage {Portada}       % Etiqueta página de la portada
\def\namereferences {Referencias}     % Nombre de la sección de referencias
\def\nomltcont {Índice de Contenidos} % Nombre del índice de contenidos
\def\nomltfigure {Lista de Figuras}   % Nombre del índice de la lista de figuras
\def\nomltsrc{Lista de Código Fuente} % Nombre del índice de la lista de código
\def\nomlttable {Lista de Tablas}     % Nombre del índice de la lista de tablas
\def\nomltwsrc {Código}               % Etiqueta leyenda del código fuente
\def\nomltwfigure {Figura}            % Etiqueta leyenda de las figuras
\def\nomltwtable {Tabla}              % Etiqueta leyenda de las tablas

% REFERENCIAS
\def\referenceassection {false}       % Considera las referencias como sección
\def\twocolumnreferences {false}      % Referencias en dos columnas
\def\typereference {apa}              % Tipo de referencias

% CONFIGURACIÓN DE CONTENIDO, PORTADA, HEADERS
\def\gradecodeonportrait {false}      % Muestra el código del curso
\def\showfooter {true}                % Muestra el footer
\def\showheadertitle {true}           % Muestra título de la sección en el header

% MÁRGENES DE PÁGINA
\def\firstpagemargintop {3.8}         % Margen superior página portada [cm]
\def\pagemarginbottom {2.7}           % Margen inferior página [cm]
\def\pagemarginleft {2.54}            % Margen izquierdo página [cm]
\def\pagemarginright {2.54}           % Margen derecho página [cm]
\def\pagemargintop {3.0}              % Margen superior página [cm]

% DECLARACIÓN DE LIBRERÍAS
\usepackage{array}\usepackage{bigstrut}\usepackage{booktabs}\usepackage[makeroom]{cancel}\usepackage{caption}\usepackage{chngcntr}\usepackage{color}\usepackage{colortbl}\usepackage{datetime}\usepackage[inline]{enumitem}\usepackage[bottom,norule,hang]{footmisc}\usepackage{fancyhdr}\usepackage{floatrow}\usepackage{gensymb,textcomp}\usepackage{geometry}\usepackage{graphicx}\usepackage{ifthen}\usepackage{mathtools}\usepackage{multicol}\usepackage{notoccite}\usepackage{pdfpages}\usepackage{lipsum}\usepackage{longtable}\usepackage{listings}\usepackage{rotating}\usepackage{sectsty}\usepackage{selinput}\usepackage{setspace}\usepackage{subfig}\usepackage{ulem}\usepackage{url}\usepackage{wasysym}\usepackage{wrapfig}\usepackage{xcolor}\usepackage{epstopdf}\usepackage{float}\usepackage{multirow}\ifthenelse{\equal{\referenceassection}{true}}{\usepackage{etoolbox}}{}\ifthenelse{\equal{\showborderonlinks}{false}}{\usepackage[pdfencoding=auto,psdextra,hidelinks]{hyperref}}{\usepackage[pdfencoding=auto,psdextra]{hyperref}}

% DECLARACIÓN DE FUNCIONES
\newcommand{\throwerror}[2]{\errmessage{Error: \noexpand#1 #2 (linea \the\inputlineno)}}\newcommand{\emptyvarerr}[3]{\ifx\hfuzz#2\hfuzz\throwerror{#1}{#3}\fi}\newcommand{\referenceindexentry}{\addcontentsline{toc}{section}{\namereferences}\ifthenelse{\equal{\showheadertitle}{true}}{\fancyhead[L]{\nouppercase{\namereferences}}}{}}\newcommand{\newemptypage}{\newpage\null\thispagestyle{empty}\newpage\addtocounter{page}{-1}}\newcommand{\setcaptionmargincm}[1]{\captionsetup{margin=#1cm}}\newcommand{\setpagemargincm}[4]{\newgeometry{left=#1cm, top=#2cm, right=#3cm, bottom=#4cm}}\newcommand{\newp}{\hbadness=10000 \vspace{\defaultnewlinesize} \par}\newcommand{\newpar}[1]{\hbadness=10000 #1 \newp}\newcommand{\newparnl}[1]{#1 \par}\newcommand{\lpow}[2]{{#1}_{#2}}\newcommand{\pow}[2]{{#1}^{#2}}\newcommand{\fracpartial}[2]{\frac{\partial #1}{\partial #2}}\newcommand{\fracdpartial}[2]{\frac{{\partial}^{2} #1}{\partial {#2}^{2}}}\newcommand{\fracnpartial}[3]{\frac{{\partial}^{#3} #1}{\partial {#2}^{#3}}}\newcommand{\fracderivat}[2]{\frac{\text{d} #1}{\text{d} #2}}\newcommand{\fracdderivat}[2]{\frac{{\text{d}}^{2} #1}{\text{d} {#2}^{2}}}\newcommand{\fracnderivat}[3]{\frac{{\text{d}}^{#3} #1}{\text{d} {#2}^{#3}}}\newcommand{\topequal}[2]{\overbrace{#1}^{\mathclap{#2}}}\newcommand{\underequal}[2]{\underbrace{#1}_{\mathclap{#2}}}\newcommand{\topsequal}[2]{\overbracket{#1}^{\mathclap{#2}}}\newcommand{\undersequal}[2]{\underbracket{#1}_{\mathclap{#2}}}\newcommand{\resizeitem}[2]{\emptyvarerr{\resizeitem}{#1}{Tamano del nuevo objeto no definido}\emptyvarerr{\resizeitem}{#2}{Objeto a redimensionar no definido}\resizebox{#1\textwidth}{!}{#2}}\newcommand{\sectionanum}[1]{\emptyvarerr{\sectionanum}{#1}{Titulo no definido}\section*{#1}\addcontentsline{toc}{section}{#1}\ifthenelse{\equal{\showheadertitle}{true}}{\fancyhead[L]{\nouppercase{#1}}}{}\stepcounter{section}}\newcommand{\sectionanumheadless}[1]{\emptyvarerr{\sectionanumnoheadless}{#1}{Titulo no definido}\section*{#1}\addcontentsline{toc}{section}{#1}\stepcounter{section}}\newcommand{\subsectionanum}[1]{\emptyvarerr{\subsectionanum}{#1}{Subtitulo no definido}\subsection*{#1}\addcontentsline{toc}{subsection}{#1}\stepcounter{subsection}}\newcommand{\subsubsectionanum}[1]{\emptyvarerr{\subsubsectionanum}{#1}{Sub-subtitulo no definido}\subsubsection*{#1}\addcontentsline{toc}{subsubsection}{#1}\stepcounter{subsubsection}}\newcommand{\sectionanumnoi}[1]{\emptyvarerr{\sectionanumnoi}{#1}{Titulo no definido}\section*{#1}\ifthenelse{\equal{\showheadertitle}{true}}{\fancyhead[L]{\nouppercase{#1}}}{}\stepcounter{section}}\newcommand{\sectionanumnoiheadless}[1]{\emptyvarerr{\sectionanumnoi}{#1}{Titulo no definido}\section*{#1}\ifthenelse{\equal{\showheadertitle}{true}}{\fancyhead[L]{\nouppercase{#1}}}{}\stepcounter{section}}\newcommand{\subsectionanumnoi}[1]{\emptyvarerr{\subsectionanumnoi}{#1}{Subtitulo no definido}\subsection*{#1}\stepcounter{subsection}}\newcommand{\subsubsectionanumnoi}[1]{\emptyvarerr{\subsubsectionanumnoi}{#1}{Sub-subtitulo no definido}\addcontentsline{toc}{subsubsection}{#1}\subsubsection*{#1}\stepcounter{subsubsection}}\newcommand{\insertindextitle}[2]{\emptyvarerr{\insertindextitle}{#1}{Titulo no definido}\ifx\hfuzz#2\hfuzz\addtocontents{toc}{\protect\addvspace{\indextitlemargin}}\else\addtocontents{toc}{\protect\addvspace{#2 pt}}\fi\addtocontents{toc}{\noindent\hyperref[swpn]{\textbf{#1}}}}\newcommand{\insertequation}[2][]{\emptyvarerr{\insertequation}{#2}{Ecuacion no definida}\vspace{-0.1cm}\begin{equation}\text{#1} #2\end{equation}\vspace{-0.26cm}\par}\newcommand{\insertequationcaptioned}[3][]{\emptyvarerr{\insertequationcaptioned}{#2}{Ecuacion no definida}\ifx\hfuzz#3\hfuzz\insertequation[#1]{#2}\else\vspace{0cm}\begin{equation}\text{#1} #2\end{equation}\begin{center}\vspace{-0.15cm}\textit{#3} \par\vspace{0.05cm}\end{center}\fi}\newcommand{\insertequationgathered}[2][]{\emptyvarerr{\insertequationgathered}{#2}{Ecuacion no definida}\vspace{-0.4cm}\begin{gather}\text{#1} #2\end{gather}\par\vspace{-0.10cm}}\newcommand{\insertequationgatheredcaptioned}[3][]{\emptyvarerr{\insertequationgatheredcaptioned}{#2}{Ecuacion no definida}\ifx\hfuzz#3\hfuzz\insertequationgathered[#1]{#2}\else\vspace{0cm}\begin{gather}\text{#1} #2\end{gather}\begin{center}\vspace{-0.15cm}\textit{#3} \par\end{center}\fi}\newcommand{\insertequationalign}[2][]{\emptyvarerr{\insertequationalign}{#2}{Ecuacion no definida}\vspace{-0.4cm}\begin{align}\text{#1} #2\end{align}\par\vspace{-0.10cm}}\newcommand{\insertequationaligncaptioned}[3][]{\emptyvarerr{\insertequationaligncaptioned}{#2}{Ecuacion no definida}\ifx\hfuzz#3\hfuzz\insertequationalign[#1]{#2}\else\vspace{0cm}\begin{align}\text{#1} #2\end{align}\begin{center}\vspace{-0.15cm}\textit{#3} \par\end{center}\fi}\newcommand{\insertimage}[4][]{\emptyvarerr{\insertimage}{#2}{Direccion de la imagen no definida}\emptyvarerr{\insertimage}{#3}{Parametros de la imagen no definidos}\vspace{\margintopimages}\begin{figure}[H]\centering\includegraphics[#3]{\defaultimagefolder#2}\ifx\hfuzz#4\hfuzz\vspace{\captionlessmargin}\else\caption{#4 #1}\fi\end{figure}\vspace{\marginbottomimages}}\newcommand{\insertimageboxed}[4][]{\emptyvarerr{\insertimageboxed}{#2}{Direccion de la imagen no definida}\emptyvarerr{\insertimageboxed}{#3}{Parametros de la imagen no definidos}\vspace{\margintopimages}\begin{figure}[H]\centering\fbox{\includegraphics[#3]{\defaultimagefolder#2}}\ifx\hfuzz#4\hfuzz\vspace{\captionlessmargin}\else\caption{#4 #1}\fi\end{figure}\vspace{\marginbottomimages}}\newcommand{\insertdoubleimage}[8][]{\emptyvarerr{\insertdoubleimage}{#2}{Direccion de la imagen 1 no definida}\emptyvarerr{\insertdoubleimage}{#3}{Parametros de la imagen 1 no definidos}\emptyvarerr{\insertdoubleimage}{#5}{Direccion de la imagen 2 no definida}\emptyvarerr{\insertdoubleimage}{#6}{Parametros de la imagen 2 no definidos}\vspace{\margintopimages}\captionsetup{margin=0.45cm}\begin{figure}[H] \centering\subfloat[#4]{\includegraphics[#3]{\defaultimagefolder#2}}\hspace{0.2cm}\subfloat[#7]{\includegraphics[#6]{\defaultimagefolder#5}}\setcaptionmargincm{\captionlrmargin}\ifx\hfuzz#8\hfuzz\vspace{\captionlessmargin}\else\caption{#8 #1}\fi\end{figure}\setcaptionmargincm{\captionlrmargin}\vspace{\marginbottomimages}}\newcommand{\insertdoubleeqimage}[7][]{\insertdoubleimage[#1]{#2}{#6}{#3}{#4}{#6}{#5}{#7}}\newcommand{\inserttripleimage}[8][]{\emptyvarerr{\inserttripleimage}{#2}{Direccion de la imagen 1 no definida}\emptyvarerr{\inserttripleimage}{#3}{Parametros de la imagen 1 no definidos}\emptyvarerr{\inserttripleimage}{#4}{Direccion de la imagen 2 no definida}\emptyvarerr{\inserttripleimage}{#5}{Parametros de la imagen 2 no definidos}\emptyvarerr{\inserttripleimage}{#6}{Direccion de la imagen 3 no definida}\emptyvarerr{\inserttripleimage}{#7}{Parametros de la imagen 3 no definidos}\vspace{\margintopimages}\captionsetup{margin=0.45cm}\begin{figure}[H] \centering\subfloat[]{\includegraphics[#3]{\defaultimagefolder#2}}\hspace{0.1cm}\subfloat[]{\includegraphics[#5]{\defaultimagefolder#4}}\hspace{0.1cm}\subfloat[]{\includegraphics[#7]{\defaultimagefolder#6}}\setcaptionmargincm{\captionlrmargin}\ifx\hfuzz#8\hfuzz\vspace{\captionlessmargin}\else\caption{#8 #1}\fi\end{figure}\setcaptionmargincm{\captionlrmargin}\vspace{\marginbottomimages}}\newcommand{\inserttripleeqimage}[6][]{\inserttripleimage[#1]{#2}{#5}{#3}{#5}{#4}{#5}{#6}}\newcommand{\insertquadimage}[7][]{\emptyvarerr{\insertquadimage}{#2}{Direccion de la imagen 1 no definida}\emptyvarerr{\insertquadimage}{#3}{Direccion de la imagen 2 no definida}\emptyvarerr{\insertquadimage}{#4}{Direccion de la imagen 3 no definida}\emptyvarerr{\insertquadimage}{#5}{Direccion de la imagen 4 no definida}\emptyvarerr{\insertquadimage}{#6}{Propiedades de las imagenes no definidos}\vspace{\margintopimages}\captionsetup{margin=0.45cm}\begin{figure}[H] \centering\subfloat[]{\includegraphics[#6]{\defaultimagefolder#2}}\hspace{0.1cm}\subfloat[]{\includegraphics[#6]{\defaultimagefolder#3}}\hspace{0.1cm}\subfloat[]{\includegraphics[#6]{\defaultimagefolder#4}}\hspace{0.1cm}\subfloat[]{\includegraphics[#6]{\defaultimagefolder#5}}\setcaptionmargincm{\captionlrmargin}\ifx\hfuzz#7\hfuzz\vspace{\captionlessmargin}\else\caption{#7 #1}\fi\end{figure}\setcaptionmargincm{\captionlrmargin}\vspace{\marginbottomimages}}\newcommand{\insertpentaimage}[8][]{\emptyvarerr{\insertpentaimage}{#2}{Direccion de la imagen 1 no definida}\emptyvarerr{\insertpentaimage}{#3}{Direccion de la imagen 2 no definida}\emptyvarerr{\insertpentaimage}{#4}{Direccion de la imagen 3 no definida}\emptyvarerr{\insertpentaimage}{#5}{Direccion de la imagen 4 no definida}\emptyvarerr{\insertpentaimage}{#6}{Direccion de la imagen 5 no definida}\emptyvarerr{\insertpentaimage}{#7}{Propiedades de las imagenes no definidas}\vspace{\margintopimages}\captionsetup{margin=0.45cm}\begin{figure}[H] \centering\subfloat[]{\includegraphics[#7]{\defaultimagefolder#2}}\hspace{0.1cm}\subfloat[]{\includegraphics[#7]{\defaultimagefolder#3}}\hspace{0.1cm}\subfloat[]{\includegraphics[#7]{\defaultimagefolder#4}}\hspace{0.1cm}\subfloat[]{\includegraphics[#7]{\defaultimagefolder#5}}\hspace{0.1cm}\subfloat[]{\includegraphics[#7]{\defaultimagefolder#6}}\setcaptionmargincm{\captionlrmargin}\ifx\hfuzz#8\hfuzz\vspace{\captionlessmargin}\else\caption{#8 #1}\fi\end{figure}\setcaptionmargincm{\captionlrmargin}\vspace{\marginbottomimages}}\newcommand{\inserthexaimage}[9][]{\emptyvarerr{\inserthexaimage}{#2}{Direccion de la imagen 1 no definida}\emptyvarerr{\inserthexaimage}{#3}{Direccion de la imagen 2 no definida}\emptyvarerr{\inserthexaimage}{#4}{Direccion de la imagen 3 no definida}\emptyvarerr{\inserthexaimage}{#5}{Direccion de la imagen 4 no definida}\emptyvarerr{\inserthexaimage}{#6}{Direccion de la imagen 5 no definida}\emptyvarerr{\inserthexaimage}{#7}{Direccion de la imagen 6 no definida}\emptyvarerr{\inserthexaimage}{#8}{Propiedades de las imagenes no definidas}\vspace{\margintopimages}\captionsetup{margin=0.45cm}\begin{figure}[H] \centering\subfloat[]{\includegraphics[#8]{\defaultimagefolder#2}}\hspace{0.1cm}\subfloat[]{\includegraphics[#8]{\defaultimagefolder#3}}\hspace{0.1cm}\subfloat[]{\includegraphics[#8]{\defaultimagefolder#4}}\hspace{0.1cm}\subfloat[]{\includegraphics[#8]{\defaultimagefolder#5}}\hspace{0.1cm}\subfloat[]{\includegraphics[#8]{\defaultimagefolder#6}}\hspace{0.1cm}\subfloat[]{\includegraphics[#8]{\defaultimagefolder#7}}\setcaptionmargincm{\captionlrmargin}\ifx\hfuzz#9\hfuzz\vspace{\captionlessmargin}\else\caption{#9 #1}\fi\end{figure}\setcaptionmargincm{\captionlrmargin}\vspace{\marginbottomimages}}\newcommand{\insertimageleft}[5][]{\emptyvarerr{\insertimageleft}{#2}{Direccion de la imagen no definida}\emptyvarerr{\insertimageleft}{#3}{Ancho de la imagen no defindo}\emptyvarerr{\insertimageleft}{#4}{Altura en lineas de la imagen no definida}\begin{wrapfigure}[#4]{l}{#3\textwidth}\setcaptionmargincm{0}\ifthenelse{\equal{\figurecaptiontop}{true}}{}{\vspace{\marginfloatimages}}\centering\includegraphics[width=\linewidth]{\defaultimagefolder#2}\ifx\hfuzz#5\hfuzz\vspace{\captionlessmargin}\else\caption{#5 #1}\fi\setcaptionmargincm{\captionlrmargin}\end{wrapfigure}}\newcommand{\insertimageright}[5][]{\emptyvarerr{\insertimageright}{#2}{Direccion de la imagen no definida}\emptyvarerr{\insertimageright}{#3}{Ancho de la imagen no defindo}\emptyvarerr{\insertimageright}{#4}{Altura en lineas de la imagen no definida}\begin{wrapfigure}[#4]{r}{#3\textwidth}\setcaptionmargincm{0}\ifthenelse{\equal{\figurecaptiontop}{true}}{}{}\centering\includegraphics[width=\linewidth]{\defaultimagefolder#2}\ifx\hfuzz#5\hfuzz\vspace{\captionlessmargin}\else\caption{#5 #1}\fi\setcaptionmargincm{\captionlrmargin}\end{wrapfigure}}

% DECLARACIÓN DE AMBIENTES Y ESTILOS
\definecolor{dkgreen}{rgb}{0,0.6,0}\definecolor{gray}{rgb}{0.5,0.5,0.5}\definecolor{mauve}{rgb}{0.58,0,0.82}\definecolor{mygreen}{rgb}{0,0.6,0}\definecolor{mygray}{rgb}{0.5,0.5,0.5}\definecolor{mymauve}{rgb}{0.58,0,0.82}\definecolor{codegreen}{rgb}{0,0.6,0}\definecolor{codegray}{rgb}{0.5,0.5,0.5}\definecolor{codepurple}{rgb}{0.58,0,0.82}\definecolor{backcolour}{rgb}{0.95,0.95,0.92}\newcolumntype{P}[1]{>{\centering\arraybackslash}p{#1}}\SelectInputMappings{aacute={á},Ntilde={Ñ},Euro={€}}\lstdefinestyle{C}{language=C,numbers=left,stepnumber=1,numbersep=5pt,backgroundcolor=\color{white},showspaces=false,showstringspaces=false,showtabs=false,tabsize=2,captionpos=b,breaklines=true,breakatwhitespace=true,title=\lstname}\lstdefinestyle{Java}{language=Java,aboveskip=3mm,belowskip=3mm,showstringspaces=false,columns=flexible,basicstyle={\small\ttfamily},numbers=left,numberstyle=\tiny\color{gray},keywordstyle=\color{blue},commentstyle=\color{dkgreen},stringstyle=\color{mauve},breaklines=true,breakatwhitespace=true,tabsize=3,backgroundcolor=\color{backcolour}}\lstdefinestyle{Matlab}{language=Matlab,breaklines=true,morekeywords={matlab2tikz},keywordstyle=\color{blue},morekeywords=[2]{1}, keywordstyle=[2]{\color{black}},backgroundcolor=\color{backcolour},identifierstyle=\color{black},stringstyle=\color{mylilas},commentstyle=\color{mygreen},showstringspaces=false,numbers=left,showstringspaces=false,numberstyle=\tiny\color{gray},numbersep=9pt,basicstyle={\small\ttfamily},tabsize=3,breaklines=true,aboveskip=3mm,belowskip=3mm,emph=[1]{for,end,break},emphstyle=[1]\color{red}}\lstdefinestyle{Python}{language=Python,backgroundcolor=\color{backcolour},commentstyle=\color{codegreen},keywordstyle=\color{magenta},numberstyle=\tiny\color{codegray},stringstyle=\color{codepurple},basicstyle=\footnotesize,breakatwhitespace=false,breaklines=true,captionpos=b,keepspaces=true,numbers=left,numbersep=5pt,showspaces=false,showstringspaces=false,showtabs=false,tabsize=3,basicstyle={\small\ttfamily}}\lstset{literate={á}{{\'a}}1 {é}{{\'e}}1 {í}{{\'i}}1 {ó}{{\'o}}1 {ú}{{\'u}}1{Á}{{\'A}}1 {É}{{\'E}}1 {Í}{{\'I}}1 {Ó}{{\'O}}1 {Ú}{{\'U}}1{à}{{\`a}}1 {è}{{\`e}}1 {ì}{{\`i}}1 {ò}{{\`o}}1 {ù}{{\`u}}1{À}{{\`A}}1 {È}{{\'E}}1 {Ì}{{\`I}}1 {Ò}{{\`O}}1 {Ù}{{\`U}}1{ä}{{\"a}}1 {ë}{{\"e}}1 {ï}{{\"i}}1 {ö}{{\"o}}1 {ü}{{\"u}}1{Ä}{{\"A}}1 {Ë}{{\"E}}1 {Ï}{{\"I}}1 {Ö}{{\"O}}1 {Ü}{{\"U}}1{â}{{\^a}}1 {ê}{{\^e}}1 {î}{{\^i}}1 {ô}{{\^o}}1 {û}{{\^u}}1{Â}{{\^A}}1 {Ê}{{\^E}}1 {Î}{{\^I}}1 {Ô}{{\^O}}1 {Û}{{\^U}}1{œ}{{\oe}}1 {Œ}{{\OE}}1 {æ}{{\ae}}1 {Æ}{{\AE}}1 {ß}{{\ss}}1{ű}{{\H{u}}}1 {Ű}{{\H{U}}}1 {ő}{{\H{o}}}1 {Ő}{{\H{O}}}1{ç}{{\c c}}1 {Ç}{{\c C}}1 {ø}{{\o}}1 {å}{{\r a}}1 {Å}{{\r A}}1{€}{{\EUR}}1 {£}{{\pounds}}1}

% CONFIGURACIÓN INICIAL DEL DOCUMENTO
\setlength{\headheight}{64pt}\setcounter{MaxMatrixCols}{20}\setlength{\footnotemargin}{3mm}\renewcommand{\baselinestretch}{\defaultinterline}\setcaptionmargincm{\captionlrmargin}\hfuzz=100pt \vfuzz=100pt\hbadness=2000 \vbadness=\maxdimen\captionsetup{belowskip=\captionbottommargin pt,aboveskip=\captiontopmargin pt}\ifthenelse{\equal{\figurecaptiontop}{true}}{\floatsetup[figure]{capposition=top}}{}\ifthenelse{\equal{\centeredcaption}{true}}{\captionsetup{justification=centering}}{}\bibliographystyle{\typereference}\ifthenelse{\equal{\referenceassection}{true}}{\patchcmd{\thebibliography}{*}{}{}{}}{}\makeatletter\ifthenelse{\equal{\twocolumnreferences}{true}}{\renewenvironment{thebibliography}[1]{\begin{multicols}{2}[\section*{\refname}]\@mkboth{\MakeUppercase\refname}{\MakeUppercase\refname}\list{\@biblabel{\@arabic\c@enumiv}}{\settowidth\labelwidth{\@biblabel{#1}}\leftmargin\labelwidth\advance\leftmargin\labelsep\@openbib@code\usecounter{enumiv}\let\p@enumiv\@empty\renewcommand\theenumiv{\@arabic\c@enumiv}}\sloppy\clubpenalty 4000\@clubpenalty \clubpenalty\widowpenalty 4000\sfcode`\.\@m}{\def\@noitemerr{\@latex@warning{Ambiente `thebibliography' no definido}}\endlist\end{multicols}}}{}\makeatother\hypersetup{pdfauthor={\autordeldocumento},pdftitle={\nombredelinforme},pdfsubject={\temaatratar},pdfkeywords={\nombreuniversidad, \nombredelcurso,\codigodelcurso, \localizacionuniversidad},pdfcreator={ppizarror, pdfLaTeX},pdfproducer={Template LaTeX informe v\templateversion\ | (Pablo Pizarro) ppizarror.com}}

% INICIO DE LAS PÁGINAS
\begin{document}
	
% PORTADA
\newpage\renewcommand{\thepage}{\nameportraitpage}\setpagemargincm{\pagemarginleft}{\firstpagemargintop}{\pagemarginright}{\pagemarginbottom}\pagestyle{fancy} \fancyhf{}\fancyhead[L]{\nombreuniversidad \\ \nombrefacultad \\ \departamentouniversidad}\fancyhead[R]{\includegraphics[scale=\imagendepartamentoescala]{\imagendepartamento}}\ifthenelse{\equal{\gradecodeonportrait}{true}}{\vspace*{3cm}\begin{center}\huge {\nombredelcurso} \\\vspace{0.3cm}\large {Código del curso: \codigodelcurso} \\\vspace{1.5cm}\Huge {\nombredelinforme} \\\vspace{0.3cm}\large {\temaatratar}\end{center}}{\vspace*{5cm}\begin{center}\huge {\nombredelcurso} \\\vspace{1cm}\Huge {\nombredelinforme} \\\vspace{0.3cm}\large {\temaatratar}\end{center}}\vfill\tablaintegrantes

% CONFIGURACIÓN DE PÁGINA Y ENCABEZADOS
\newpage \pagenumbering{Roman}\setcounter{page}{1}\setcounter{footnote}{1}\setpagemargincm{\pagemarginleft}{\pagemargintop}{\pagemarginright}{\pagemarginbottom}\def\arraystretch{\tablepadding}\decimalpoint\ifthenelse{\equal{\showsectiononcaption}{true}}{\counterwithin{equation}{section}\counterwithin{figure}{section}\counterwithin{lstlisting}{section}\counterwithin{table}{section}}{}\setcounter{tocdepth}{\indexdepth}\renewcommand{\sectionmark}[1]{\markboth{#1}{}}\renewcommand{\listfigurename}{\nomltfigure}\renewcommand{\listtablename}{\nomlttable}\renewcommand{\contentsname}{\nomltcont}\renewcommand{\lstlistlistingname}{\nomltsrc}\renewcommand{\tablename}{\nomltwtable}\renewcommand{\figurename}{\nomltwfigure}\renewcommand{\lstlistingname}{\nomltwsrc}\renewcommand\refname{\namereferences}\pagestyle{fancy} \fancyhf{}\ifthenelse{\equal{\showheadertitle}{true}}{\fancyhead[L]{\nouppercase{\rightmark}}}{}\fancyhead[R]{\small \rm \nouppercase{\thepage}}\ifthenelse{\equal{\showfooter}{true}}{\fancyfoot[L]{\small \rm \textit{\nombredelinforme}}\fancyfoot[R]{\small \rm \textit{\codigodelcurso \ \nombredelcurso}}\renewcommand{\footrulewidth}{0.5pt}}{}\renewcommand{\headrulewidth}{0.5pt}\sectionfont{\typefonttitle \etypefonttitle \selectfont}

% =========================== RESUMEN O ABSTRACT ===========================
% Inserta un título sin número y sin cabecera (el header)
\sectionanumheadless{Resumen}

% Ejemplo de dos párrafos en latín, esta linea puede borrarse sin problema
\lipsum[1] \newp \lipsum[3]

% TABLA DE CONTENIDOS - ÍNDICE
\ifthenelse{\equal{\showindex}{true}}{\newpage\sectionfont{\typefonttitle \etypefonttitlei \selectfont}\subsectionfont{\typefontsubtitle \etypefontsubtitlei \selectfont}\subsubsectionfont{\typefontsubsubtitlei \etypefontsubsubtitlei \selectfont}\ifthenelse{\equal{\showindexofcontents}{true}}{\tableofcontents}{}\ifthenelse{\equal{\showindexoffigures}{true}}{\listoffigures}{}\ifthenelse{\equal{\showindexoftables}{true}}{\listoftables}{}\ifthenelse{\equal{\showindexofsourcecode}{true}}{\lstlistoflistings}{}}{}

% CONFIGURACIONES FINALES - INICIO DE LAS SECCIONES
\newpage\ifthenelse{\equal{\showheadertitle}{true}}{\fancyhead[L]{\nouppercase{\leftmark}}}{}\sectionfont{\typefonttitle \etypefonttitle \selectfont}\subsectionfont{\typefontsubtitle \etypefontsubtitle \selectfont}\subsubsectionfont{\typefontsubsubtitle \etypefontsubsubtitle\selectfont}\renewcommand{\thepage}{\arabic{page}}\setcounter{page}{1}\setcounter{section}{0}\setcounter{footnote}{0}

% ========================= INICIO DEL DOCUMENTO =========================

% Template:     Informe/Reporte LaTeX
% Documento:    Archivo de ejemplo
% Versión:      6.1.1 (05/12/2018)
% Codificación: UTF-8
%
% Autor: Pablo Pizarro R. @ppizarror
%        Facultad de Ciencias Físicas y Matemáticas
%        Universidad de Chile
%        pablo.pizarro@ing.uchile.cl, ppizarror.com
%
% Manual template: [https://latex.ppizarror.com/Template-Informe/]
% Licencia MIT:    [https://opensource.org/licenses/MIT/]

% ------------------------------------------------------------------------------
% NUEVA SECCIÓN
% ------------------------------------------------------------------------------
% Las secciones se inician con \section, si se quiere una sección sin número se
% pueden usar las funciones \sectionanum (sección sin número) o la función
% \sectionanumnoi para crear el mismo título sin numerar y sin aparecer en el índice.
\section{Informes con \LaTeX}

	% SUB-SECCIÓN
	% Las sub-secciones se inician con \subsection, si se quiere una sub-sección sin
	% número se pueden usar las funciones \subsectionanum (nuevo subtítulo sin numeración)
	% o la función \subsectionanumnoi para crear el mismo subtítulo sin numerar y sin
	% aparecer en el índice
	\subsection{Una breve introducción}
		
		Este es un párrafo, puede contener múltiples \quotes{Expresiones} así como fórmulas o referencias\footnote{Las referencias se hacen utilizando la expresión \texttt{\textbackslash label}\{etiqueta\}.} a fórmulas como \eqref{eqn:identidad-imposible}. A continuación se muestra un ejemplo de inserción de imágenes o figuras (como la Figura \ref{img:testimage}) con el comando \href{https://latex.ppizarror.com/informe.html#hlp-imagen}{\texttt{\textbackslash insertimage}}:

		% Para insertar una imagen se puede usar la función \insertimage la cual
		% toma un primer parámetro opcional para definir una etiqueta (dentro de
		% los corchetes), luego toma la dirección de la imagen, sus parámetros
		% (en este caso se definió la escala de 0.15) y una leyenda opcional.
		\insertimage[\label{img:testimage}]{ejemplos/test-image.png}{scale=0.15}{Where are you? de \quotes{Internet}.}

		A continuación\footnote{Como se puede observar las funciones \texttt{\textbackslash insert...} añaden un párrafo automáticamente.} se muestra un ejemplo de inserción de ecuaciones simples con el comando \href{https://latex.ppizarror.com/informe.html#hlp-formulae}{\texttt{\textbackslash insertequation}}:

		% Se inserta una ecuación, el primer parámetro entre [] es opcional
		% (permite identificar con una etiqueta para poder referenciarlo después
		% con \ref), seguido de aquello se escribe la ecuación en modo bruto sin signos $.
		\insertequation[\label{eqn:identidad-imposible}]{\pow{a}{k}=\pow{b}{k}+\pow{c}{k} \quad \forall k>2}

		% Se añade párrafo de prueba. Notar que no se requiere añadir un salto
		% de línea después de insertar una ecuación.
		\lipsum[75]

		% Los párrafos se pueden añadir con \newp, esta función se hizo para
		% evitar errores y warnings por parte del compilador de LaTeX.
		\newp Este es un nuevo párrafo insertado con el comando \href{https://latex.ppizarror.com/informe.html#hlp-parrafo}{\texttt{\textbackslash newp}}. Si no te gustan los comandos \texttt{\textbackslash newp}, \texttt{\textbackslash newpar} o \texttt{\textbackslash newparnl} simplemente puedes usar los salto de línea convencionales acompañado de \texttt{\textbackslash par}.

	% SUB-SECCIÓN
	\subsection{Añadiendo tablas}

		También puedes usar tablas, ¡Crearlas es muy fácil!. Puedes usar el plugin \href{https://www.ctan.org/tex-archive/support/excel2latex/}{Excel2Latex} \cite{ref2} de Excel para convertir las tablas a \LaTeX\xspace o bien utilizar el \quotes{creador de tablas online} \cite{ref3}.

		% Tabla generada con el plugin Excel2Latex
		\begin{table}[htbp]
			\centering
			\caption{Ejemplo de tablas.}
			\begin{tabular}{ccc}
				\hline
				\textbf{Columna 1} & \textbf{Columna 2} & \textbf{Columna 3} \bigstrut\\
				\hline
				$\omega$ & $\nu$ & $\delta$ \bigstrut[t]\\
				$\beta$ & $\gamma$ & $\epsilon$ \\
				$\Phi$ & $\Theta$ & $\varSigma$ \bigstrut[b]\\
				\hline
			\end{tabular}
			\label{tab:tabla-1}
		\end{table}


% ------------------------------------------------------------------------------
% NUEVA SECCIÓN
% ------------------------------------------------------------------------------
\newpage
\section{Aquí un nuevo tema}

	% SUB-SECCIÓN
	\subsection{Haciendo informes como un profesional}

		% Se inserta una imagen flotante en la izquierda del documento con
		% \insertimageleft, al igual que las demás funciones, el primer parámetro
		% es opcional, luego viene la ubicación de la imagen, seguido de la escala
		% (un 30% del ancho de página) y por último su leyenda. Para insertar una
		% imagen flotante en la derecha se utiliza \insertimageright usando los
		% mismos parámetros.
		\insertimageleft[\label{img:imagen-izquierda}]{ejemplos/test-image-wrap}{0.3}{Apolo flotando a la izquierda.}

		\lipsum[1]

		% Párrafos de ejemplo
		\newp \lipsum[115]
		\newp \lipsum[2]

		% Agrega una ecuación con leyenda
		\insertequationcaptioned[\label{eqn:formulasinsentido}]{\int_{a}^{b} f(x) \dd{x} = \fracnpartial{f(x)}{x}{\eta} \cdotp \textstyle \sum_{x=a}^{b} f(x)\cancelto{1+\frac{\epsilon}{k}}{(1+\Delta x)}}{Ecuación sin sentido.}

		% Aquí no es necesario usar \newp dado que todas las funciones \insert... añaden un párrafo nuevo por defecto
		\lipsum[115]

		\newp \lipsum[4]

	% Inserta un subtítulo sin número
	\subsection{Otros párrafos más normales}

		% Párrafos
		\lipsum[7]
		\newp \lipsum[2]

		% Se inserta una ecuación larga con el entorno gathered (1 solo número de ecuación)
		\insertgathered[\label{eqn:eqn-larga}]{
			\lpow{\Lambda}{f} = \frac{L\cdot f}{W} \cdot \frac{\pow{\lpow{Q}{e}}{2}}{8 \pow{\pi}{2} \pow{W}{4} g} + \sum_{i=1}^{l} \frac{f \cdot \big( M - d\big)}{l \cdot W} \cdot \frac{\pow{\big(\lpow{Q}{e}- i\cdot Q\big)}{2}}{8 \pow{\pi}{2} \pow{W}{4} g}\\
			Q_e = 2.5Q \cdot \int_{0}^{e} V(x) \dd{x} + \aasin{ \bigg(1+\frac{1}{1-e}\bigg) }
		}

		% Nuevo párrafo
		\lipsum[4]

		% Se inserta un multicols, con esto se pueden escribir párrafos en varias columnas
		\begin{multicols}{2}

			% Párrafo 1
			\lipsum[4]

			% Ecuación encerrada en una caja
			\insertequation[]{ \boxed{f(x) = \fracdpartial{u}{t}} }

			% Párrafo 2 del multicols
			\lipsum[1]

		\end{multicols}

	% SUB-SECCIÓN
	\subsection{Ejemplos de inserción de código fuente}

		% A continuación se crea una función auxiliar, esta es una herramienta
		% extremadamente importante y muy útil. Esta función de ejemplo toma dos
		% parámetros, uno es el lenguaje del código fuente, el segundo el
		% identificador en el manual.
		\newcommand{\insertsrcmanual}[2]{\href{https://latex.ppizarror.com/informe.html\#hlp-srccode\&srctype=#1}{#2}}

		El template permite la inserción de los siguientes lenguajes de programación de forma nativa: \insertsrcmanual{bash}{bash}, \insertsrcmanual{c}{C}, \insertsrcmanual{csharp}{C\#}, \insertsrcmanual{cpp}{C++}, \insertsrcmanual{cuda}{cuda}, \insertsrcmanual{docker}{DOCKER}, \insertsrcmanual{html5}{HTML5}, \insertsrcmanual{java}{Java}, \insertsrcmanual{js}{Javascript}, \insertsrcmanual{json}{JSON}, \insertsrcmanual{kotlin}{Kotlin}, \insertsrcmanual{latex}{LaTeX}, \insertsrcmanual{matlab}{Matlab}, \insertsrcmanual{perl}{Perl}, \insertsrcmanual{php}{PHP}, \insertsrcmanual{plaintext}{Texto plano}, \insertsrcmanual{pseudocode}{Pseudocódigo}, \insertsrcmanual{python}{Python}, \insertsrcmanual{ruby}{Ruby}, \insertsrcmanual{scala}{Scala}, \insertsrcmanual{sql}{SQL} y \insertsrcmanual{xml}{XML}. Para insertar un código fuente se debe usar el entorno \texttt{sourcecode}, o el entorno \texttt{sourcecodep} si es que se quiere utilizar parámetros adicionales. \newp

		A continuación se presenta un ejemplo de inserción de código fuente en Python (Código \ref{codigo-python}), Java (Código \ref{codigo-java}) y Matlab (Código \ref{codigo-matlab}):

% Se define el lenguaje del código. Cuidado: Los códigos en LaTeX son sensibles a las tabulaciones y espacios en blanco
\begin{sourcecode}[\label{codigo-python}]{python}{Ejemplo en Python.}
import numpy as np

def incmatrix(genl1, genl2):
	m = len(genl1)
	n = len(genl2)
	M = None # Comentario 1
	VT = np.zeros((n*m, 1), int) # Comentario 2
\end{sourcecode}

\begin{sourcecode}[\label{codigo-java}]{java}{Ejemplo en Java.}
import java.io.IOException;
import javax.servlet.*;

// Hola mundo
public class Hola extends GenericServlet {
	public void service(ServletRequest request, ServletResponse response)
	throws ServletException, IOException{
		response.setContentType("text/html");
		PrintWriter pw = response.getWriter();
		pw.println("Hola, mundo!");
		pw.close();
	}
}
\end{sourcecode}

\begin{sourcecode}[\label{codigo-matlab}]{matlab}{Ejemplo en Matlab.}
% Se crea gráfico
f = figure(1); hold on;
movegui(f, 'center');
xlabel('td/Tn'); ylabel('FAD=Umax/Uf0');
title('Espectro de pulso de desplazamiento');

for j = 1:length(BETA)
	fad = ones(1, NDATOS); % Arreglo para el FAD
	for i = 1:NDATOS
		[t, u_t, ~, ~] = main(BETA(j), r(i), M, K, F0, 0);
		fad(i) = max(abs(u_t)) / uf0;
	end
end
\end{sourcecode}

	% SUB-SECCIÓN
	\subsection{Añadir múltiples imágenes}

	El template ofrece el entorno \href{https://latex.ppizarror.com/informe.html#hlp-images}{\texttt{images}} que permite insertar múltiples imágenes de una manera muy sencilla \footnote{Desde la versión \texttt{6.0.0} no se da soporte a las funciones de inserción de imágenes múltiples \texttt{\textbackslash insertdoubleimage}, \texttt{\textbackslash insertdoubleeqimage},\texttt{\textbackslash inserttripleimage}, \texttt{\textbackslash inserttripleeqimage}, \texttt{\textbackslash insertquadimage}, \texttt{\textbackslash insertpentaimage} y \texttt{\textbackslash inserthexaimage}.}. Para crear imágenes múltiples se deben usar las siguientes instrucciones:

\begin{sourcecode}{latex}{}
\begin{images}[\label{imagenmultiple}]{Ejemplo de imagen múltiple.}
	\addimage{ejemplos/test-image}{width=6.5cm}{Ciudad.}
	\addimage{ejemplos/test-image-wrap}{width=5cm}{Apolo.}
	\addimagenewline
	\addimage{ejemplos/test-image}{width=12cm}{Ciudad más grande.}
\end{images}
\end{sourcecode}

	Obteniendo así:

	\begin{images}{Ejemplo de imagen múltiple.}
		\addimage{ejemplos/test-image}{width=6.5cm}{Ciudad.}
		\addimage{ejemplos/test-image-wrap}{width=5cm}{Apolo.}
		\addimagenewline
		\addimage{ejemplos/test-image}{width=12cm}{Ciudad más grande.}
	\end{images}


% ------------------------------------------------------------------------------
% NUEVA SECCIÓN
% ------------------------------------------------------------------------------
% Inserta una sección sin número
\sectionanum{Más ejemplos}

	% Inserta un subtítulo sin número
	\subsectionanum{Listas y Enumeraciones}

		Hacer listas enumeradas con \LaTeX\ es muy fácil con el template\footnote{También puedes revisar el manual de las enumeraciones en \url{http://www.texnia.com/archive/enumitem.pdf}.}, para ello debes usar el comando \texttt{\textbackslash begin\{enumerate\}}, cada elemento comienza por \texttt{\textbackslash item}, resultando así:

		\begin{enumerate}
			\item Grecia
			\item Abracadabra
			\item Manzanas
		\end{enumerate}

		También se puede cambiar el tipo de enumeración, se pueden usar letras, números romanos, entre otros. Esto se logra cambiando el \textbf{label} del objeto \texttt{enumerate}. A continuación se muestra un ejemplo usando letras con el estilo \texttt{\textbackslash alph} \footnote{Con \texttt{\textbackslash Alph} las letras aparecen en mayúscula.}, números romanos con \texttt{\textbackslash roman} \footnote{Con \texttt{\textbackslash Roman} los números romanos salen en mayúscula.} o números griegos con \texttt{\textbackslash greek}\footnote{Una característica propia del template, con \texttt{\textbackslash Greek} las letras griegas están escritas en mayúscula.}:

		\begin{multicols}{3}
			\begin{enumeratebf}[label=\alph*) ] % Fuente en negrita
				\item Peras
				\item Manzanas
				\item Naranjas
			\end{enumeratebf}

			\begin{enumerate}[label=\greek*) ]
				\item Matemáticas
				\item Lenguaje
				\item Filosofía
			\end{enumerate}

			\begin{enumerate}[label=\roman*) ]
				\item Rojo
				\item Café
				\item Morado
			\end{enumerate}
		\end{multicols}

		Para hacer listas sin numerar con \LaTeX\ hay que usar el comando \texttt{\textbackslash begin\{itemize\}}, cada elemento empieza por \texttt{\textbackslash item}, resultando:

		\begin{multicols}{3}
			\begin{itemize}[label={--}]
				\item Peras
				\item Manzanas
				\item Naranjas
			\end{itemize}

			\begin{enumerate}[label={*}]
				\item Rojo
				\item Café
				\item Morado
			\end{enumerate}

			\begin{itemize}
				\item Árboles
				\item Pasto
				\item Flores
			\end{itemize}
		\end{multicols}

	% Inserta un subtítulo sin número
	\subsectionanum{Otros}

		Recuerda revisar el manual de todas las funciones y configuraciones de este template visitando el siguiente link: \url{https://latex.ppizarror.com/Template-Informe/}. Si necesitas una ayuda muy específica sobre el template, o si tienes alguna sugerencia, me puedes enviar un correo a \insertemail{pablo.pizarro@ing.uchile.cl}.


% ------------------------------------------------------------------------------
% REFERENCIAS (ESTILO BIBTEX), revisar configuración \stylecitereferences
% ------------------------------------------------------------------------------
\newpage % Salto de página
\begin{references}
	\bibitem{ref1}
	Template Informe en \LaTeX.
	\textit{¡Revisa el manual online de este template!} \\
	\url{https://latex.ppizarror.com/Template-Informe/}

	\bibitem{ref2}
	Excel2Latex.
	\textit{Importa de forma sencilla tus tablas de Excel a \LaTeX.} \\
	\url{https://www.ctan.org/tex-archive/support/excel2latex/}

	\bibitem{ref3}
	Overleaf.
	\textit{Uno de los mejores editores online para \LaTeX, renovado con su versión 2.0.} \\
	\href{https://es.overleaf.com}{\texttt{https://es.overleaf.com/}}
\end{references}


% ------------------------------------------------------------------------------
% ANEXO
% ------------------------------------------------------------------------------
\newpage
\begin{anexo}
	\section{Cálculos realizados}

		\subsection{Metodología}
			\lipsum[1]

			% Imagen, se numerará automáticamente con la letra del anexo según
			% la configuración \appendixindepobjnum
			\insertimage[\label{img:anexo-2}]{ejemplos/test-image.png}{scale=0.2}{Imagen en anexo.}

		\subsection{Resultados}
			\lipsum[10]

			% Tablas
			\begin{table}[htbp]
				\centering
				\caption{Tabla de cálculo.}
				\begin{tabular}{ccc}
					\hline
					\textbf{Elemento} & $\epsilon_i$ & \boldmath{}\textbf{Valor}\unboldmath{} \bigstrut\\
					\hline
					A     & 10    & 3,14$\pi$ \bigstrut[t]\\
					B     & 20    & 6 \\
					C     & 30    & 7 \\
					\end{tabular}
				\label{tab:anexo-1}
			\end{table}

	\newpage
	\section{Más cálculos}

		% Párrafo
		\lipsum[1]

		% Nuevo párrafo con identación
		\newp \lipsum[4]

		% Tabla de encuestas
		\begin{table}[htbp]
			\centering
			\caption{Resultados encuesta.}
			\begin{tabular}{ccc}
				\hline
				\textbf{Herramienta} & \textbf{Nota} & \textbf{Recomendado} \bigstrut\\
				\hline
				\LaTeX & 100\% & Si $\checkmark$ \\
				Microsoft Word \textsuperscript{\textregistered} & 0\% & No $\frownie$\\
			\end{tabular}
			\label{tab:anexo-2}
		\end{table}

\end{anexo}
 % Ejemplo, se puede borrar

% REFERENCIAS
\newpage % Inserta una nueva página
\referenceindexentry % Agrega las referencias al índice
\begin{thebibliography}{99}
	
	\bibitem{ref1}
	Template Informe en \LaTeX.
	\textit{Revisa el manual online de este template!} \\
	\url{http://ppizarror.com/Template-Informe/}
	
	\bibitem{ref2}
	\texttt{Excel2Latex}
	\textit{Importa de forma sencilla tus tablas de Excel a \LaTeX.} \\
	\url{https://www.ctan.org/tex-archive/support/excel2latex/}
	
	\bibitem{ref3}
	Tables Generator.
	\textit{Convierte fácilmente tus tablas, o crea unas con un intuitivo editor de tablas.} \\
	\url{http://www.tablesgenerator.com/}
	
\end{thebibliography}

% FIN DEL DOCUMENTO
\end{document}