% Template:     Informe/Reporte LaTeX
% Advertencia:  Documento generado automáticamente a partir del main.tex y los
%               archivos .tex de la carpeta lib/ para crear un sólo archivo.
% Versión:      3.0.2 (15/04/2017)
% Codificación: UTF-8
%
% Autor: Pablo Pizarro R.
%        Facultad de Ciencias Físicas y Matemáticas.
%        Universidad de Chile.
%        pablo.pizarro@ing.uchile.cl, ppizarror.com
%
% Sitio web del proyecto: [http://ppizarror.com/Template-Informe/]
% Licencia: MIT           [https://opensource.org/licenses/MIT]

% CREACIÓN DEL DOCUMENTO, FUENTE E IDIOMA
\documentclass[letterpaper,11pt]{article} % Articulo tamaño carta, fuente 11
\usepackage[utf8]{inputenc}               % Codificación UTF-8
\usepackage[T1]{fontenc}                  % Soporta caracteres acentuados
\usepackage{lmodern}                      % Tipografía moderna
\usepackage[spanish]{babel}               % Idioma del documento en español
\def\templateversion{3.0.2}               % Versión del template
                    

% INFORMACIÓN DEL DOCUMENTO
\newcommand{\nombredelinforme}{Título del informe}
\newcommand{\temaatratar}{Tema a tratar}
\newcommand{\fecharealizacion}{\today}
\newcommand{\fechaentrega}{\today}

\newcommand{\autordeldocumento}{Nombre del autor o grupo}
\newcommand{\nombredelcurso}{Curso}
\newcommand{\codigodelcurso}{CO-1234}

\newcommand{\nombreuniversidad}{Universidad de Chile}
\newcommand{\nombrefacultad}{Facultad de Ciencias Físicas y Matemáticas}
\newcommand{\departamentouniversidad}{Departamento de la Universidad}
\newcommand{\imagendeldepartamento}{images/departamentos/fcfm}
\newcommand{\imagendeldepartamentoescl}{0.2}
\newcommand{\localizacionuniversidad}{Santiago, Chile}


% INTEGRANTES, PROFESORES Y FECHAS
\newcommand{\tablaintegrantes}{
\begin{minipage}{1.0\textwidth}
\begin{flushright}
\begin{tabular}{ll}
	Integrantes:
		& \begin{tabular}[t]{@{}l@{}}
			Integrante 1 \\
			Integrante 2
		\end{tabular} \\
	Profesores:
		& \begin{tabular}[t]{@{}l@{}}
			Profesor 1 \\
			Profesor 2
		\end{tabular} \\
	Auxiliares:
		& \begin{tabular}[t]{@{}l@{}}
			Auxiliar 1 \\
			Auxiliar 2
		\end{tabular}\\
	Ayudantes:
		& \begin{tabular}[t]{@{}l@{}}
			Ayudante 1 \\
			Ayudante 2
		\end{tabular}\\
	\multicolumn{2}{l}{Ayudante del laboratorio: Ayudante} \\
	& \\
	\multicolumn{2}{l}{Fecha de realización: \fecharealizacion} \\
	\multicolumn{2}{l}{Fecha de entrega: \fechaentrega} \\
	\multicolumn{2}{l}{\localizacionuniversidad}
\end{tabular}
\end{flushright}
\end{minipage}}


% CONFIGURACIONES
\newcommand{\defaultimagefolder}{images/}         % Directorio de las imágenes
\newcommand{\defaultnewlinesize}{11pt}            % Tamaño del salto de línea
\newcommand{\defaultinterlind}{1.0}               % Entrelineado por defecto
\newcommand{\tipofuentetitulo}{\huge}             % Tamaño títulos
\newcommand{\tipofuentesubtitulo}{\Large}         % Tamaño subtítulos
\newcommand{\tipofuentesubsubtitulo}{\large}      % Tamaño sub-subtítulos
\newcommand{\tipofuentetituloi}{\huge}           % Tamaño títulos en el índice
\newcommand{\tipofuentesubtituloi}{\Large}        % Tamaño subtítulos en el índice
\newcommand{\tipofuentesubsubtituloi}{\large}     % Tamaño sub-subtit. en el índ.
\newcommand{\etipofuentetitulo}{\bfseries}        % Estilo títulos
\newcommand{\etipofuentesubtitulo}{\bfseries}     % Estilo subtítulos
\newcommand{\etipofuentesubsubtitulo}{\bfseries}  % Estilo sub-subtítulos
\newcommand{\etipofuentetituloi}{\bfseries}       % Estilo títulos en el índice
\newcommand{\etipofuentesubtituloi}{\bfseries}    % Estilo subtítulos en el índice
\newcommand{\etipofuentesubsubtituloi}{\bfseries} % Estilo sub-subti. en el índice
\newcommand{\tiporeferencias}{apa}                % Tipo de referencias
\newcommand{\nomltcontend}{Índice de Contenidos}  % Nombre del índ. de contenidos
\newcommand{\nomlttablas}{Lista de Tablas}        % Nombre de la lista de tablas
\newcommand{\nomltfiguras}{Lista de Figuras}      % Nombre de la lista de figuras
\newcommand{\nomltsrc}{Lista de Códigos Fuente}   % Nombre del código fuente
\newcommand{\nomltwtablas}{Tabla}                 % Nombre de las tablas
\newcommand{\nomltwfigura}{Figura}                % Nombre de las figuras
\newcommand{\nomltwcodfuente}{Código Fuente}      % Nombre del código fuente
\newcommand{\indexdepth}{3}                       % Profundidad del índice
\newcommand{\tablepadding}{1.1}                   % Padding de las tablas
\newcommand{\defaultcaptionmargin}{2.9}           % Márgenes de las leyendas [cm]
\newcommand{\defaultpagemarginleft}{2.5}          % Margen izquierdo página [cm]
\newcommand{\defaultpagemarginright}{2.5}         % Margen derecho página [cm]
\newcommand{\defaultpagemargintop}{3.0}           % Margen superior página [cm]
\newcommand{\defaultpagemarginbottom}{2.7}        % Margen inferior página [cm]
\newcommand{\defaultfirstpagemargintop}{3.8}      % Margen superior portada [cm]
\newcommand{\defaultmarginfloatimages}{-13pt}     % Margen sup. fig. flotante [pt]
\newcommand{\defaultmargintopimages}{0.0cm}       % Margen superior figura [cm]
\newcommand{\defaultmarginbottomimages}{-0.2cm}   % Margen inferior figura [cm]
\newcommand{\nombrepaginaportada}{Portada}        % Etiqueta página de la portada

% CONFIGURACIONES BOOLEANAS (TRUE, FALSE)
\newcommand{\codigocursoenportada}{false}  % Muestra el código del curso
\newcommand{\showborderonlinks}{false}     % Muestra un recuadro en cada enlace
\newcommand{\showfooter}{true}             % Muestra el footer
\newcommand{\showheadertitle}{true}        % Muestra título de la sección
\newcommand{\showindexofcontents}{true}    % Muestra la lista de contenidos
\newcommand{\showindexoffigures}{true}     % Muestra la lista de figuras
\newcommand{\showindexofsourcecode}{false} % Muestra la lista de códigos fuente
\newcommand{\showindexoftables}{true}      % Muestra la lista de tablas
\newcommand{\twocolumnreferences}{false}   % Referencias en dos columnas


% DECLARACIÓN DE LIBRERÍAS
\usepackage{amsmath}                  % Fórmulas matemáticas
\usepackage{amssymb}                  % Símbolos matemáticos
\usepackage{amsthm}                   % Teoremas matemáticos
\usepackage{array}                    % Añade nuevas características a las tablas
\usepackage{bigstrut}                 % Líneas horizontales en tablas
\usepackage{booktabs}                 % Permite manejar elem. visuales en tablas
\usepackage[makeroom]{cancel}         % Cancelar términos en fórmulas
\usepackage{caption}                  % Leyendas
\usepackage{color}                    % Colores
\usepackage{colortbl}                 % Administración de color en tablas
\usepackage{datetime}                 % Fechas
\usepackage[inline]{enumitem}         % Permite enumerar ítems
\usepackage[bottom, norule]{footmisc} % Estilo pié de página
\usepackage{fancyhdr}                 % Encabezados y pié de páginas
\usepackage{float}                    % Administrador de posiciones de objetos
\usepackage{textcomp, gensymb}        % Simbología común
\usepackage{geometry}                 % Dimensiones y geometría del documento
\usepackage{graphicx}                 % Propiedades extra para los gráficos
\usepackage{ifthen}                   % Permite el manejo de condicionales
\usepackage{mathtools}                % Permite utilizar notaciones matemáticas
\usepackage{multicol}                 % Múltiples columnas
\usepackage{pdfpages}                 % Permite administrar páginas en pdf
\usepackage{lipsum}                   % Permite crear textos dummy
\usepackage{longtable}                % Permite utilizar tablas en varias hojas
\usepackage{listings}                 % Permite añadir código fuente
\usepackage{rotating}                 % Permite rotación de objetos
\usepackage{sectsty}                  % Cambia el estilo de los títulos
\usepackage{selinput}                 % Compatibilidad con acentos
\usepackage{setspace}                 % Cambia el espacio entre líneas
\usepackage{subfig}                   % Permite agrupar imágenes
\usepackage{tikz}                     % Permite dibujar
\usepackage{ulem}                     % Permite tachar, subrayar, etc
\usepackage{url}                      % Permite añadir enlaces
\usepackage{wasysym}                  % Contiene caracteres misceláneos
\usepackage{wrapfig}                  % Permite comprimir imágenes
\usepackage{xcolor}                   % Paquete de colores avanzado

% LIBRERÍAS DEPENDIENTES
\usetikzlibrary{babel}                % Asociado a tikz
\usepackage{chngcntr}                 % Agrega números de secciones a las leyendas
\usepackage{epstopdf}                 % Convierte archivos .eps a pdf
\usepackage{multirow}                 % Agrega nuevas opciones a las tablas
\ifthenelse{                          % Permite añadir enlaces y referencias
\equal{\showborderonlinks}{false}}{   % Links sin recuadro rojo
\usepackage[hidelinks]{hyperref}}{
\usepackage{hyperref}}


% DECLARACIÓN DE FUNCIONES
\newcommand{\throwerror}[2]{
	% Lanza un mensaje de error
	% 	#1	Función del error
	%	#2	Mensaje
	\errmessage{Error: \noexpand#1 #2 (linea \the\inputlineno)}
}
\newcommand{\quotes}[1]{
	% Insertar cita
	% 	#1	Texto
	''#1''
}
\newcommand{\quotesit}[1]{
	% Insertar cita en itálico
	% 	#1	Texto
	\textit{\quotes{#1}}
}
\newcommand{\newemptypage}{
	% Crea una página vacía
	\newpage\null\thispagestyle{empty}\newpage
	\addtocounter{page}{-1}
}
\newcommand{\setcaptionmargincm}[1]{
	% Cambiar el margen
	% 	#1	Margen en centímetros
	\captionsetup{margin=#1cm}
}
\newcommand{\setpagemargincm}[4]{
	% Cambia márgenes de las páginas [cm]
	% 	#1	Margen izquierdo
	%	#2	Margen superior
	%	#3	Margen derecho
	%	#4	Margen inferior
	\newgeometry{left=#1cm, top=#2cm, right=#3cm, bottom=#4cm}
}
\newcommand{\newp}{
	% Inserta nueva línea	
	\hbadness=10000 \vspace{\defaultnewlinesize} \par
}
\newcommand{\newpar}[1]{
	% Insertar párrafo
	% 	#1	Párrafo
	\hbadness=10000 #1 \newp
}
\newcommand{\newparnl}[1]{
	% Insertar párrafo sin nueva línea al final
	% 	#1	Párrafo
	#1 \par
}
\newcommand{\lpow}[2]{
	% Insertar sub-índice, a_b
	% 	#1	Elemento inferior (a)
	%	#2	Elemento superior (b)
	{#1}_{#2}
}
\newcommand{\pow}[2]{
	% Insertar elevado, a^b
	% 	#1	Elemento inferior (a)
	%	#2	Elemento superior (b)
	{#1}^{#2}
}
\newcommand{\fracpartial}[2]{
	% Fracción de derivadas parciales af/ax
	% 	#1	Función a derivar (f)
	%	#2	Variable a derivar (x)
	\frac{\partial #1}{\partial #2}
}
\newcommand{\fracdpartial}[2]{
	% Fracción de derivadas parciales dobles a^2f/ax^2
	% 	#1	Función a derivar (f)
	%	#2	Variable a derivar (x)
	\frac{{\partial}^{2} #1}{\partial {#2}^{2}}
}
\newcommand{\fracnpartial}[3]{
	% Fracción de derivadas parciales en n, a^nf/ax^n
	% 	#1	Función a derivar (f)
	%	#2	Variable a derivar (x)
	%	#3	Orden (n)
	\frac{{\partial}^{#3} #1}{\partial {#2}^{#3}}
}
\newcommand{\fracderivat}[2]{
	% Fracción de derivadas df/dx
	% 	#1	Función a derivar (f)
	%	#2	Variable a derivar (x)
	\frac{\text{d} #1}{\text{d} #2}
}
\newcommand{\fracdderivat}[2]{
	% Fracción de derivadas dobles d^2/dx^2
	% 	#1	Función a derivar (f)
	%	#2	Variable a derivar (x)
	\frac{{\text{d}}^{2} #1}{\text{d} {#2}^{2}}
}
\newcommand{\fracnderivat}[3]{
	% Fracción de derivadas en n d^nf/dx^n
	% 	#1	Función a derivar (f)
	%	#2	Variable a derivar (x)
	%	#3	Orden de la derivada (n)	
	\frac{{\text{d}}^{#3} #1}{\text{d} {#2}^{#3}}
}
\newcommand{\topequal}[2]{
	% Llave superior de equivalencia
	% 	#1	Elemento a igualar
	%	#2	Igualdad
	\overbrace{#1}^{\mathclap{#2}}
}
\newcommand{\underequal}[2]{
	% Llave inferior de equivalencia
	% 	#1	Elemento a igualar
	%	#2	Igualdad
	\underbrace{#1}_{\mathclap{#2}}
}
\newcommand{\topsequal}[2]{
	% Rectángulo superior de equivalencia
	% 	#1	Elemento a igualar
	%	#2	Igualdad
	\overbracket{#1}^{\mathclap{#2}}
}
\newcommand{\undersequal}[2]{
	% Rectángulo inferior de equivalencia
	% 	#1	Elemento a igualar
	%	#2	Igualdad
	\underbracket{#1}_{\mathclap{#2}}
}
\newcommand{\resizeitem}[2]{
	% Crea un resizebox de tamaño textwidth
	% 	#1	Tamaño del nuevo objeto (En textwidth)
	%	#2	Objeto a redimensionar
	\resizebox{#1\textwidth}{!}{#2}
}
\newcommand{\newtitleanum}[1]{
	% Insertar un título sin número
	% 	#1	Título
	\addcontentsline{toc}{section}{#1}
	\section*{#1}
	\ifthenelse{\equal{\showheadertitle}{true}}{
		\fancyhead[L]{\nouppercase{#1}}}{}
	\stepcounter{section}
}
\newcommand{\newtitleanumheadless}[1]{
	% Insertar un título sin número sin alterar el header
	% 	#1	Título
	\addcontentsline{toc}{section}{#1}
	\section*{#1}
	\stepcounter{section}
}
\newcommand{\newsubtitleanum}[1]{
	% Insertar un subtítulo sin número
	% 	#1	Subtítulo
	\addcontentsline{toc}{subsection}{#1}
	\subsection*{#1}
	\stepcounter{subsection}
}
\newcommand{\newsubsubtitleanum}[1]{
	% Insertar un sub-subtítulo sin número
	% 	#1	Subtítulo
	\addcontentsline{toc}{subsubsection}{#1}
	\subsubsection*{#1}
	\stepcounter{subsubsection}
}
\newcommand{\newtitleanumnoi}[1]{
	% Insertar un título sin número sin indexar
	% 	#1	Título
	\section*{#1}
	\ifthenelse{\equal{\showheadertitle}{true}}{
		\fancyhead[L]{\nouppercase{#1}}}{}
	\stepcounter{section}
}
\newcommand{\newtitleanumnoiheadless}[1]{
	% Insertar un título sin número sin indexar sin cambiar el header
	% 	#1	Título
	\section*{#1}
	\ifthenelse{\equal{\showheadertitle}{true}}{
		\fancyhead[L]{\nouppercase{#1}}}{}
	\stepcounter{section}
}
\newcommand{\newsubtitleanumnoi}[1]{
	% Insertar un subtítulo sin número sin indexar
	% 	#1	Subtítulo
	\subsection*{#1}
	\stepcounter{subsection}
}
\newcommand{\newsubsubtitleanumnoi}[1]{
	% Insertar un sub-subtítulo sin número sin indexar
	% 	#1	Sub-subtítulo
	\addcontentsline{toc}{subsubsection}{#1}
	\subsubsection*{#1}
	\stepcounter{subsubsection}
}
\newcommand{\insertequation}[2][]{
	% Insertar una ecuación
	% 	#1	Label (opcional)
	%	#2	Ecuación
	\vspace{-0.1cm}
	\begin{equation}
		\text{#1} #2
	\end{equation}
	\vspace{-0.23cm}
	\par
}
\newcommand{\insertequationcaptioned}[3][]{
	% Insertar una ecuación con leyenda
	% 	#1	Label (opcional)
	%	#2	Ecuación
	%	#3	Caption
	\vspace{0cm}
	\begin{equation}
		\text{#1} #2
	\end{equation}
	\begin{center}
		\vspace{-0.15cm}
		\textit{#3} \par
		\vspace{0.05cm}
	\end{center}
}
\newcommand{\insertequationgathered}[2][]{
	% Insertar una ecuación con el ambiente gather
	% 	#1	Label (opcional)
	%	#2	Ecuación
	\vspace{-0.4cm}
	\begin{gather}
		\text{#1} #2
	\end{gather}
	\par
	\vspace{-0.10cm}
}
\newcommand{\insertequationgatheredcaptioned}[3][]{
	% Insertar una ecuación (gather) con leyenda
	% 	#1	Label (opcional)
	%	#2	Ecuación
	%	#3	Caption
	\vspace{0cm}
	\begin{gather}
		\text{#1} #2
	\end{gather}
	\begin{center}
		\vspace{-0.15cm}
		\textit{#3} \par
	\end{center}
}
\newcommand{\insertequationalign}[2][]{
	% Insertar una ecuación con el ambiente align
	% 	#1	Label (opcional)
	%	#2	Ecuación
	\vspace{-0.4cm}
	\begin{align}
		\text{#1} #2
	\end{align}
	\par
	\vspace{-0.10cm}
}
\newcommand{\insertequationaligncaptioned}[3][]{
	% Insertar una ecuación (align) con leyenda
	% 	#1	Label (opcional)
	%	#2	Ecuación
	%	#3	Caption
	\vspace{0cm}
	\begin{align}
		\text{#1} #2
	\end{align}
	\begin{center}
		\vspace{-0.15cm}
		\textit{#3} \par
	\end{center}
}
\newcommand{\insertimage}[4][]{
	% Insertar una imagen
	% 	#1	Label (opcional)
	%	#2	Dirección de la imagen
	%	#3	Parámetros de la imagen
	%	#4	Caption de la imagen
	\vspace{\defaultmargintopimages}
	\ifx\hfuzz#2\hfuzz
		\throwerror{\insertimage}{}
	\fi
	\begin{figure}[H]
		\centering
		\includegraphics[#3]{\defaultimagefolder#2}
		\ifx\hfuzz#4\hfuzz
			\vspace{0.2cm}
		\else
			\caption{#4 #1}
		\fi
	\end{figure}
	#1
	\vspace{\defaultmarginbottomimages}
}
\newcommand{\insertimageboxed}[4][]{
	% Insertar una imagen con recuadro
	% 	#1	Label (opcional)
	%	#2	Dirección de la imagen
	%	#3	Parámetros de la imagen
	%	#4	Caption de la imagen
	\vspace{\defaultmargintopimages}
	\begin{figure}[H]
		\centering
		\fbox{\includegraphics[#3]{\defaultimagefolder#2}}
		\caption{#4 #1}
	\end{figure}
	\vspace{\defaultmarginbottomimages}
}
\newcommand{\insertimagefixed}[5][]{
	% Insertar una imagen de ancho fijo a la página
	% 	#1	Label (opcional)
	%	#2	Dirección de la imagen
	%	#3	Parámetros de la imagen
	%	#4	Tamaño de la imagen en textwidth
	%	#5	Caption de la imagen
	\vspace{\defaultmargintopimages}
	\begin{figure}[H]
		\centering
		\resizebox{#3\textwidth}{!}{
			\includegraphics[#4]{\defaultimagefolder#2}
		}
		\caption{#5 #1}
	\end{figure}
	\vspace{\defaultmarginbottomimages}
}
\newcommand{\insertimageboxedfixed}[5][]{
	% Insertar una imagen recuadrada de ancho fijo
	% 	#1	Label (opcional)
	%	#2	Dirección de la imagen
	%	#3	Parámetros de la imagen
	%	#4	Tamaño de la imagen en textwidth
	%	#5	Caption de la imagen
	\vspace{\defaultmargintopimages}
	\begin{figure}[H]
		\centering
		\resizebox{#3\textwidth}{!}{
			\fbox{\includegraphics[#4]{\defaultimagefolder#2}}
		}
		\caption{#5 #1}
	\end{figure}
	\vspace{\defaultmarginbottomimages}
}
\newcommand{\insertdoubleimage}[8][]{
	% Insertar una imagen doble
	% 	#1	Label (opcional)
	%	#2	Dirección de la imagen 1
	%	#3	Parámetros de la imagen 1
	%	#4	Caption de la imagen 1
	%	#5	Dirección de la imagen 2
	%	#6	Parámetros de la imagen 2
	%	#7	Caption de la imagen 2
	%	#8	Caption de la imagen doble
	\vspace{\defaultmargintopimages}
	\captionsetup{margin=0.45cm}
	\begin{figure}[H] \centering
		\subfloat[#4]{
			\includegraphics[#3]{\defaultimagefolder#2}}
		\hspace{0.2cm}
		\subfloat[#7]{
			\includegraphics[#6]{\defaultimagefolder#5}}
		\setcaptionmargincm{\defaultcaptionmargin}
		\caption{#8 #1}
	\end{figure}
	\setcaptionmargincm{\defaultcaptionmargin}
	\vspace{\defaultmarginbottomimages}
}
\newcommand{\insertdoubleeqimage}[7][]{
	% Insertar una imagen doble, igual propiedades
	% 	#1	Label (opcional)
	%	#2	Dirección de la imagen 1
	%	#3	Caption de la imagen 1
	%	#4	Dirección de la imagen 2
	%	#5	Caption de la imagen 2
	%	#6	Propiedades de las imágenes
	%	#7 	Caption de la imagen doble
	\insertdoubleimage[#1]{#2}{#6}{#3}{#4}{#6}{#5}{#7}
}
\newcommand{\inserttripleimage}[8][]{
	% Insertar una imagen triple
	% 	#1	Label (opcional)
	%	#2	Dirección de la imagen 1
	%	#3	Parámetros de la imagen 1
	%	#4	Dirección de la imagen 2
	%	#5	Parámetros de la imagen 2
	%	#6	Dirección de la imagen 3
	%	#7	Parámetros de la imagen 3
	%	#8	Caption de la imagen triple
	\vspace{\defaultmargintopimages}
	\captionsetup{margin=0.45cm}
	\begin{figure}[H] \centering
		\subfloat[]{
			\includegraphics[#3]{\defaultimagefolder#2}}
		\hspace{0.1cm}
		\subfloat[]{
			\includegraphics[#5]{\defaultimagefolder#4}}
		\hspace{0.1cm}
		\subfloat[]{
			\includegraphics[#7]{\defaultimagefolder#6}}
		\setcaptionmargincm{\defaultcaptionmargin}
		\caption{#8 #1}
	\end{figure}
	\setcaptionmargincm{\defaultcaptionmargin}
	\vspace{\defaultmarginbottomimages}
}
\newcommand{\inserttripleeqimage}[6][]{
	% Insertar una imagen triple, igual propiedades
	% 	#1	Label (opcional)
	%	#2	Dirección de la imagen 1
	%	#3	Dirección de la imagen 2
	%	#4	Dirección de la imagen 3
	%	#5	Propiedades de las imágenes
	%	#6	Caption de la imagen triple
	\inserttripleimage[#1]{#2}{#5}{#3}{#5}{#4}{#5}{#6}
}
\newcommand{\insertquadimage}[7][]{
	% Insertar una imagen cuádruple, igual propiedades
	% 	#1	Label (opcional)
	%	#2	Dirección de la imagen 1
	%	#3	Dirección de la imagen 2
	%	#4	Dirección de la imagen 3
	%	#5	Dirección de la imagen 4
	%	#6	Propiedades de las imágenes
	%	#7	Caption de la imagen cuádruple
	\vspace{\defaultmargintopimages}
	\captionsetup{margin=0.45cm}
	\begin{figure}[H] \centering
		\subfloat[]{
			\includegraphics[#6]{\defaultimagefolder#2}}
		\hspace{0.1cm}
		\subfloat[]{
			\includegraphics[#6]{\defaultimagefolder#3}}
		\hspace{0.1cm}
		\subfloat[]{
			\includegraphics[#6]{\defaultimagefolder#4}}
		\hspace{0.1cm}
		\subfloat[]{
			\includegraphics[#6]{\defaultimagefolder#5}}
		\setcaptionmargincm{\defaultcaptionmargin}
		\caption{#7 #1}
	\end{figure}
	\setcaptionmargincm{\defaultcaptionmargin}
	\vspace{\defaultmarginbottomimages}
}
\newcommand{\insertpentaimage}[8][]{
	% Insertar una imagen quíntuple, igual propiedades
	% 	#1	Label (opcional)
	%	#2	Dirección de la imagen 1
	%	#3	Dirección de la imagen 2
	%	#4	Dirección de la imagen 3
	%	#5	Dirección de la imagen 4
	%	#6	Dirección de la imagen 5
	%	#7	Propiedades de las imágenes
	%	#8	Caption de la imagen quíntuple
	\vspace{\defaultmargintopimages}
	\captionsetup{margin=0.45cm}
	\begin{figure}[H] \centering
		\subfloat[]{
			\includegraphics[#7]{\defaultimagefolder#2}}
		\hspace{0.1cm}
		\subfloat[]{
			\includegraphics[#7]{\defaultimagefolder#3}}
		\hspace{0.1cm}
		\subfloat[]{
			\includegraphics[#7]{\defaultimagefolder#4}}
		\hspace{0.1cm}
		\subfloat[]{
			\includegraphics[#7]{\defaultimagefolder#5}}
		\hspace{0.1cm}
		\subfloat[]{
			\includegraphics[#7]{\defaultimagefolder#6}}
		\setcaptionmargincm{\defaultcaptionmargin}
		\caption{#8 #1}
	\end{figure}
	\setcaptionmargincm{\defaultcaptionmargin}
	\vspace{\defaultmarginbottomimages}
}
\newcommand{\insertimageleft}[5][]{
	% Insertar una imagen a la izquierda
	% 	#1	Label (opcional)
	%	#2	Dirección de la imagen
	%	#3	Ancho de la imagen (en textwidth)
	%	#4	Caption de la imagen
	%	#5	Altura en líneas de la imagen
	\begin{wrapfigure}[#5]{l}{#3\textwidth}
		\setcaptionmargincm{0}
		\vspace{\defaultmarginfloatimages}
		\centering
		\includegraphics[width=\linewidth]{\defaultimagefolder#2}
		\caption{#4 #1}
		\setcaptionmargincm{\defaultcaptionmargin}
	\end{wrapfigure}
}
\newcommand{\insertimageright}[5][]{
	% Insertar una imagen a la derecha
	% 	#1	Label (opcional)
	%	#2	Dirección de la imagen
	%	#3	Ancho de la imagen (en textwidth)
	%	#4	Caption de la imagen
	%	#5	Altura en líneas de la imagen
	\begin{wrapfigure}[#5]{r}{#3\textwidth}
		\setcaptionmargincm{0}
		\vspace{\defaultmarginfloatimages}
		\centering
		\includegraphics[width=\linewidth]{\defaultimagefolder#2}
		\caption{#4 #1}
		\setcaptionmargincm{\defaultcaptionmargin}
	\end{wrapfigure}
}


% DECLARACIÓN DE AMBIENTES Y ESTILOS
\definecolor{dkgreen}{rgb}{0,0.6,0}
\definecolor{gray}{rgb}{0.5,0.5,0.5}
\definecolor{mauve}{rgb}{0.58,0,0.82}
\definecolor{mygreen}{rgb}{0,0.6,0}
\definecolor{mygray}{rgb}{0.5,0.5,0.5}
\definecolor{mymauve}{rgb}{0.58,0,0.82}
\definecolor{codegreen}{rgb}{0,0.6,0}
\definecolor{codegray}{rgb}{0.5,0.5,0.5}
\definecolor{codepurple}{rgb}{0.58,0,0.82}
\definecolor{backcolour}{rgb}{0.95,0.95,0.92}
\newcolumntype{P}[1]{
	% Columna centrada en tablas
	>{\centering\arraybackslash}p{#1}
}
\SelectInputMappings{
	% Definición de acentos
	aacute={á},
	Ntilde={Ñ},
	Euro={€}
}
\lstdefinestyle{C}{
	% Estilo de lenguaje C
	language=C,
	numbers=left,
	stepnumber=1,
	numbersep=5pt,
	backgroundcolor=\color{white},
	showspaces=false,
	showstringspaces=false,
	showtabs=false,
	tabsize=2,
	captionpos=b,
	breaklines=true,
	breakatwhitespace=true,
	title=\lstname
}
\lstdefinestyle{Java}{
	% Estilo de lenguaje Java
	language=Java,
	aboveskip=3mm,
	belowskip=3mm,
	showstringspaces=false,
	columns=flexible,
	basicstyle={\small\ttfamily},
	numbers=left,
	numberstyle=\tiny\color{gray},
	keywordstyle=\color{blue},
	commentstyle=\color{dkgreen},
	stringstyle=\color{mauve},
	breaklines=true,
	breakatwhitespace=true,
	tabsize=3,
	backgroundcolor=\color{backcolour}
}
\lstdefinestyle{Matlab}{
	% Estilo de lenguaje Matlab
	language=Matlab,
	breaklines=true,
	morekeywords={matlab2tikz},
	keywordstyle=\color{blue},
	morekeywords=[2]{1}, keywordstyle=[2]{\color{black}},
	backgroundcolor=\color{backcolour},
	identifierstyle=\color{black},
	stringstyle=\color{mylilas},
	commentstyle=\color{mygreen},
	showstringspaces=false,
	numbers=left,
	showstringspaces=false,
	numberstyle=\tiny\color{gray},
	numbersep=9pt,
	basicstyle={\small\ttfamily},
	tabsize=3,
	breaklines=true,
	aboveskip=3mm,
	belowskip=3mm,
	emph=[1]{for,end,break},emphstyle=[1]\color{red}
}
\lstdefinestyle{Python}{
	% Estilo de lenguaje Python
	language=Python,
	backgroundcolor=\color{backcolour},
	commentstyle=\color{codegreen},
	keywordstyle=\color{magenta},
	numberstyle=\tiny\color{codegray},
	stringstyle=\color{codepurple},
	basicstyle=\footnotesize,
	breakatwhitespace=false,
	breaklines=true,
	captionpos=b,
	keepspaces=true,
	numbers=left,
	numbersep=5pt,
	showspaces=false,
	showstringspaces=false,
	showtabs=false,
	tabsize=3,
	basicstyle={\small\ttfamily}
}
\lstset{literate=
	{á}{{\'a}}1 {é}{{\'e}}1 {í}{{\'i}}1 {ó}{{\'o}}1 {ú}{{\'u}}1
	{Á}{{\'A}}1 {É}{{\'E}}1 {Í}{{\'I}}1 {Ó}{{\'O}}1 {Ú}{{\'U}}1
	{à}{{\`a}}1 {è}{{\`e}}1 {ì}{{\`i}}1 {ò}{{\`o}}1 {ù}{{\`u}}1
	{À}{{\`A}}1 {È}{{\'E}}1 {Ì}{{\`I}}1 {Ò}{{\`O}}1 {Ù}{{\`U}}1
	{ä}{{\"a}}1 {ë}{{\"e}}1 {ï}{{\"i}}1 {ö}{{\"o}}1 {ü}{{\"u}}1
	{Ä}{{\"A}}1 {Ë}{{\"E}}1 {Ï}{{\"I}}1 {Ö}{{\"O}}1 {Ü}{{\"U}}1
	{â}{{\^a}}1 {ê}{{\^e}}1 {î}{{\^i}}1 {ô}{{\^o}}1 {û}{{\^u}}1
	{Â}{{\^A}}1 {Ê}{{\^E}}1 {Î}{{\^I}}1 {Ô}{{\^O}}1 {Û}{{\^U}}1
	{œ}{{\oe}}1 {Œ}{{\OE}}1 {æ}{{\ae}}1 {Æ}{{\AE}}1 {ß}{{\ss}}1
	{ű}{{\H{u}}}1 {Ű}{{\H{U}}}1 {ő}{{\H{o}}}1 {Ő}{{\H{O}}}1
	{ç}{{\c c}}1 {Ç}{{\c C}}1 {ø}{{\o}}1 {å}{{\r a}}1 {Å}{{\r A}}1
	{€}{{\EUR}}1 {£}{{\pounds}}1
}


% CONFIGURACIÓN INICIAL DEL DOCUMENTO
\decimalpoint                        % Se define el punto decimal
\counterwithin{equation}{section}    % Añade número de sección a las ecuaciones
\counterwithin{figure}{section}      % Añade número de sección a las figuras
\counterwithin{table}{section}       % Añade número de sección a las tablas
\bibliographystyle{\tiporeferencias} % Estilo APA para las referencias
\setlength{\headheight}{64pt}        % Tamaño de la cabecera sin fancyhdr
\setcounter{tocdepth}{\indexdepth}   % Se ajusta la profundidad del índice
\setcounter{MaxMatrixCols}{20}       % Número máximo de columnas en matrices
\hypersetup{
	pdfauthor={\autordeldocumento},
	pdftitle={\nombredelinforme},
	pdfsubject={\temaatratar},
	pdfkeywords={\nombreuniversidad, \nombredelcurso,
	\codigodelcurso, \localizacionuniversidad},
	pdfcreator={pdfLaTeX, ppizarror},
	pdfproducer={Template LaTeX informe v\templateversion}}
\renewcommand{\baselinestretch}{\defaultinterlind} % Ajuste del entrelineado
\setcaptionmargincm{\defaultcaptionmargin}         % Margen por defecto
\makeatletter
\ifthenelse{\equal{\twocolumnreferences}{true}}{
	% Bibliografía en 2 columnas
	\renewenvironment{thebibliography}[1]
	{\begin{multicols}{2}[\section*{\refname}]
		\@mkboth{\MakeUppercase\refname}{\MakeUppercase\refname}
		\list{\@biblabel{\@arabic\c@enumiv}}
		{\settowidth\labelwidth{\@biblabel{#1}}
			\leftmargin\labelwidth
			\advance\leftmargin\labelsep
			\@openbib@code
			\usecounter{enumiv}
			\let\p@enumiv\@empty
			\renewcommand\theenumiv{\@arabic\c@enumiv}}
		\sloppy
		\clubpenalty 4000
		\@clubpenalty \clubpenalty
		\widowpenalty 4000
		\sfcode`\.\@m}
		{\def\@noitemerr
		{\@latex@warning{Ambiente `thebibliography' no definido}}
		\endlist\end{multicols}}}{}
\makeatother
	

% PORTADA
\begin{document}
\newpage
\renewcommand{\thepage}{\nombrepaginaportada}
\setpagemargincm{\defaultpagemarginleft}{\defaultfirstpagemargintop}
{\defaultpagemarginright}{\defaultpagemarginbottom}
\pagestyle{fancy}
\fancyhf{}
\fancyhead[L]{
\nombreuniversidad \\ \nombrefacultad \\ \departamentouniversidad}
\fancyhead[R]{
\includegraphics[scale=\imagendeldepartamentoescl]{\imagendeldepartamento}}
\ifthenelse{\equal{\codigocursoenportada}{true}}{
	\vspace*{3cm}
	\begin{center}
		\huge {\nombredelcurso} \\
		\vspace{0.3cm}
		\large {Código del curso: \codigodelcurso} \\
		\vspace{1.5cm}
		\Huge {\nombredelinforme} \\
		\vspace{0.3cm}
		\large {\temaatratar}
	\end{center}
}{
	\vspace*{5cm}
	\begin{center}
		\huge {\nombredelcurso} \\
		\vspace{1cm}
		\Huge {\nombredelinforme} \\
		\vspace{0.3cm}
		\large {\temaatratar}
	\end{center}
}
\vfill
\tablaintegrantes


% CONFIGURACIÓN DE PÁGINA Y ENCABEZADOS
\newpage
\pagenumbering{Roman}
\setcounter{page}{1}
\setcounter{footnote}{1}
\setpagemargincm{\defaultpagemarginleft}{\defaultpagemargintop}
{\defaultpagemarginright}{\defaultpagemarginbottom}
\def\arraystretch{\tablepadding} % Se ajusta el padding de las tablas
\renewcommand{\sectionmark}[1]{\markboth{#1}{}} % Se modifica el estilo del header
\renewcommand{\listfigurename}{\nomltfiguras} % Nombre del índice de figuras
\renewcommand{\listtablename}{\nomlttablas} % Nombre del índice de tablas
\renewcommand{\contentsname}{\nomltcontend} % Nombre del índice
\renewcommand{\lstlistlistingname}{\nomltcodfuente} % Nombre índice código fuente
\renewcommand{\tablename}{\nomltwtablas} % Nombre de la leyenda de las tablas
\renewcommand{\figurename}{\nomltwfigura} % Nombre de la leyenda de las figuras
\renewcommand{\lstlistingname}{\nomltsrc} % Nombre leyenda del código fuente
\pagestyle{fancy} \fancyhf{} % Se crean los headers y footers
\ifthenelse{\equal{\showheadertitle}{true}}{ % Header izq, nombre sección
	\fancyhead[L]{\nouppercase{\rightmark}}
}{}
\fancyhead[R]{\small \rm \nouppercase{\thepage}} % Header der, número página
\ifthenelse{\equal{\showfooter}{true}}{
	\fancyfoot[L]{
		\small \rm \textit{\nombredelinforme} % Footer izq, título del informe
	}
	\fancyfoot[R]{
		\small \rm \textit{\codigodelcurso \ \nombredelcurso} % Footer der, curso
	}
	\renewcommand{\footrulewidth}{0.5pt} % Ancho de la barra del footer
}{}
\renewcommand{\headrulewidth}{0.5pt} % Ancho de la barra del header
\sectionfont{
	\tipofuentetitulo \etipofuentetitulo \selectfont % Tipo de título para abstract
}


% ========================= RESUMEN O ABSTRACT =========================
\newtitleanumheadless{Resumen}

% Ejemplo de dos párrafos en latín, esta linea puede borrarse sin problema
\lipsum[1] \newp \lipsum[3]


% TABLA DE CONTENIDOS - ÍNDICE
\newpage
\sectionfont{\tipofuentetituloi \etipofuentetituloi \selectfont}
\subsectionfont{\tipofuentesubtituloi \etipofuentesubtituloi \selectfont}
\subsubsectionfont{\tipofuentesubsubtituloi \etipofuentesubsubtituloi \selectfont}
\ifthenelse{\equal{\showindexofcontents}{true}}{\tableofcontents}{} % Contenido
\ifthenelse{\equal{\showindexoffigures}{true}}{\listoffigures}{} % Figuras
\ifthenelse{\equal{\showindexoftables}{true}}{\listoftables}{} % Tablas
\ifthenelse{\equal{\showindexofsourcecode}{true}}{\lstlistoflistings}{} % Código


% CONFIGURACIONES FINALES - INICIO DE LAS SECCIONES
\newpage
\ifthenelse{
	\equal{\showheadertitle}{true}}{\fancyhead[L]{\nouppercase{\leftmark}}}{}
\sectionfont{\tipofuentetitulo \etipofuentetitulo \selectfont}
\subsectionfont{\tipofuentesubtitulo \etipofuentesubtitulo \selectfont}
\subsubsectionfont{\tipofuentesubsubtitulo \etipofuentesubsubtitulo \selectfont}
\renewcommand{\thepage}{\arabic{page}}
\setcounter{page}{1}
\setcounter{section}{0}
\setcounter{footnote}{0}


% ======================== INICIO DEL DOCUMENTO ========================

% Template:     Informe/Reporte LaTeX
% Documento:    Archivo de ejemplo
% Versión:      6.1.1 (05/12/2018)
% Codificación: UTF-8
%
% Autor: Pablo Pizarro R. @ppizarror
%        Facultad de Ciencias Físicas y Matemáticas
%        Universidad de Chile
%        pablo.pizarro@ing.uchile.cl, ppizarror.com
%
% Manual template: [https://latex.ppizarror.com/Template-Informe/]
% Licencia MIT:    [https://opensource.org/licenses/MIT/]

% ------------------------------------------------------------------------------
% NUEVA SECCIÓN
% ------------------------------------------------------------------------------
% Las secciones se inician con \section, si se quiere una sección sin número se
% pueden usar las funciones \sectionanum (sección sin número) o la función
% \sectionanumnoi para crear el mismo título sin numerar y sin aparecer en el índice.
\section{Informes con \LaTeX}

	% SUB-SECCIÓN
	% Las sub-secciones se inician con \subsection, si se quiere una sub-sección sin
	% número se pueden usar las funciones \subsectionanum (nuevo subtítulo sin numeración)
	% o la función \subsectionanumnoi para crear el mismo subtítulo sin numerar y sin
	% aparecer en el índice
	\subsection{Una breve introducción}
		
		Este es un párrafo, puede contener múltiples \quotes{Expresiones} así como fórmulas o referencias\footnote{Las referencias se hacen utilizando la expresión \texttt{\textbackslash label}\{etiqueta\}.} a fórmulas como \eqref{eqn:identidad-imposible}. A continuación se muestra un ejemplo de inserción de imágenes o figuras (como la Figura \ref{img:testimage}) con el comando \href{https://latex.ppizarror.com/informe.html#hlp-imagen}{\texttt{\textbackslash insertimage}}:

		% Para insertar una imagen se puede usar la función \insertimage la cual
		% toma un primer parámetro opcional para definir una etiqueta (dentro de
		% los corchetes), luego toma la dirección de la imagen, sus parámetros
		% (en este caso se definió la escala de 0.15) y una leyenda opcional.
		\insertimage[\label{img:testimage}]{ejemplos/test-image.png}{scale=0.15}{Where are you? de \quotes{Internet}.}

		A continuación\footnote{Como se puede observar las funciones \texttt{\textbackslash insert...} añaden un párrafo automáticamente.} se muestra un ejemplo de inserción de ecuaciones simples con el comando \href{https://latex.ppizarror.com/informe.html#hlp-formulae}{\texttt{\textbackslash insertequation}}:

		% Se inserta una ecuación, el primer parámetro entre [] es opcional
		% (permite identificar con una etiqueta para poder referenciarlo después
		% con \ref), seguido de aquello se escribe la ecuación en modo bruto sin signos $.
		\insertequation[\label{eqn:identidad-imposible}]{\pow{a}{k}=\pow{b}{k}+\pow{c}{k} \quad \forall k>2}

		% Se añade párrafo de prueba. Notar que no se requiere añadir un salto
		% de línea después de insertar una ecuación.
		\lipsum[75]

		% Los párrafos se pueden añadir con \newp, esta función se hizo para
		% evitar errores y warnings por parte del compilador de LaTeX.
		\newp Este es un nuevo párrafo insertado con el comando \href{https://latex.ppizarror.com/informe.html#hlp-parrafo}{\texttt{\textbackslash newp}}. Si no te gustan los comandos \texttt{\textbackslash newp}, \texttt{\textbackslash newpar} o \texttt{\textbackslash newparnl} simplemente puedes usar los salto de línea convencionales acompañado de \texttt{\textbackslash par}.

	% SUB-SECCIÓN
	\subsection{Añadiendo tablas}

		También puedes usar tablas, ¡Crearlas es muy fácil!. Puedes usar el plugin \href{https://www.ctan.org/tex-archive/support/excel2latex/}{Excel2Latex} \cite{ref2} de Excel para convertir las tablas a \LaTeX\xspace o bien utilizar el \quotes{creador de tablas online} \cite{ref3}.

		% Tabla generada con el plugin Excel2Latex
		\begin{table}[htbp]
			\centering
			\caption{Ejemplo de tablas.}
			\begin{tabular}{ccc}
				\hline
				\textbf{Columna 1} & \textbf{Columna 2} & \textbf{Columna 3} \bigstrut\\
				\hline
				$\omega$ & $\nu$ & $\delta$ \bigstrut[t]\\
				$\beta$ & $\gamma$ & $\epsilon$ \\
				$\Phi$ & $\Theta$ & $\varSigma$ \bigstrut[b]\\
				\hline
			\end{tabular}
			\label{tab:tabla-1}
		\end{table}


% ------------------------------------------------------------------------------
% NUEVA SECCIÓN
% ------------------------------------------------------------------------------
\newpage
\section{Aquí un nuevo tema}

	% SUB-SECCIÓN
	\subsection{Haciendo informes como un profesional}

		% Se inserta una imagen flotante en la izquierda del documento con
		% \insertimageleft, al igual que las demás funciones, el primer parámetro
		% es opcional, luego viene la ubicación de la imagen, seguido de la escala
		% (un 30% del ancho de página) y por último su leyenda. Para insertar una
		% imagen flotante en la derecha se utiliza \insertimageright usando los
		% mismos parámetros.
		\insertimageleft[\label{img:imagen-izquierda}]{ejemplos/test-image-wrap}{0.3}{Apolo flotando a la izquierda.}

		\lipsum[1]

		% Párrafos de ejemplo
		\newp \lipsum[115]
		\newp \lipsum[2]

		% Agrega una ecuación con leyenda
		\insertequationcaptioned[\label{eqn:formulasinsentido}]{\int_{a}^{b} f(x) \dd{x} = \fracnpartial{f(x)}{x}{\eta} \cdotp \textstyle \sum_{x=a}^{b} f(x)\cancelto{1+\frac{\epsilon}{k}}{(1+\Delta x)}}{Ecuación sin sentido.}

		% Aquí no es necesario usar \newp dado que todas las funciones \insert... añaden un párrafo nuevo por defecto
		\lipsum[115]

		\newp \lipsum[4]

	% Inserta un subtítulo sin número
	\subsection{Otros párrafos más normales}

		% Párrafos
		\lipsum[7]
		\newp \lipsum[2]

		% Se inserta una ecuación larga con el entorno gathered (1 solo número de ecuación)
		\insertgathered[\label{eqn:eqn-larga}]{
			\lpow{\Lambda}{f} = \frac{L\cdot f}{W} \cdot \frac{\pow{\lpow{Q}{e}}{2}}{8 \pow{\pi}{2} \pow{W}{4} g} + \sum_{i=1}^{l} \frac{f \cdot \big( M - d\big)}{l \cdot W} \cdot \frac{\pow{\big(\lpow{Q}{e}- i\cdot Q\big)}{2}}{8 \pow{\pi}{2} \pow{W}{4} g}\\
			Q_e = 2.5Q \cdot \int_{0}^{e} V(x) \dd{x} + \aasin{ \bigg(1+\frac{1}{1-e}\bigg) }
		}

		% Nuevo párrafo
		\lipsum[4]

		% Se inserta un multicols, con esto se pueden escribir párrafos en varias columnas
		\begin{multicols}{2}

			% Párrafo 1
			\lipsum[4]

			% Ecuación encerrada en una caja
			\insertequation[]{ \boxed{f(x) = \fracdpartial{u}{t}} }

			% Párrafo 2 del multicols
			\lipsum[1]

		\end{multicols}

	% SUB-SECCIÓN
	\subsection{Ejemplos de inserción de código fuente}

		% A continuación se crea una función auxiliar, esta es una herramienta
		% extremadamente importante y muy útil. Esta función de ejemplo toma dos
		% parámetros, uno es el lenguaje del código fuente, el segundo el
		% identificador en el manual.
		\newcommand{\insertsrcmanual}[2]{\href{https://latex.ppizarror.com/informe.html\#hlp-srccode\&srctype=#1}{#2}}

		El template permite la inserción de los siguientes lenguajes de programación de forma nativa: \insertsrcmanual{bash}{bash}, \insertsrcmanual{c}{C}, \insertsrcmanual{csharp}{C\#}, \insertsrcmanual{cpp}{C++}, \insertsrcmanual{cuda}{cuda}, \insertsrcmanual{docker}{DOCKER}, \insertsrcmanual{html5}{HTML5}, \insertsrcmanual{java}{Java}, \insertsrcmanual{js}{Javascript}, \insertsrcmanual{json}{JSON}, \insertsrcmanual{kotlin}{Kotlin}, \insertsrcmanual{latex}{LaTeX}, \insertsrcmanual{matlab}{Matlab}, \insertsrcmanual{perl}{Perl}, \insertsrcmanual{php}{PHP}, \insertsrcmanual{plaintext}{Texto plano}, \insertsrcmanual{pseudocode}{Pseudocódigo}, \insertsrcmanual{python}{Python}, \insertsrcmanual{ruby}{Ruby}, \insertsrcmanual{scala}{Scala}, \insertsrcmanual{sql}{SQL} y \insertsrcmanual{xml}{XML}. Para insertar un código fuente se debe usar el entorno \texttt{sourcecode}, o el entorno \texttt{sourcecodep} si es que se quiere utilizar parámetros adicionales. \newp

		A continuación se presenta un ejemplo de inserción de código fuente en Python (Código \ref{codigo-python}), Java (Código \ref{codigo-java}) y Matlab (Código \ref{codigo-matlab}):

% Se define el lenguaje del código. Cuidado: Los códigos en LaTeX son sensibles a las tabulaciones y espacios en blanco
\begin{sourcecode}[\label{codigo-python}]{python}{Ejemplo en Python.}
import numpy as np

def incmatrix(genl1, genl2):
	m = len(genl1)
	n = len(genl2)
	M = None # Comentario 1
	VT = np.zeros((n*m, 1), int) # Comentario 2
\end{sourcecode}

\begin{sourcecode}[\label{codigo-java}]{java}{Ejemplo en Java.}
import java.io.IOException;
import javax.servlet.*;

// Hola mundo
public class Hola extends GenericServlet {
	public void service(ServletRequest request, ServletResponse response)
	throws ServletException, IOException{
		response.setContentType("text/html");
		PrintWriter pw = response.getWriter();
		pw.println("Hola, mundo!");
		pw.close();
	}
}
\end{sourcecode}

\begin{sourcecode}[\label{codigo-matlab}]{matlab}{Ejemplo en Matlab.}
% Se crea gráfico
f = figure(1); hold on;
movegui(f, 'center');
xlabel('td/Tn'); ylabel('FAD=Umax/Uf0');
title('Espectro de pulso de desplazamiento');

for j = 1:length(BETA)
	fad = ones(1, NDATOS); % Arreglo para el FAD
	for i = 1:NDATOS
		[t, u_t, ~, ~] = main(BETA(j), r(i), M, K, F0, 0);
		fad(i) = max(abs(u_t)) / uf0;
	end
end
\end{sourcecode}

	% SUB-SECCIÓN
	\subsection{Añadir múltiples imágenes}

	El template ofrece el entorno \href{https://latex.ppizarror.com/informe.html#hlp-images}{\texttt{images}} que permite insertar múltiples imágenes de una manera muy sencilla \footnote{Desde la versión \texttt{6.0.0} no se da soporte a las funciones de inserción de imágenes múltiples \texttt{\textbackslash insertdoubleimage}, \texttt{\textbackslash insertdoubleeqimage},\texttt{\textbackslash inserttripleimage}, \texttt{\textbackslash inserttripleeqimage}, \texttt{\textbackslash insertquadimage}, \texttt{\textbackslash insertpentaimage} y \texttt{\textbackslash inserthexaimage}.}. Para crear imágenes múltiples se deben usar las siguientes instrucciones:

\begin{sourcecode}{latex}{}
\begin{images}[\label{imagenmultiple}]{Ejemplo de imagen múltiple.}
	\addimage{ejemplos/test-image}{width=6.5cm}{Ciudad.}
	\addimage{ejemplos/test-image-wrap}{width=5cm}{Apolo.}
	\addimagenewline
	\addimage{ejemplos/test-image}{width=12cm}{Ciudad más grande.}
\end{images}
\end{sourcecode}

	Obteniendo así:

	\begin{images}{Ejemplo de imagen múltiple.}
		\addimage{ejemplos/test-image}{width=6.5cm}{Ciudad.}
		\addimage{ejemplos/test-image-wrap}{width=5cm}{Apolo.}
		\addimagenewline
		\addimage{ejemplos/test-image}{width=12cm}{Ciudad más grande.}
	\end{images}


% ------------------------------------------------------------------------------
% NUEVA SECCIÓN
% ------------------------------------------------------------------------------
% Inserta una sección sin número
\sectionanum{Más ejemplos}

	% Inserta un subtítulo sin número
	\subsectionanum{Listas y Enumeraciones}

		Hacer listas enumeradas con \LaTeX\ es muy fácil con el template\footnote{También puedes revisar el manual de las enumeraciones en \url{http://www.texnia.com/archive/enumitem.pdf}.}, para ello debes usar el comando \texttt{\textbackslash begin\{enumerate\}}, cada elemento comienza por \texttt{\textbackslash item}, resultando así:

		\begin{enumerate}
			\item Grecia
			\item Abracadabra
			\item Manzanas
		\end{enumerate}

		También se puede cambiar el tipo de enumeración, se pueden usar letras, números romanos, entre otros. Esto se logra cambiando el \textbf{label} del objeto \texttt{enumerate}. A continuación se muestra un ejemplo usando letras con el estilo \texttt{\textbackslash alph} \footnote{Con \texttt{\textbackslash Alph} las letras aparecen en mayúscula.}, números romanos con \texttt{\textbackslash roman} \footnote{Con \texttt{\textbackslash Roman} los números romanos salen en mayúscula.} o números griegos con \texttt{\textbackslash greek}\footnote{Una característica propia del template, con \texttt{\textbackslash Greek} las letras griegas están escritas en mayúscula.}:

		\begin{multicols}{3}
			\begin{enumeratebf}[label=\alph*) ] % Fuente en negrita
				\item Peras
				\item Manzanas
				\item Naranjas
			\end{enumeratebf}

			\begin{enumerate}[label=\greek*) ]
				\item Matemáticas
				\item Lenguaje
				\item Filosofía
			\end{enumerate}

			\begin{enumerate}[label=\roman*) ]
				\item Rojo
				\item Café
				\item Morado
			\end{enumerate}
		\end{multicols}

		Para hacer listas sin numerar con \LaTeX\ hay que usar el comando \texttt{\textbackslash begin\{itemize\}}, cada elemento empieza por \texttt{\textbackslash item}, resultando:

		\begin{multicols}{3}
			\begin{itemize}[label={--}]
				\item Peras
				\item Manzanas
				\item Naranjas
			\end{itemize}

			\begin{enumerate}[label={*}]
				\item Rojo
				\item Café
				\item Morado
			\end{enumerate}

			\begin{itemize}
				\item Árboles
				\item Pasto
				\item Flores
			\end{itemize}
		\end{multicols}

	% Inserta un subtítulo sin número
	\subsectionanum{Otros}

		Recuerda revisar el manual de todas las funciones y configuraciones de este template visitando el siguiente link: \url{https://latex.ppizarror.com/Template-Informe/}. Si necesitas una ayuda muy específica sobre el template, o si tienes alguna sugerencia, me puedes enviar un correo a \insertemail{pablo.pizarro@ing.uchile.cl}.


% ------------------------------------------------------------------------------
% REFERENCIAS (ESTILO BIBTEX), revisar configuración \stylecitereferences
% ------------------------------------------------------------------------------
\newpage % Salto de página
\begin{references}
	\bibitem{ref1}
	Template Informe en \LaTeX.
	\textit{¡Revisa el manual online de este template!} \\
	\url{https://latex.ppizarror.com/Template-Informe/}

	\bibitem{ref2}
	Excel2Latex.
	\textit{Importa de forma sencilla tus tablas de Excel a \LaTeX.} \\
	\url{https://www.ctan.org/tex-archive/support/excel2latex/}

	\bibitem{ref3}
	Overleaf.
	\textit{Uno de los mejores editores online para \LaTeX, renovado con su versión 2.0.} \\
	\href{https://es.overleaf.com}{\texttt{https://es.overleaf.com/}}
\end{references}


% ------------------------------------------------------------------------------
% ANEXO
% ------------------------------------------------------------------------------
\newpage
\begin{anexo}
	\section{Cálculos realizados}

		\subsection{Metodología}
			\lipsum[1]

			% Imagen, se numerará automáticamente con la letra del anexo según
			% la configuración \appendixindepobjnum
			\insertimage[\label{img:anexo-2}]{ejemplos/test-image.png}{scale=0.2}{Imagen en anexo.}

		\subsection{Resultados}
			\lipsum[10]

			% Tablas
			\begin{table}[htbp]
				\centering
				\caption{Tabla de cálculo.}
				\begin{tabular}{ccc}
					\hline
					\textbf{Elemento} & $\epsilon_i$ & \boldmath{}\textbf{Valor}\unboldmath{} \bigstrut\\
					\hline
					A     & 10    & 3,14$\pi$ \bigstrut[t]\\
					B     & 20    & 6 \\
					C     & 30    & 7 \\
					\end{tabular}
				\label{tab:anexo-1}
			\end{table}

	\newpage
	\section{Más cálculos}

		% Párrafo
		\lipsum[1]

		% Nuevo párrafo con identación
		\newp \lipsum[4]

		% Tabla de encuestas
		\begin{table}[htbp]
			\centering
			\caption{Resultados encuesta.}
			\begin{tabular}{ccc}
				\hline
				\textbf{Herramienta} & \textbf{Nota} & \textbf{Recomendado} \bigstrut\\
				\hline
				\LaTeX & 100\% & Si $\checkmark$ \\
				Microsoft Word \textsuperscript{\textregistered} & 0\% & No $\frownie$\\
			\end{tabular}
			\label{tab:anexo-2}
		\end{table}

\end{anexo}
 % Se incluye el ejemplo del documento, se puede borrar


% FIN DEL DOCUMENTO
\end{document}