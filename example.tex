% Template:     Informe/Reporte LaTeX
% Documento:    Archivo de ejemplo
% Versión:      3.5.2 (13/05/2017)
% Codificación: UTF-8
%
% Autor: Pablo Pizarro R.
%        Facultad de Ciencias Físicas y Matemáticas
%        Universidad de Chile
%        pablo.pizarro@ing.uchile.cl, ppizarror.com
%
% Sitio web del proyecto: [http://ppizarror.com/Template-Informe/]
% Licencia: MIT           [https://opensource.org/licenses/MIT]

% NUEVA SECCIÓN
% Las secciones se inician con \section, si se quiere una sección sin "número" se pueden usar las funciones \sectionanum (sección sin número) o la función \sectionanumnoi para crear el mismo título sin numerar y sin aparecer en el índice
\section{Informes con \LaTeX}
	
	% SUB-SECCIÓN
	% Las sub-secciones se inician con \subsection, si se quiere una sub-sección sin "número" se pueden usar las funciones \subsectionanum (nuevo subtítulo sin numeración) o la función \subsectionanumnoi para crear el mismo subtítulo sin numerar y sin aparecer en el índice
	\subsection{Una breve introducción}
		
		% Esta función sirve para rellenar con un párrafo
		\lipsum[4]
		
		% Se inserta una ecuación, el primer parámetro entre corchetes es opcional (permite identificar con una etiqueta para poder referenciarlo después con \ref), seguido de aquello se escribe la ecuación en modo bruto sin signos peso
		\insertequation[\label{eqn:identidad-imposible}]{\pow{a}{k}=\pow{b}{k}+\pow{c}{k} \quad \forall k>2}
		
		% Los párrafos se pueden añadir con \newpar, esta función se hizo para evitar errores y warnings por parte del compilador. No es necesario utilizar estas funciones después de una tabla o una imagen dado que estas ya insertan un párrafo por defecto
		Este es un párrafo, \textbf{\textbackslash newp} puede contener múltiples ''Expresiones'' así como fórmulas o referencias\footnote{Las referencias se hacen utilizando la expresión \texttt{\textbackslash label}\{etiqueta\}} a fórmulas como \eqref{eqn:formulasinsentido}. A continuación se muestra un ejemplo de inserción de imágenes (como la Figura \ref{img:testimage}):
		
		% Para insertar una imagen se puede usar la función \insertimage la cual toma un primer parámetro opcional para definir una etiqueta (dentro de los corchetes), luego toma la dirección de la imagen, sus parámetros (en este caso se definió la escala de 0.17) y una leyenda
		\insertimage[\label{img:testimage}]{ejemplos/test-image.png}{scale=0.17}{Where are you? de ``Internet``}
		
		\newparnl{Este es un párrafo sin nueva línea, si no te gustan los comandos \textbf{\textbackslash newp}, \textbf{\textbackslash newpar} o \textbf{\textbackslash newparnl} simplemente puedes usar los salto de línea convencionales. Además puedes editar las funciones, definidas en el archivo \texttt{lib/functions.tex}.}
		
	% SUB-SECCIÓN
	\subsection{Tablas!}
		
		\newp También puedes usar tablas, insertarlas es muy fácil, puedes usar el plugin de Excel \href{https://www.ctan.org/tex-archive/support/excel2latex/}{Excel2Latex} para convertir las tablas a \LaTeX o bien utilizar el ``creador de tablas online`` \textsuperscript{\cite{ref3}}.
		
		\begin{longtable}{ccc}
			\caption{Esta es una tabla que se ``corta`` en varias páginas si es que le falta espacio.}\label{tabl:foo}\\
			\hline
			Columna 1 & Columna 2 & Columna 3\\\hline
			\endfirsthead
			\hline
			Columna 1 & Columna 2 & Columna 3\\
			\hline
			\endhead
			\hline
			\endfoot
			\hline
			\endlastfoot
			$\omega$ & $\nu$ & $\delta$\\
			$\partial$ & $\nabla$ & $\mho$\\
			$\beta$ & $\gamma$ & $\epsilon$\\
			$\varepsilon$ & $\upsilon$ & $\varphi$\\
			$\Phi$ & $\Theta$ & $\varSigma$\\
			$\omega$ & $\nu$ & $\delta$\\
			$\partial$ & $\nabla$ & $\mho$\\
		\end{longtable}


% NUEVA SECCIÓN
\section{Aquí un nuevo tema}
	
	% SUB-SECCIÓN
	\subsection{Haciendo informes como un profesional}
		
		% Se inserta una imagen flotante en la izquierda del documento con \insertimageleft, al igual que las demás funciones, el primer parámetro es opcional, luego viene la ubicación de la imagen, seguido de la escala, el número de filas que utilizará y por último su leyenda. Para insertar una imagen flotante en la derecha se utiliza \insertimageright usando los mismos parámetros
		\insertimageleft[\label{img:imagen-izquierda}]{ejemplos/test-image-wrap}{0.27}{15}{Apolo}
		
		Test es una palabra inglesa aceptada por la Real Academia Española (RAE). Este concepto hace referencia a las pruebas destinadas a evaluar conocimientos, aptitudes o funciones. La palabra test puede utilizarse como sinónimo de examen. Los exámenes son muy frecuentes en el ámbito educativo ya que permiten evaluar los conocimientos adquiridos por los estudiantes. Los exámenes pueden ser orales o escritos.
		
		\newp \lipsum[115]
		
		% Se inserta un nuevo párrafo de modo inteligente, evita errores de hbox, crea un nuevo par y añade un espacio en blanco
		\newp \lipsum[2]
		
		% Agrega una ecuación con leyendas
		\insertequationcaptioned[\label{eqn:formulasinsentido}]{\int_{a}^{b} f(x) dx = \fracnpartial{f(x)}{x}{\eta} \cdotp \textstyle \sum_{x=a}^{b} f(x)\cancelto{1+\frac{\epsilon}{k}}{(1+\Delta x)}}{Ecuación sin sentido}
		
		\lipsum[115]

		\newp \lipsum[4]
		
	% SUB-SECCIÓN
	\subsection{Ejemplos de inserción de código fuente}
		
		A continuación se presenta un ejemplo de inserción de código fuente en Python \footnote{El mejor lenguaje del mundo}, Java y Matlab:
		
% Se define el lenguaje del código, cuidado: Los códigos en LaTeX son sensibles a las tabulaciones y espacios en blanco
\lstset{style=Python}
\begin{lstlisting}[caption=Ejemplo en Python]
import numpy as np

def incmatrix(genl1, genl2):
	m = len(genl1)
	n = len(genl2)
	M = None # to become the incidence matrix
	VT = np.zeros((n*m, 1), int) # dummy variable
\end{lstlisting}

\lstset{style=Java}
\begin{lstlisting}[caption=Ejemplo en Java]
import java.io.IOException; 
import javax.servlet.*;

// Hola mundo
public class Hola extends GenericServlet {
	public void service(ServletRequest request, ServletResponse response)
	throws ServletException, IOException{
		response.setContentType("text/html");
		PrintWriter pw = response.getWriter();
		pw.println("Hola, mundo!");
		pw.close();
	}
}
\end{lstlisting}

\lstset{style=Matlab}
\begin{lstlisting}[caption=Ejemplo en Matlab]
function list = find_coordinates(matrix, value)
	l = size(matrix);
	coordenates_list = cell(l(1)*l(2), 1);
	total = 1;

	% Evaluación de la matriz
	for i=1:l(1)
		for j=1:l(2)
			if matrix(i, j) == value
				coordenates_list{total} = [i j];
				total = total + 1;
			end
		end
	end

end
\end{lstlisting}
	
	% Inserta un subtítulo sin número
	\subsection{Otros párrafos más normales}
		
		% Párrafos con lipsum
		\lipsum[7]
		
		\newp \lipsum[2]
		
		% Se inserta una ecuación larga
		\insertequationalign[\label{eqn:eqn-larga}]{\lpow{\Lambda}{f} = \frac{L\cdot f}{W} \cdot \frac{\pow{\lpow{Q}{e}}{2}}{8 \pow{\pi}{2} \pow{W}{4} g}+ \sum_{i=1}^{l} \frac{f \cdot \big( M - d\big)}{l \cdot W} \cdot \frac{\pow{\big(\lpow{Q}{e}- i\cdot Q\big)}{2}}{8 \pow{\pi}{2} \pow{W}{4} g}}
		
		% Se inserta un multicols, con esto se pueden escribir en varias columnas
		\begin{multicols}{2}
			
			% Párrafo 1
			\lipsum[114]
			
			% Ecuación
			\insertequation[]{ f(x) = \fracdpartial{u}{t} }
			
			% Párrafo 2 del multicols
			\lipsum[5]
			
		\end{multicols}
		
		% Último párrafo de la subsección
		\lipsum[1]


% NUEVA SECCIÓN
% Inserta una sección sin número
\sectionanum{Más ejemplos}
	
	% Inserta un subtítulo sin número
	\subsectionanum{Listas y Enumeraciones}
		
		Hacer listas enumeradas con \LaTeX\ es muy fácil \footnote{También puedes revisar el manual de las enumeraciones en \url{http://www.texnia.com/archive/enumitem.pdf}}, para eso debes usar el comando \texttt{\textbackslash begin\{enumerate\}}, cada elemento empieza por \texttt{\textbackslash item}, resultando:
		
		\begin{enumerate}
			\item Ítem 1
			\item Abracadabra
			\item Manzanas
		\end{enumerate}
		
		También se puede cambiar el tipo de enumeración, se pueden usar letras, números romanos, entre otros. Esto se logra cambiando el \textbf{label} del objeto \texttt{enumerate}. A continuación se muestra un ejemplo usando letras con el estilo \texttt{\textbackslash alph}, números romanos con \texttt{\textbackslash roman} o números griegos con \texttt{\textbackslash greek} \footnote{Una característica propia del template}:
		
		\begin{multicols}{3}
			\begin{enumerate}[label=\alph*) ,font=\bfseries] % Fuente en negrita
				\item Peras
				\item Manzanas
				\item Naranjas
			\end{enumerate}
			
			\begin{enumerate}[label=\roman*) ]
				\item Rojo
				\item Café
				\item Morado
			\end{enumerate}
			
			\begin{enumerate}[label=\greek*) ]
				\item Matemáticas
				\item Lenguaje
				\item Filosofía
			\end{enumerate}
		\end{multicols}
		
		Para hacer listas sin numerar con \LaTeX\ hay que usar el comando \texttt{\textbackslash begin\{itemize\}}, cada elemento empieza por \texttt{\textbackslash item}, resultando:
		
		\begin{multicols}{3}
			\begin{itemize}[label={--}]
				\item Peras
				\item Manzanas
				\item Naranjas
			\end{itemize}
			
			\begin{enumerate}[label={*}]
				\item Rojo
				\item Café
				\item Morado
			\end{enumerate}
			
			\begin{itemize}
				\item Árboles
				\item Pasto
				\item Flores
			\end{itemize}
		\end{multicols}
		
	% Inserta un subtítulo sin número
	\subsectionanum{Otros}
		
		Recuerda revisar el manual de todas las funciones de este template visitando el siguiente link: \url{http://ppizarror.com/Template-Informe/}. Además si necesitas una ayuda muy específica sobre el template me puedes enviar un correo a \href{mailto:pablo.pizarro@ing.uchile.cl}{\texttt{pablo.pizarro@ing.uchile.cl}}.
		
		\newp \textcolor{magenta}{Saludos a todos!} \smiley